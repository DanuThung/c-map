\section{Twists and the HK/QK Correspondence}

Inspired by certain physical constructions derived from T-duality, Swann \cite{Swa2010} invented a general method to associate to a manifold $M$ with a torus action (subject to some conditions) another manifold, called the \emph{twist} of $M$. Moreover, tensors invariant under the action can be carried over from $M$ to its twist in unique fashion. The HK/QK correspondence, as described in \cref{sec:FSfromHKQK}, can be viewed as one instance of this general construction \cite{MS2014,MS2015}. Demonstrating this is the aim of this section.

\section{The Twist Construction}

For our purposes, it will suffice to describe the twist construction for \emph{circle} actions; the interested reader is referred to \cite{Swa2010} for twists of arbitrary-dimensional torus actions. 

Consider a manifold $M$ equipped with an $S^1$-action generated by a vector field $Z\in \mf X(M)$. To apply the twist construction, we must first construct an associated manifold, namely the total space $P$ of an $S^1$-principal bundle $\pi_M:P\to M$. We furthermore equip this principal bundle with a principal $S^1$-connection $\eta\in \Omega^1(P)$ with curvature $F$. Let $X\in \mf X(P)$ denote the vector field generating the principal action. We wish to lift the given action $S^1\action M$ to an action on $P$ which preserves the connection $\theta$ and commutes with the principal circle action.

\begin{prop}
	A lift as above exists if and only if $[\iota_Z F]=0\in H^1(M;\Z)$.
\end{prop}
\begin{myproof}
	Such a lift is specified by a vector field $Z_1\in \mf{X}(M)$, which will be the generator of our action. We then require $L_{Z_1}\theta=0$. Writing $Z_1=\tilde Z+f_1 X$, where $\tilde Z$ denotes the $\theta$-horizontal lift, this translates to
	\begin{equation*}
		0=\iota_{Z_1}\d \theta+\d \iota_{Z_1}\theta=\iota_{Z_1}\pi^*F+\d(\iota_{f_1 X}\theta)
		=\iota_{\tilde Z}\pi^*F+\d f_1 \theta(X)=\pi^*(\iota_Z F)+\d f_1
	\end{equation*}
	and therefore $\d f_1=-\pi^*(\iota_Z F)$. Such an $f_1$ is constant along fibers of $\pi$, since $X(f_1)=\iota_X\d f_1=-\iota_X\pi^*(\iota_ZF)=0$. This means that $f_1=\pi^*f$ for some $f\in C^\infty(M)$, and $\pi^*\d f=\d f_1=-\pi^*(\iota_Z F)$ and therefore $\iota_Z F=-\d f$ and in particular, $[\iota_ZF]=0\in H^1(M;\Z)$.
	
	Since $\operatorname{Lie}(S^1)$ is one-dimensional, $Z_1$ trivially defines a Lie algebra. Choosing $f_1$ as above\footnote{Notice that $f_1$ is only determined up to a real constant; this corresponds to the one-loop deformation parameter in the HK/QK correspondence.}, we obtain a circle actions $S^1_{Z_1}\action P$; we check that it commutes with the principal $S^1$-action. The horizontal component of $[Z_1,X]$, which is measured by its projection to $M$, vanishes because of naturality of the Lie bracket and verticality of $X$. Thus, we only need to investigate the vertical component. We can instead check the vertical component of $[\tilde Z,X]$, since $[f_1X,X]=0$ by constancy of $f_1$ on the fibers. Thus, we compute
	\begin{equation*}
		\theta([\tilde Z,X])=\tilde Z(\theta(X))-(L_{\tilde Z}\theta)(X)
		=-\d \theta(\tilde Z,X)=-\pi^*F(\tilde Z,X)=0
	\end{equation*} 
	where the final step uses verticality of $X$. Finally, if $\iota_Z F=-\d f$ for some $f\in C^\infty(M)$, then $\pi^*(\iota_Z F)=\d f_1$ where $f_1=\pi^*f$, and hence $L_{Z_1}\theta=0$. This establishes the converse statement.
\end{myproof}

\begin{mydef}
	Let $F$ be a closed 2-form on a manifold $M$. A circle action $S^1\action M$ is called \emph{$F$-Hamiltonian} if $\iota_ZF=-\d f$ for some $f\in C^\infty(M)$.
\end{mydef}