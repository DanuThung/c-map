\documentclass{scrartcl}
% % % % % PACKAGES

%General Packages

\usepackage[automark]{scrlayer-scrpage}																	
\usepackage{amsfonts}												%Mathematics fonts
\usepackage{mathtools}												%General mathematics symbols
\usepackage{amsmath}
\usepackage{stmaryrd}												%Extra math symbols
\usepackage{amssymb}												%More symbols
\usepackage{extarrows}												%Extendible arrows
\usepackage{dsfont} 												%Identity matrix symbol
\usepackage{mathrsfs}												%To get mathscr
\usepackage{relsize}												%Scaling symbols with reference to pre-existing symbols
\usepackage{accents}												%Accents on math symbols
\usepackage[T1]{fontenc}											%Accents output improvement
\usepackage[latin1]{inputenc}										%Accents input improvement
\usepackage[english]{babel}
\usepackage{subcaption} 											%Subfigures etc
\usepackage{cancel} 												%Striking through things
\usepackage{setspace}												%line spacing
\usepackage[top=1in,bottom=1in,left=1.25in,right=1.25in]{geometry}	%margins
\usepackage[symbol]{footmisc}										%Some footnote margin thing
\usepackage{enumerate} 												%Numbered lists
\usepackage{booktabs}

%Pictures & TikZ Packages

\usepackage{graphicx}												%Pictures	
\usepackage{epstopdf}												%Converts .eps to .pdf files
\usepackage{tikz}													%TikZ Drawings
\usetikzlibrary{3d,patterns,arrows,bending,arrows.meta,				%TikZ Libraries
	shapes.geometric,knots,intersections,
	decorations.markings,decorations.pathmorphing,
	decorations.pathreplacing}										
\usepackage{tikz-cd}												%Commutative Diagrams


%Mathematics Packages

\usepackage{amsthm}													%Theorems etc

%Referencing

\usepackage[colorlinks=true]{hyperref}								%hyperlinks
\usepackage[noabbrev]{cleveref}										%better croff-refs
\crefname{prop}{proposition}{propositions}
\crefname{lem}{lemma}{lemmata}						


% % % % % CUSTOM COMMANDS

%Derivatives/Differentials

\let\underdot=\d
\newcommand{\od}[2]{\frac{\mathrm{d} #1}{\mathrm{d} #2}}
\newcommand{\odd}[2]{\frac{\mathrm{d}^2 #1}{\mathrm{d} #2^2}}
\newcommand{\p}{\partial}
\newcommand{\pd}[2]{\frac{\partial #1}{\partial #2}}
\newcommand{\pdd}[2]{\frac{\partial^2 #1}{\partial #2^2}}
\newcommand{\fd}[2]{\frac{\delta #1}{\delta #2}}
\renewcommand{\d}{\mathrm{d}}
\newcommand{\dif}{D}


%Common Sets/Spaces

\newcommand{\RP}{\mathbb{R}\mathrm{P}}
\newcommand{\CP}{\mathbb{C}\mathrm{P}}
\newcommand{\HP}{\mathbb{H}\mathrm{P}}
\renewcommand{\P}{\mathbb{P}}
\newcommand{\N}{\mathbb{N}}
\newcommand{\Z}{\mathbb{Z}}
\newcommand{\Q}{\mathbb{Q}}
\newcommand{\R}{\mathbb{R}}
\newcommand{\C}{\mathbb{C}}
\renewcommand{\H}{\mathbb{H}}
\renewcommand{\O}{\mathbb{O}}

%Math-operators



\renewcommand{\Im}{\operatorname{Im}}
\renewcommand{\Re}{\operatorname{Re}}

\DeclareMathOperator{\Graph}{graph}
\DeclareMathOperator{\Gr}{Gr}

\DeclareMathOperator{\im}{im}
\DeclareMathOperator{\rank}{rank}
\DeclareMathOperator{\ord}{ord}
\DeclareMathOperator{\tr}{tr}
\DeclareMathOperator{\incl}{incl}
\DeclareMathOperator{\pr}{proj}
\DeclareMathOperator{\diag}{diag}
\DeclareMathOperator{\Span}{span}
\DeclareMathOperator{\codim}{codim}

\DeclareMathOperator{\Hom}{Hom}
\DeclareMathOperator{\End}{End}
\DeclareMathOperator{\Aut}{Aut}
\DeclareMathOperator{\coker}{coker}
\DeclareMathOperator{\Stab}{Stab}
\DeclareMathOperator{\Diff}{Diff}
\DeclareMathOperator{\Bs}{Bs}
\DeclareMathOperator{\id}{id}
\DeclareMathOperator{\Mat}{Mat}
\newcommand{\Unit}{\mathds{1}}

\DeclareMathOperator{\td}{td}
\DeclareMathOperator{\ch}{ch}
\DeclareMathOperator{\Spin}{Spin}
\newcommand{\Spinc}{\Spin^c}

\DeclareMathOperator{\Ad}{Ad}
\DeclareMathOperator{\ad}{ad}

\DeclareMathOperator{\supp}{supp}
\DeclareMathOperator{\interior}{int}
\DeclareMathOperator{\vol}{vol}

\DeclareMathOperator{\sgn}{sgn}

\DeclareMathOperator{\Tor}{Tor}
\DeclareMathOperator{\Ext}{Ext}
\DeclareMathOperator*{\free}{\scalerel*{\ast}{\scaleobj{1}{\sum}}}

\newcommand{\trans}{\mathrel{\text{\tpitchfork}}}
\makeatletter
\newcommand{\tpitchfork}{%
	\vbox{
		\baselineskip\z@skip
		\lineskip-.52ex
		\lineskiplimit\maxdimen
		\m@th
		\ialign{##\crcr\hidewidth\smash{$-$}\hidewidth\crcr$\pitchfork$\crcr}
	}%
}
\makeatother

%Other

\newcommand{\action}{\curvearrowright}
\newcommand{\rightaction}{\curvearrowleft}

\newcommand{\ubar}[1]{\underaccent{\bar}{#1}}
\def\mathunderline#1#2{\color{#1}\underline{{\color{black}#2}}\color{black}}

\newcommand{\abs}[1]{\left\lvert #1 \right\rvert}
\newcommand{\norm}[1]{\left\lVert #1 \right\rVert}
\newcommand{\expvalue}[1]{\left\langle #1 \right\rangle}

\setlength{\parindent}{0pt}

\newcommand{\mf}[1]{\mathfrak{#1}}
\newcommand{\mc}[1]{\mathcal{#1}}
\newcommand{\ms}[1]{\mathscr{#1}}

\newcommand{\bdy}{\partial}
\newcommand{\pt}{\mathrm{pt}}
\DeclareMathOperator{\Bl}{Bl}


%Theorem Styles

\newtheoremstyle{mythm}% name of the style to be used
{}% measure of space to leave above the theorem. E.g.: 3pt
{}% measure of space to leave below the theorem. E.g.: 3pt
{\slshape}% name of font to use in the body of the theorem
{}% measure of space to indent
{\bfseries\sffamily}% name of head font
{.}% punctuation between head and body
{ }% space after theorem head; " " = normal interword space
{}% Manually specify head
\newtheoremstyle{mydef}% name of the style to be used
{}% measure of space to leave above the theorem. E.g.: 3pt
{}% measure of space to leave below the theorem. E.g.: 3pt
{}% name of font to use in the body of the theorem
{}% measure of space to indent
{\bfseries\sffamily}% name of head font
{.}% punctuation between head and body
{ }% space after theorem head; " " = normal interword space
{}% Manually specify head

\theoremstyle{mythm}
\newtheorem{thm}{Theorem}[section]
\newtheorem{prop}[thm]{Proposition}
\newtheorem{cor}[thm]{Corollary}
\newtheorem{lem}[thm]{Lemma}
\theoremstyle{mydef}
\newtheorem{mydef}[thm]{Definition}
\newtheorem{rem}[thm]{Remark}
\newtheorem{ex}[thm]{Example}
\newtheorem{exer}{Exercise}[subsection]
\newenvironment{myproof}[1][\proofname]{
	\proof[\sffamily\upshape#1]
}{\endproof}

\newcommand{\proofclear}{\hfill \qedsymbol}

% % % % % MISCELLANEOUS STUFF

\clearscrheadfoot
\ihead[]{}
\ohead[]{}
\cfoot[]{\pagemark}
\pagestyle{scrheadings}

\deffootnote[1em]{0em}{1em}{%
	\textsuperscript{\thefootnotemark}%
}
\setfootnoterule{3em}


\newcommand\numberthis{\stepcounter{equation}\tag{\theequation}}


\newenvironment{numberedlist}{\begin{enumerate}[\upshape(i)]}{\end{enumerate}}
\newenvironment{letteredlist}{\begin{enumerate}[\upshape a)]}{\end{enumerate}}

\renewcommand{\thesection}{\arabic{section}}
\renewcommand{\thesubsection}{(\alph{subsection})}
\renewcommand{\thesubsubsection}{(\roman{subsubsection})}
\renewcommand{\autodot}{}

%Inverse diagonal dots:

\makeatletter
\def\Ddots{\mathinner{\mkern1mu\raise\p@
		\vbox{\kern7\p@\hbox{.}}\mkern2mu
		\raise4\p@\hbox{.}\mkern2mu\raise7\p@\hbox{.}\mkern1mu}}
\makeatother


%TikZ

\tikzset{% 
	arrowat/.style={%
		postaction={decorate,decoration={
				markings,
				mark=at position #1 with {\arrow[xshift=2pt]{>}}}}
	}
}

\tikzset{% 
	backarrowat/.style={%
		postaction={decorate,decoration={
				markings,
				mark=at position #1 with {\arrow[xshift=2pt]{<}}}}
	}
}
\title{Notes for Thesis}
\author{}
\date{}
\begin{document}
\maketitle
\tableofcontents

\section{Computations: In Progress}

\subsection{Cohomology Ring of $Z$}

\subsubsection{Additive Structure of \texorpdfstring{$H^*(G_2/SO(4);\Z)$}{the Integral Cohomology of G2 over SO(4)}}

Since $Z$ is an $S^2$-bundle over $M=G_2/SO(4)$, we should start with the cohomology ring of $M$. It is shown on page 529 of Hirzebruch and Borel's classic paper on homogeneous spaces that $H^*(M;\Z_2)$ is generated by two elements $u_2,u_3$ in degree $2,3$ respectively, subject to the relations $u_2^3=u_3^2$ and $u_3^3=0$. Furthermore, the real cohomology is of rank one in degrees zero, four and eight. Poincar\'e duality shows that the generator in degree four must square to the generator in degree eight; up to orientation this can be taken to be the positive generator. This determines $H^*(M;\R)$. Finally, it is known that $H^*(M;\Z)$ has no torsion of odd order.
%WORK NEEDED: Ishitoya & Toda: WHY?!

\medskip

Now we want to determine $H^*(M;\Z)$. We know some thing about the first few cohomology groups already. Since $M$ is simply connected, and $H_0(M;G)$ is always free of rank one (hence $\Ext$ vanishes), $H^1(M;\Z)\cong \Hom(H_1(M;\Z);\Z)=0$. Secondly, using the Hurewicz theorem we see that $\pi_2(M)\twoheadrightarrow H_2(M;\Z)$. The long exact homotopy sequence of the fibration $SO(4)\to G_2\to M$ contains the following piece:
\begin{equation*}
	\begin{tikzcd}
		\dots \ar[r] & \pi_2(G_2)=0 \ar[r] & \pi_2(M) \ar[r] & \pi_1(SO(4))=\Z_2 \ar[r] & \pi_1(G_2)=0
	\end{tikzcd}
\end{equation*}
This shows that $\pi_2(M)=\Z_2$ and therefore $H_2(M;\Z)=\Z_2$. The universal coefficients theorem then shows that $H^2(M;\Z)=0$. 

\medskip

The short exact sequence
\begin{equation*}
	\begin{tikzcd}
		0 \ar[r] & \Z \ar[r,"2\cdot"] & \Z \ar[r] & \Z_2 \ar[r] & 0
	\end{tikzcd}
\end{equation*}
induces a long exact sequence on cohomology. We find
\begin{equation*}
	\begin{tikzcd}[column sep=small]
		0\ar[r] & H^2(M;\Z_2)=\Z_2 \ar[r] & H^3(M;\Z) \ar[r,"2\cdot"] & H^3(M;\Z) \ar[r] 
		& H^3(M;\Z_2)=\Z_2 \ar[r] & \dots
	\end{tikzcd}
\end{equation*}
Thus, the map $H^3(M;\Z)\xrightarrow{2\cdot}H^3(M;\Z)$ has kernel $\Z_2$ and cokernel contained in $\Z_2$. The fact that it has no free part or odd torsion then implies that $H^3(M;\Z)=\Z_{2^k}$.
%WORK NEEDED: How to distinguish Z_4={2^k} from Z_2?
In fact, $H^3(M;\Z)=\Z_2$ and therefore the next piece of the long exact sequence yields
\begin{equation*}
	\begin{tikzcd}
		\dots\ar[r] & \Z_2 \ar[r,"0"] & H^4(M;\Z) \ar[r,"2\cdot"] & H^4(M;\Z) \ar[r] & H^4(M;\Z_2)=\Z_2 \ar[r] & \dots
	\end{tikzcd}
\end{equation*}
$H^4(M;\Z)=\Z\oplus T_4$ where $T_4$ is the torsion in degree four, which only contains summands of the form $\Z_{2n}$ for $n\in \N$. But if $T_4\neq \{0\}$, the multiplication map $2\cdot$ cannot be injective, hence $H^4(M;\Z)=\Z$. The map $H^4(M;\Z)\to H^4(M;\Z_2)$ is then onto and the next piece becomes
\begin{equation*}
	\begin{tikzcd}
		\dots \ar[r,two heads] & \Z_2 \ar[r,"0"] & H^5(M;\Z) \ar[r,"2\cdot"] 
		&H^5(M;\Z) \ar[r] & H^5(M;\Z_2)=\Z_2 \ar[r] &\dots
	\end{tikzcd}
\end{equation*}
Thus, the map $2\cdot$ must be an injective map of the pure torsion group $H^5(M;\Z)$. But that is only possible if $H^5(M;\Z)=0$, since there is no odd torsion. We move on to the next degree:
\begin{equation*}
	\begin{tikzcd}
		0\ar[r] & \Z_2 \ar[r] & H^6(M;\Z) \ar[r,"2\cdot"] 
		& H^6(M;\Z) \ar[r] & \Z_2 \ar[r] & 0
	\end{tikzcd}
\end{equation*}
where we used that $H^7(M;\Z)\cong H_1(M;\Z)=0$ by simple connectedness. This once again establishes that $H^6(M;\Z)=\Z_{2^k}$ for some $k\in \N$
%WORK NEEDED: How to distinguish Z_4={2^k} from Z_2?
and in fact $H^6(M;\Z)=\Z_2$. We have obtained:

\begin{equation*}
	H^k(M;\Z)= 
	\begin{cases}
		0\qquad &k=1,2,5,7 \\
		\Z_2 &k=3,6\\
		\Z &k=0,4,8
	\end{cases}
\end{equation*}

\subsubsection{Ring Structure of \texorpdfstring{$H^*(G_2/SO(4);\Z)$}{the Integral Cohomology of G2 over SO(4)}}

It follows from Poincar\'e duality that the generator of $H^4(M;\Z)$ squares to a generator (without loss of generality, the positive one) of $H^8(M;\Z)$. All that remains is to see whether the square of the generator in degree three vanishes or not.
%WORK NEEDED: Does it vanish or not?!
In fact, it squares to the generator of $H^6(M;\Z)$.

\subsubsection{Additive Structure of \texorpdfstring{$H^*(Z;\Z)$}{the Integral Cohomology of Z}}

We know that $Z$ is an $S^2$-bundle over $M$, which is the sphere bundle of an oriented, rank three bundle. This allows us to formulate the Gysin sequence
\begin{equation*}
	\begin{tikzcd}[column sep=small]
		\dots \ar[r] & H^k(M;\Z) \ar[r,"\smile e"] & H^{k+3}(M;\Z) \ar[r,"\pi^*"] & H^{k+3}(Z;\Z) \ar[r] & H^{k+1}(M;\Z) \ar[r] & \dots
	\end{tikzcd}
\end{equation*}
where $e$ is the Euler class of the rank three bundle (which is 2-torsion). It is easy to see that $H^1(Z;\Z)=0$ (either from the Gysin sequence or simple connectedness). We start in degree two:
\begin{equation*}
	\begin{tikzcd}
		0\ar[r] & H^2(M;\Z)=0\ar[r] & H^2(Z;\Z) \ar[r] & H^0(M;\Z) \ar[r,"\smile e"] 
		& H^3(M;\Z) \ar[r] & \dots
	\end{tikzcd}
\end{equation*}
We know that $H^2(Z;\Z)$ is not zero, since $Z$ is K\"ahler. The existence of an injective homomorphism $H^2(Z;\Z)\to\Z$ implies that $H^2(Z;\Z)=\Z$. For degree three, we find
\begin{equation*}
	\begin{tikzcd}
		\Z\ar[r,"\smile e"] & \Z_2 \ar[r] & H^3(Z;\Z) \ar[r] & 0
	\end{tikzcd}
\end{equation*}
Thus, 
\begin{equation*}
	H^3(Z;\Z)=
	\begin{cases}
		0 \qquad & e\neq 0\\
		\Z_2 & e=0
	\end{cases}
\end{equation*}
The next degree is easy:
\begin{equation*}
	\begin{tikzcd}
		0\ar[r,"\smile e"] & \Z\ar[r] & H^4(Z;\Z) \ar[r] & 0
	\end{tikzcd}
\end{equation*}
and $H^4(Z;\Z)=\Z$ is forced. In degree five, we find the same ambiguity as in degree three:
\begin{equation*}
	\begin{tikzcd}
		0\ar[r] & H^5(Z;\Z) \ar[r] & \Z_2 \ar[r,"\smile e"] & \Z_2 \ar[r] & \dots
	\end{tikzcd}
\end{equation*}
If $e\neq 0$, it generates $H^3(M;\Z)$ and therefore squares to the generator of $H^6(M;\Z)$, i.e. $\smile e$ is an isomorphism in this case. This shows that
\begin{equation*}
	H^5(Z;\Z)=
	\begin{cases}
		0 \qquad & e\neq 0\\
		\Z_2 & e=0
	\end{cases}
\end{equation*}
Continuing on, we find
\begin{equation*}
	\begin{tikzcd}
		\Z_2 \ar[r,"\smile e"] & \Z_2 \ar[r] & H^6(Z;\Z) \ar[r] & \Z \ar[r] & 0
	\end{tikzcd}
\end{equation*}
and a similar result:
\begin{equation*}
	H^6(Z;\Z)=
	\begin{cases}
		\Z \qquad & e\neq 0\\
		\Z\oplus\Z_2 & e=0
	\end{cases}
\end{equation*}
Degree seven is simple:
\begin{equation*}
	\begin{tikzcd}
		0\ar[r] & H^7(Z;\Z) \ar[r] & 0 
	\end{tikzcd}
\end{equation*}
Next up, we determine
\begin{equation*}
	\begin{tikzcd}
		0 \ar[r,"\smile e"] & \Z\ar[r] & H^8(Z;\Z) \ar[r] & \Z_2 \ar[r] & 0
	\end{tikzcd}
\end{equation*}
hence $H^8(Z;\Z)=\Z$ or $\Z\oplus \Z_2$. The universal coefficients theorem
\begin{equation*}
	\begin{tikzcd}
		0 \ar[r] & \Ext(H_7(Z),\Z) \ar[r] & H^8(Z;\Z) \ar[r] & \Hom(H_8(Z),\Z) \ar[r] & 0
	\end{tikzcd}
\end{equation*}
determines which it is, depending on the Euler class:
\begin{equation*}
	H^8(Z;\Z)=
	\begin{cases}
		\Z \qquad &e\neq 0\\
		\Z\oplus\Z_2 &e=0
	\end{cases}
\end{equation*}
In fact, it is $\Z$. Finally, it is easy to check that $H^9(Z;\Z)=0$. We conclude:
\begin{equation*}
	H^k(Z;\Z)=
	\begin{cases}
		\begin{cases}
			0 \qquad &k=\text{odd}\\
			\Z &k=\text{even},\leq 10
		\end{cases}\qquad &e\neq 0\\
		\\
		\begin{cases}
			0\qquad & k=1,7,9\\
			\Z_2 &k=3,5\\
			\Z &k=0,2,4,10\\
			\Z\oplus\Z_2 &k=6,8
		\end{cases} &e=0
	\end{cases}
\end{equation*}

Now, in our case, Salamon showed that $Z=S(S^2H)$ and furthermore $w_2(S^2H)=\epsilon\neq 0\in H^2(M;\Z_2)$. It is a general fact that for a rank-3 oriented vector bundle $V$, $e(V)=\beta(w_2(V))$, where $\beta:H^2(M;\Z_2)\to H^3(M;\Z)$ is the Bockstein homomorphism. 
%WORK NEEDED: Apparently from the universal bundle stuff
Thus, since we know that $\beta(\epsilon)\neq 0$, we are in the case $e\neq 0$; the cohomology of $Z$ takes on the simplest possible form.


\subsubsection{Ring Structure of \texorpdfstring{$H^*(Z;\Z)$}{the Integral Cohomology of Z}}

The Gysin sequence shows that $\pi^*:H^4(M;\Z)\to H^4(Z;\Z)$ is an isomorphism, i.e. if we denote the positive generator of $H^4(M;\Z)$ by $g_4^M$, then the positive generator of $H^4(Z;\Z)$ can be chosen to be $g_4\coloneqq \pi^* g_4^M$. Similarly, $\pi^*$ on degree eight is seen to be multiplication by $2$, hence we set $g_8\coloneqq \frac{1}{2}\pi^*g_8^M$. But at the same time, $g_8^M=(g_4^M)^2$ and therefore naturality of pullback shows that $g_4^2=2g_8$. 

\medskip

Since there is no torsion, Poincar\'e duality yields a \emph{non-degenerate} pairing 
\begin{equation*}
	\begin{tikzcd}[row sep=0cm]
		H^k(Z) \times H^{10-k} \ar[r] & \Z \\
		(\alpha,\beta)\ar[r,mapsto] & (\alpha\smile\beta)(\mu)
	\end{tikzcd}
\end{equation*}
where $\mu$ is the fundamental class. We therefore have generators $g_{2k}\in H^{2k}(Z;\Z)$ which satisfy $g_{2k}\smile g_{10-2k}=g_{10}$. In particular, we find $g_2g_8=\frac{1}{2}g_2g_4^2=g_4g_6$ and therefore $g_6=\frac{1}{2}g_2g_4$. Now, the cohomology ring is completely determined by specifying the constant $\alpha$ in $g_2^2=\alpha g_4$. This number is determined by the Chern number $c_1^5(Z)$, since if we write $c_1(Z)=d_2 g_2$ for some $d_2\in \Z$, we find
\begin{equation*}
	c_1^5=\langle d_2^5g_2^5,Z\rangle
	=2d_2^5\alpha^2\langle g_2g_8,Z\rangle=2d_2^5\alpha^2
\end{equation*}
For the twistor space over $G_2/SO(4)$ the Hilbert polynomial is given by Semmelman \& Weingart. It allows us to deduce that $c_1^5(Z)=4374=2\cdot 3^7$. $Z$, being a twistor space over a space of dimension $4n=8$, is a holomorphic contact manifold. This implies that $c_1(Z)$ has divisibility at least $n+1=3$, thus $d_2=3$ and we find that $\alpha^2=9$, i.e. $\alpha=\pm 3$.

\begin{rem}
	The same arguments go through for $Q$, but in this case one obtains $\alpha=\pm 1$.
\end{rem}

The coefficient $\alpha$ can be determined using Pontryagin classes. Salamon showed that $T\pi\cong L^2$ and we saw that $c_1(Z)=3g_2=3c_1(L^2)$, hence $c_1(L^2)=g_2$ and $p_1(L^2)=p_1(T\pi)=g_2^2=\alpha g_4$. Using the fact that the normal bundle of $S^2\subset \R^3$ is trivial, we see that as bundles on $S^2$, $TS^2\oplus \underline \R\cong \underline \R\oplus\underline \R\oplus\underline \R$. We can apply this isomorphism fiberwise to the sphere bundle $Z=S(S^2H)$ to find that $\pi^*(S^2H)\cong T\pi\oplus \underline \R$. This implies that $\pi^*p_1(S^2H)=p_1(T\pi)=\alpha g_4$. 

\begin{rem}
	For any $SO(3)$-bundle, it also holds that $p_1\equiv w_2^2\mod 4$. This shows that in our case $\alpha$ must be odd.
	%WORK NEEDED: Universal bundle over classifying space of SO(3) stuff?
\end{rem}

Thus, the only thing left to do is to calculate $p_1(S^2H)$.


\subsection{Chern Classes of $Z$}

The Hilbert polynomial argument, using Semmelman \& Weingart, shows that $c_1^5=4374$, $c_1^3c_2=2106$. Furthermore, the Hodge numbers determine the Chern number $c_1c_4=90$. Furthermore, recall that the cohomology ring of $Z$ depends on just one constant, $\alpha$, for which we found $\alpha^2=9$. The value of $\alpha$ can be found from $p_1(S^2H)$, but will not be needed here. 

\medskip

Let $c_k=d_{2k}g_{2k}$ for some $d_{2k}\in \Z$ and $g_{2k}$ the generator in degree $2k$. It is clear that $c_5=6g_{10}$ from the Euler characteristic. Using the ring structure, we have $c_1^5=2\alpha^2d_2^5=18\cdot d_2^5\implies d_2=3$. Since $g_2g_8=g_{10}$, we then deduce that $c_4=30g_8$ from $c_1c_4=90$. Now, $c_1^3c_2=27\cdot 2\alpha d_4=2106$ shows that $\alpha d_4=3\cdot 13$. Hence $d_4=\pm 13$, depending on $\alpha$. 

\medskip

The final Chern class $c_3$ can be determined using the second Pontryagin class $p_2(Z)$. Note that it only depends on $\alpha^2$. We decompose the tangent bundle of $Z$ as $TZ=\pi^*TM\oplus T\pi$ where $M=G_2/SO(4)$ is the base space of the twistor fibration and $T\pi$ is the vertical tangent bundle (along the projection $\pi:Z\to M$). Since there is no $2$-torsion in degrees $4n$, we have $p(Z)=\pi^*p(M) \cdot p(T\pi)$. 

\medskip

Because $T\pi$ has rank two, $p_1(T\pi)=c_1^2(T\pi)$ and Salamon's work shows that $T\pi\cong L^2$ (in Salamon's notation; $L$ itself is not globally defined) and that $c_1(Z)=3g_2=3c_1(L^2)$, hence $p_1(T\pi)=9\alpha g_4$. In addition, he shows that $7p_2(M)-p_1^2(M)=45$ and $7p_1^2(M)-4p_2(M)=0$, whence $p_1(M)=\pm 2 g_4^M$ and $p_2=g_8^M$. Hence 
\begin{equation*}
	p_2(Z)=c_1^2(T\pi)\pi^* p_1(M)+\pi^*p_2(M)
	=\pm 2g_2^2g_4+2g_8=(\pm 4\alpha +2)g_8
\end{equation*}
and we find
\begin{equation*}
	p_2(Z)=2c_4(Z)-2c_3(Z)c_1(Z)+c_2^2(Z)=(\pm 4\alpha+2)g_8
\end{equation*}
%WORK NEEDED: Why is it not pullback of p_2?!?!

\medskip


\section{Side Questions}

\begin{numberedlist}
	\item Does the Todd genus of the nearly K\"{a}hler manifolds obtained by flipping fibers of twistor spaces always vanish?
	\item What about the $S^2$-bundles over $G_2/SO(4)$ with $e\neq 0$ but different values of $\alpha$?
\end{numberedlist}

\section{Finished Computations on $Q$}

\subsection{Is $Q$ 3-Symmetric?}

Find $Q$ in some table by Wolf \& Gray classifying 3-symmetric spaces. We need to do this to know that it has a canonical nearly K\"{a}hler structure. This, in turn, follows from some classification results by Nagy and Butruille which say that nearly K\"{a}hler structures come from products of just three families of examples:
\begin{numberedlist}
	\item Complex 3-dimensional manifolds.
	\item Twistor spaces of quaternionic K\"{a}hler manifolds.
	\item 3-symmetric spaces.
\end{numberedlist}
We suspect $Q$ is an ``irreducible'' one, and it fails the first two criteria. Thus, it should be 3-symmetric. This is confirmed by page 111 from Wolf \& Gray's paper ``Homogeneous Spaces Defined by Lie Group Automorphisms, I'', which has $G_2/U(2)$ in a list of 3-symmetric spaces. Furthermore, it is indicated that there are four invariant almost complex structures, hence it must be $Q$ (and not $Z$).

\subsection{Chern Classes and Numbers of \texorpdfstring{$Q$}{Q} with its Standard Complex Structure}

$Q$ is a five-dimensional complex manifold; a quadric in $\CP^6$. The normal bundle sequence
\begin{equation*}
	\begin{tikzcd}
		0 \ar[r] & TQ \ar[r] & T\CP^6|_Q \ar[r] & \nu(Q) \ar[r] & 0
	\end{tikzcd}
\end{equation*}
splits in the complex category due to the existence of Hermitian metrics (which, in turn, exist due to the existence of complex bump functions), i.e. $T\CP^6|_Q\cong TQ\oplus \nu(Q)$. Let $\iota$ be the inclusion $Q\hookrightarrow \CP^6$, and recall that $\nu(Q)=\iota^*\mc O(2)$, because $Q$ is defined by (the zero set of) a quadratic polynomial on $\C^7$, i.e. a section of $\mc O(2)$. Using naturality of Chern classes, we find
\begin{align*}
	\iota^* c(\CP^6)&=c(Q)\smile \iota^*c(\mc O(2))\\
	(1+\iota^*\alpha)^7&=c(Q)(1+2\iota^*\alpha)\\
	&=(1+c_1(Q)+\dots +c_5(Q))(1+2\iota^*\alpha)
\end{align*}
where $\alpha$ is the standard generator of $H^2(\CP^n)$. Set $x=\iota^*\alpha$. Looking at this level by level, we find
\begin{align*}
	7x&=c_1(Q)+2x\implies c_1(Q)=5x\\
	21x^2&=2c_1(Q)x+c_2(Q)=10x^2+c_2(Q)
	\implies c_2(Q)=11x^2 \\
	35x^3&=2c_2(Q)x+c_3(Q)=22x^3+c_3(Q)\implies c_3(Q)=13x^3\\
	35x^4&=2c_3(Q)x+c_4(Q)=26x^4+c_4(Q)\implies c_4(Q)=9x^4\\
	21x^5&=2c_4(Q)x+c_5(Q)=18x^5+c_5(Q)\implies c_5(Q)=3x^5
\end{align*}
The total Chern class of $Q$ is therefore given by 
\begin{equation}
	c(Q)=1+5x+11x^2+13x^3+9x^4+3x^5
\end{equation}
Now, we can easily compute Chern numbers by evaluating on $[Q]\in H^{10}(\CP^6)$. Because $Q$ is a quadric, $[Q]=2[\CP^5]$, i.e. the Chern numbers are given by \emph{twice the coefficient of the Chern classes}:

\begin{table}[ht!]\centering
	\begin{tabular}{ll} \toprule
		Chern Class & Chern Number \\ \midrule
		$c_5$ 		& $6$ \\
		$c_1^5$ 	& $6250$\\
		$c_1^3c_2$	& $2750$ \\
		$c_1^2c_3$	& $650$ \\
		$c_1c_4$	& $90$ \\
		$c_1c_2^2$	& $1210$ \\
		$c_2c_3$	& $286$ \\ \bottomrule
	\end{tabular}
	\caption{Chern Numbers of $Q$ with standard complex structure.}
\end{table}

This verifies that the first column of table 10 of Grama et al. is indeed $Q$.

\subsection{Chern Classes and Numbers of $\P(TS^6)$ and $\P(T^*S^6)$}

\subsubsection{The Cohomology Ring of a Projectivized Bundle}

This should be done using a computation analogous to Kotschick's paper on positive curvature. The projectivized tangent bundle $\CP^5\to \P(TS^6)\to S^6$ has a cohomology ring that can be computed using the Leray-Hirsch theorem, which states the following (for more information, see Bott \& Tu, page 269 and onwards):

\begin{thm}[Leray-Hirsch]
	Let $E$ be a fiber bundle over $M$ with fiber $F$. Suppose $M$ has a finite, good cover (this is satisfied if $M$ is compact). If there are global cohomology classes $e_1,\dots,e_n$ on $E$ which, when restricted to each fiber, freely generated the cohomology of the fiber, then $H^*(E)$ is a free module over $H^*(M)$ with basis $\{e_1,\dots,e_n\}$, i.e.
	\begin{equation*}
		H^*(E)\cong H^*(M)\otimes \Z[e_1,\dots,e_n]\cong H^*(M)\otimes H^*(F)
	\end{equation*}
\end{thm}

Furthermore, this yields a way of defining the Chern classes of $E$. Consider $\P(E)$, where $E$ is a complex vector bundle of rank $n$. It has certain universal bundles over it, namely the pullback bundle $\pi^*E=\{(\ell,v)\mid \pi_{\P(E)}\ell=\pi(v)\}$, i.e. the bundle whose fiber over $\ell_p$ (which maps to $p\in M$ under $\pi_{\P(E)}$) is $E_p$. It has a tautological subbundle $L=\{(\ell,v)\in \pi^*E\mid v\in \ell\}$. Its dual is, as usual, called the hyperplane bundle and denoted by $H$. 

\medskip

Set $y=c_1(H)$. Then $1,y,\dots,y^{n-1}$ freely generate $H^*(\P(E_p))$, because $y$ restricts to the hyperplane class on each fiber.
%WORK NEEDED Understand this better
The Leray-Hirsch theorem tells us that $H^*(\P(E))$ is a free module over $H^*(M)$, generated by $1,y,\dots,y^{n-1}$. In particular, $y^n$ can be expressed as a linear combination of the lower powers, with coefficients in $H^*(M)$.

\begin{mydef}
	The \emph{Chern classes} of $E$ are the coefficients $c_1(E),\dots,c_n(E)$ satisfying
	\begin{equation*}
		y^n+c_1(E)y^{n-1}+\dots+c_n(E)=0\qquad \qquad c_i(E)\in H^{2i}(M)
	\end{equation*}
\end{mydef}

Thus, the ring structure of $H^*(\P(E))$ is given by
\begin{equation*}
	H^*(\P(E))=H^*(M)[y]/\langle y^n+c_1(E)y^{n-1}+\dots + c_n(E)\rangle
\end{equation*}

\subsubsection{Application to $\P(TS^6)$}

Let $\alpha\in H^6(S^6)$ be an orientation class. Then $c_3(TS^6)=k\cdot \alpha$ for some $k\in\Z$. In fact, since the top Chern number is the Euler characteristic $\chi(S^6)=c_3(S^6)=2$, i.e. $k=2$. All the other classes must vanish because the cohomology of $S^6$ is so simple. As discussed before, $S^6$ admits an almost complex structure induced by viewing it as the unit sphere in $\Im \O$. This endows the tangent bundle with the structure of a complex vector bundle. Therefore, we can apply the above discussion to $TS^6$, viewed as a rank $3$ complex vector bundle. We conclude:

\begin{prop}
	The cohomology ring of $\P(TS^6)$ is generated by two elements: $x\in H^6(\P(TS^6))$ and $y\in H^2(\P(TS^6))$ which satisfy the relations
	\begin{equation*}
		x^2=0 \qquad \qquad y^3=-2 x
	\end{equation*}
	$H^{10}(\P(TS^6))$ has positive generator $xy^2$.
\end{prop}
\begin{myproof}
	The only thing that we have not yet explained is $x^2=0$. But $x=\pi^*\alpha$, where $\pi:\P(TS^6)\to S^6$ is the obvious projection. The relation follows for dimensional reasons. The generator in top degree is $xy^2$, since $x$ is a (by definition positive) generator of the base space cohomology and $y$ is too, for each fiber.
\end{myproof}

\begin{prop}
	The total Chern class of $\P(TS^6)$ is given by 
	\begin{equation*}
		c(\P(TS^6))=1+3y+3y^2+2x+6xy+6xy^2
	\end{equation*}
\end{prop}
\begin{myproof}
	Since $T(\P(TM))\cong T\pi\oplus \pi^*(TM)$ for any almost complex manifold $M$, we have $c(\P(TS^6))=c(T\pi)\cdot c(\pi^*(TS^6)=c(T\pi)\cdot \pi^*c(S^6)$. Here, $T\pi$ is the vertical tangent bundle. Clearly, $c(\pi^*(TS^6))=1+2x=1-y^3$ by naturality of the Chern class. To calculate $c(T\pi)$, we use the \emph{(relative) Euler sequence}
	\begin{equation*}
		\begin{tikzcd}
			0 \ar[r] & L \ar[r] & \pi^*(TS^6) \ar[r] & L\otimes T\pi \ar[r] & 0
		\end{tikzcd}
	\end{equation*}
	As complex bundles, every short exact sequence splits:
	\begin{equation*}
		\pi^*(TS^6)=L\oplus (L\otimes T\pi)\Leftrightarrow
		H\otimes \pi^*(TS^6)=\underline{\C}\oplus T\pi
	\end{equation*}
	Therefore we have 
	\begin{equation*}
		c(T\pi)=c(\pi^*(TS^6)\otimes H)
	\end{equation*}
	For any complex vector bundle $E$ and complex line bundle $L$, the Chern classes of $E\otimes L$ are given---using exercise 4.4.6 from Huybrechts' book---by:
	\begin{equation*}
		c_i(E\otimes L)=\sum_{j=0}^i 
		\begin{pmatrix}
			\rank_\C E-j \\ i-j 
		\end{pmatrix}
		c_j(E)c_1(L)^{i-j}
	\end{equation*}
	Again, $c(\pi^*(TS^6))=1+2x$ and $c(H)=1+y$, where $kx=c_3(\pi^*(TS^6))$ and of course $y=c_1(H)$. We compute (recall that $\pi:\P(TS^6)\to S^6$ is a rank two complex vector bundle):
	\begin{align*}
		c_1(T\pi)&=3y\\
		c_2(T\pi)&=3y^2\\
		\implies c(T\pi)&=1+3y+3y^2
	\end{align*}
	We conclude:
	\begin{equation*}
		c(\P(TS^6))=(1+3y+3y^2)(1+2x)
		=1+3y+3y^2+2x+6xy+6xy^2
	\end{equation*}
\end{myproof}\unskip

\subsubsection{Application to $\P(T^*S^6)$}

\begin{prop}
	The cohomology of $\P(T^*S^6)$ is generated by classes $x\in H^6(\P(T^*S^6))$ and $y\in H^2(\P(TS^6))$ which satisfy the relations
	\begin{equation*}
		x^2=0 \qquad \qquad z^3=2x
	\end{equation*}
	and $H^{10}(\P(T^*S^6))$ has positive generator $xz^2$.
\end{prop}
\begin{myproof}
	Identical to the case $\P(TS^6)$. The positive generator is $xz^2$ for the same reasons as before.
\end{myproof}

\begin{prop}
	The total Chern class of $\P(T^*S^6)$ is given by
	\begin{equation*}
		c(\P(T^*S^6))=1+3z+3z^2+2x+6xz+6xz^2
	\end{equation*}
\end{prop}
\begin{myproof}
	$c(\pi^*(TS^6))=1+2x$ and in this case $c(T\pi)=c(\pi^*(TS^6)\otimes H)$. All in all, we find:
	\begin{equation*}
		c(\P(T^*S^6))=(1+3z+3z^2)(1+2x)=1+3z+3z^2+2x+6xz+6xz^2
	\end{equation*}
	as claimed.
\end{myproof}

The Chern numbers are summarized in \cref{tab:numbers}.

\begin{table}[ht!]\centering
	\begin{tabular}{lll} \toprule
		Chern Number& $\P(TS^6)$ & $\P(T^*S^6)$ \\ \midrule
		$c_5$ 		& $6$ 	&  $6$\\
		$c_1^5$ 	& $-486$&  $486$\\
		$c_1^3c_2$	& $-162$&  $162$\\
		$c_1^2c_3$	& $18$ 	&  $18$\\
		$c_1c_4$	& $18$ 	&  $18$\\
		$c_1c_2^2$	& $-54$ &  $54$\\
		$c_2c_3$	& $6$	&  $6$\\ \bottomrule
	\end{tabular}
	\caption{Chern Numbers of $\P(TS^6)$.}\label{tab:numbers}
\end{table}

\subsection{Flipping the Fibers}

\subsubsection{The Fibrations over $S^6$}

We want the Chern numbers of $\P(TS^6)$, $\P(T^*S^6)$ and $Q$ with standard structure after flipping the fibers of the fibrations over $S^6$. The fiber has $\dim_\C=2$, hence the orientation is not affected.

\medskip

Recall that we realized $Q$ as $\P(TS^6)$ (and $\P(T^*S^6)$). The Chern numbers were computed by writing $T(\P(TS^6))=T\pi\oplus \pi^*TS^6$; the Whitney sum formula tells us $c(\P(TS^6))=c(T\pi)c(\pi^*TS^6)=c(T\pi)(1+2x)$ where $x=\pi^*\alpha$ with $\alpha$ the positive generator of $H^6(S^6)$. Now we can simply flip the fiber, i.e. consider the conjugate complex structure on the fibers. Call the resulting almost complex manifold $R$. Then $TR=\overline{T\pi}\oplus \pi^*TS^6$ and we note that since the fibers have $\dim_\C=2$, the orientation of $R$ is the same as that of $\P(TS^6)$. The Chern class of $\overline{T\pi}$ is $1-c_1(T\pi)+c_2(T\pi)$, so we find that the total Chern class is
\begin{equation*}
	c(R)=c(\P(TS^6))\frac{c(\overline{T\pi})}{c(T\pi)}
	=c(\P(TS^6))\frac{1-3y+3y^2}{1+3y+3y^2}
\end{equation*}
Working out this calculation, we get:
\begin{equation*}
	c(R)=1-3y+3y^2+2x-6xy+6xy^2
\end{equation*}
which is also the result obtained by using $c_k(\overline{T\pi})=(-1)^kc_k(T\pi)$. Recall the relation $y^3=-2x$ and the fact that $xy^2$ is the positive generator. Similarly we can flip the fibers of $\P(T^*S^6)$ to obtain the space $S$ which again has Chern numbers that are the same up to sign, determined by the Chern class
\begin{equation*}
	c(S)=c(\P(T^*S^6))\frac{1-3z+3z^2}{1+3z+3z^2}
	=1-3z+3z^2+2x-6xz+6xz^2
\end{equation*}
For the calculation, remember that $z^3=2x$ and $xz^2$ is the positive generator. The Chern numbers of $R$ are those of $\P(T^*S^6)$ and those of $S$ correspond to $\P(TS^6)$.

\medskip

When flipping the fiber for the standard structure on $Q$, we have to be more careful: The fibers are holomorphic submanifolds but the fibration is not holomorphic in any sense. Calling the fibration $\pi$, we thus have $TQ=T\pi\oplus D$ where $D$ is a complementary complex vector bundle (of rank three). 

\medskip

Recall that $c(Q)=1+5x+11x^2+13x^3+9x^4+3x^5$, where $x=\iota^*\alpha$ and $\alpha$ is the positive generator of $H^2(\CP^n)$. Using the Whitney sum formula, we can determine the Chern classes of the subbundles. For the first Chern class, we have the equation
\begin{equation*}
	5x=c_1(T\pi)+c_1(D)
\end{equation*}
Of course $c_1(T\pi)=d\cdot x$ for some $d\in\Z$. From our earlier discussion on the cohomology of $Q$ we know that $x$ restricts to the positive generator of $H^2$ of the fiber. Since each fiber is just a copy of $\CP^2$, this means that $c_1(T\pi)=3x$. Thus, $c_1(D)=2x$. We proceed similarly for $c_2$:
\begin{equation*}
	11x^2=c_2(T\pi)+c_1(T\pi)c_1(D)+c_2(D)\Leftrightarrow 5x^2=c_2(T\pi)+c_2(D)
\end{equation*}
As before, $x^2$ restricts to a positive generator of the fourth degree cohomology on the fiber and the Chern classes of $\CP^2$ then tell us $c_2(T\pi)=3x^2$. Hence $c_2(D)=2x^2$. It is clear that the $c_3(D)=x^3$ since
\begin{equation*}
	1+5x+11x^2+13x^3+9x^4+3x^5=(1+3x+3x^2)(1+2x+2x^2+x^3)
\end{equation*}
Now, we flip the fiber to obtain the almost complex manifold $P$ which has Chern class
\begin{equation*}
	c(P)=c(Q)\frac{1-3x+3x^2}{1+3x+3x^2}=1-x-x^2+x^3+3x^4+3x^5
\end{equation*}
The Chern numbers for the ``flipped fibrations'' over $S^6$ are shown in \cref{tab:flippedS6}. Note that the Chern numbers of $P$ correspond to those of the last column of the first table of Grama et al.---this should be the nearly K\"{a}hler structure.

\begin{table}[ht!]\centering
	\begin{tabular}{llll} \toprule
		Chern Number& $P$	& $R$	& $S$	\\ \midrule
		$c_5$ 		& $6$ 	& $6$	& $6$ 	\\
		$c_1^5$ 	& $-2$	& $486$	& $-486$\\
		$c_1^3c_2$	& $2$	& $162$ & $-162$\\
		$c_1^2c_3$	& $2$	& $18$ 	& $18$	\\
		$c_1c_4$	& $-6$	& $18$ 	& $18$	\\
		$c_1c_2^2$	& $-2$	& $54$ 	& $-54$	\\
		$c_2c_3$	& $-2$	& $6$	& $6$	\\ \bottomrule
	\end{tabular}
	\caption{Chern Numbers of the ``flipped fibrations'' $P,P',Q'$ of the underlying manifold $G_2/U(2)$ over $S^6$.}\label{tab:flippedS6}
\end{table}

\subsubsection{The Fibrations over $G_2/SO(4)$}

We start by considering the projectivized tangent bundle of $S^6$. Recall that $c(\P(TS^6))=1+3y+3y^2-y^3-3y^4-3y^5$ and $y^5$ is $-2$ times the positive generator $xy^2$ of the top degree cohomology. Using polynomial long division one can show that $c(\P(TS^6))$ contains a factor of the form $1+Cy$ ($C\in\Z$) only for $C=-1$. Hence
\begin{equation*}
	c(\P(TS^6))=c(T\pi)c(D)=(1-y)c(D)
\end{equation*}
which determines $c(D)$ as well. Flipping the fiber to obtain a space $R'$ and keeping in mind that the orientation is now reversed so that $\langle y^5,[R']\rangle=2$, we have:
\begin{equation*}
	c(R')=(1+y)c(D)=c(\P(TS^6))\frac{1+y}{1-y}=1+5y+11y^2+13y^3+9y^4+3y^5
\end{equation*}
Clearly, the Chern numbers will be those of $Q$ with its standard structure. 

\medskip

Playing the same game for $\P(T^*S^6)$, which has Chern class $c(\P(T^*S^6))=1+3z+3z^2+z^3+3z^4+3z^5$ ($z$ is twice the positive generator $xz^2$ of top degree cohomology), we find that the fiber has Chern class $c(T\pi)=1+z$ and after flipping the fiber (and orientation, which ensures $\langle z^5,[S']\rangle=-2$) to obtain $S'$, we find
\begin{equation*}
	c(S')=c(\P(T^*S^6))\frac{1-z}{1+z}=1+z-z^2-z^3+3z^4-3z^5
\end{equation*}
The Chern numbers are easily computed.

\medskip

We proceed in analogous fashion for the standard structure on $Q$: $c(Q)$ contains a factor $1+Cx$ ($C\in\Z$) only for $C=1$. Hence $c(T\pi)=1+x$ and flipping the fiber will yield $P'$ with Chern class
\begin{equation*}
	c(P')=(1+5x+11x^2+13x^3+9x^4+3x^5)\frac{1-x}{1+x}
	=1+3x+3x^2-x^3-3x^4-3x^5
\end{equation*}
It is easily checked that $C=1$ (in our earlier notation) is also the only option if one requires $c_5(P')=\chi(Q)=6$ since we are not changing the homotopy type (this check also works for $R',S'$, of course). Keeping in mind that the orientation is now switched, hence $\langle x^5,[P']\rangle=-2$, we find the Chern numbers of $\P(TS^6)$. 

\begin{table}[ht!]\centering
	\begin{tabular}{llll} \toprule
		Chern Number& $P'$	& $R'$	& $S'$	\\ \midrule
		$c_5$ 		& $6$ 	& $6$	& $6$ 	\\
		$c_1^5$ 	& $-486$& $6250$& $-2$	\\
		$c_1^3c_2$ 	& $-162$& $2750$& $2$	\\
		$c_1^2c_3$ 	& $18$	& $650$	& $2$	\\
		$c_1c_4$ 	& $18$	& $90$	& $-6$	\\
		$c_1c_2^2$	& $-54$	& $1210$& $-2$	\\
		$c_2c_3$	& $6$	& $286$	& $-2$	\\ \bottomrule
	\end{tabular}
	\caption{Chern Numbers of the ``flipped fibrations'' $P',R',S'$ over $G_2/SO(4)$.}
\end{table}

\subsection{Summary: The Homogeneous Almost Complex Structures on $Q$}

The above computations can be summarized as follows: On the level of Chern numbers (classes?), we have established that there are four distinct possibilities that arise from different almost complex structures on $Q$. I conjecture that this classification and the resulting relations are true on the level of almost complex manifolds (i.e. those with the same Chern numbers \emph{are} the same). 

We denote the almost complex manifolds by $\P(TS^6)$, $\P(T^*S^6)$, $Q_\text{std}$ and $N$ (where the last one is the associated with the nearly K\"{a}hler structure). The following diagram summarizes our calculations:
\begin{equation*}
	\begin{tikzcd}[column sep=3.5cm,row sep=1.5cm]
		Q_\text{std} \ar[r,leftrightarrow,"\text{flip }S^6\text{-fibration}"] 
		\ar[d,leftrightarrow,"\text{flip }G_2/SO(4)\text{-fibration}"']
		& N \ar[d,leftrightarrow,"\text{flip }G_2/SO(4)\text{-fibration}"]\\
		\P(T^*S^6) \ar[r,leftrightarrow,"\text{flip }S^6\text{-fibration}"']
		& \P(TS^6)
	\end{tikzcd}
\end{equation*}

\section{Finished Computations on $Z$}

\subsection{Some Chern Numbers of $Z$ via the Hilbert Polynomial}

For this, we use the Hilbert polynomial (from Semmelman and Weingart). On page 159, Semmelman and Weingart give the expression 
\begin{align*}
	P(r)&=\frac{1}{120}(r+2)(3r+5)(2r+3)(3r+4)(r+1)\\
	&=\frac{3}{20}r^5+\frac{9}{8}r^4+\frac{10}{3}r^3+\frac{39}{8}r^2+\frac{211}{60}r+1
\end{align*}
for the Hilbert polynomial of $G_2/SO(4)$ (which is a so-called \emph{Wolf space}). On the other hand, Kotschick \& Terzic note that on a complex contact manifold (such as the twistor space $Z$) of complex dimension $2n+1$, there is a line bundle $L$ on $Z$ such that $c_1(Z)=(n+1)c_1(L)\eqqcolon (n+1)a$. The Hilbert polynomial can then be computed as follows:
\begin{equation*}
	P(r)=\chi(Z,L^r)=\sum_{i=0}^{2n+1}(-1)^i \dim_\C H^i(Z,L^r)
\end{equation*}
The Hirzebruch-Riemann-Roch theorem tells us that
\begin{equation*}
	P(r)=\langle \ch (L^r)\td(Z),[Z]\rangle
\end{equation*}
The Chern characters and Todd classes can be expressed in terms of the Chern classes as follows:
\begin{align*}
	\ch_1&=c_1 \\
	\ch_2&=\frac{1}{2}c_1^2-c_2(Z)\\
	\ch_3&=\frac{1}{6}c_1^3-\frac{1}{2}c_1c_2+\frac{1}{2}c_3\\
	\ch_4&=\frac{1}{24}c_1^4-\frac{1}{6}c_1^2c_2
	+\frac{1}{12}c_2^2+\frac{1}{6}c_1c_3-\frac{1}{6}c_4\\
	\ch_5&=\frac{1}{120}c_1^5-\frac{1}{24}c_1^3c_2+\frac{1}{24}c_1c_2^2
	+\frac{1}{24}c_1^2c_3-\frac{1}{24}c_2c_3-\frac{1}{24}c_1c_4+\frac{1}{24}c_5\\
	\td_1&=\frac{1}{2}c_1\\
	\td_2&=\frac{1}{12}c_1^2+\frac{1}{12}c_2\\
	\td_3&=\frac{1}{24}c_1c_2\\
	\td_4&=-\frac{1}{720}c_1^4+\frac{1}{180}c_1^2c_2+\frac{1}{240}c_2^2
	+\frac{1}{720}c_1c_3-\frac{1}{720}c_4\\
	\td_5&=-\frac{1}{1440}c_1^3c_2+\frac{1}{480}c_1c_2^2+\frac{1}{1440}c_1^2c_3
	-\frac{1}{1440}c_1c_4
\end{align*}
Note that $\ch(A\otimes B)=\ch A\cdot \ch B$, i.e. $\ch (L^r)=(\ch L)^r$. Since $L$ is rank one, we find
\begin{equation*}
\ch L=\sum_{k=0}^\infty \frac{a^k}{k!}=e^a\implies \ch L^r=e^{ra}
\end{equation*}
In particular, for the first five terms we have
\begin{align*}
	\ch L^r&=1+ra+\frac{r^2}{2}a^2+\frac{r^3}{6}a^3
	+\frac{r^4}{24}a^4+\frac{r^5}{120}a^5+\dots\\
	&=1+\frac{c_1}{3}r+\frac{c_1^2}{18}r^2+\frac{c_1^3}{162}r^3
	+\frac{c_1^4}{1944}r^4+\frac{c_1^5}{29160}r^5
\end{align*}
Where we used that in our case $\dim_\C Z=5$ and hence $c_1(Z)=3a$, i.e. $a=c_1(Z)/3$.

For the evaluation of $\ch(L^r)\td Z$ on $[Z]$, we are interested in the degree-ten component. In terms of Chern numbers, we find:
\begin{align*}
	\langle \ch(L^r)\td(Z),[Z]\rangle&=\frac{c_1^5}{29160}r^5+\frac{c_1^5}{3888}r^4
	+\frac{c_1^5+c_1^3c_2}{1944}r^3+\frac{c_1^3c_2}{432}r^2\\
	&+\frac{1}{2160}\bigg(-c_1^5+4c_1^3c_2+3c_1c_2^2
	+c_1^2c_3-c_1c_4\bigg)r\\
	&+\frac{1}{1440}\bigg(-c_1^3c_2+3c_1c_2^2+c_1^2c_3-c_1c_4\bigg)
\end{align*}
Matching coefficients with the expression from Semmelman and Weingart, we deduce the values of some Chern numbers:
\begin{align*}
	c_1^5&=\frac{29160\cdot 3}{20}=\frac{3888\cdot 9}{8}=4374\\
	c_1^3c_2&=\frac{1944\cdot 10}{3}-4374=\frac{432\cdot 39}{8}=2106\
\end{align*}
The lowest order terms give a relation between three remaining Chern numbers:
\begin{equation*}
	3c_1c_2^2+c_1^2c_3-c_1c_4=\frac{2160\cdot 211}{60}+4374-8424=1440+2106=3546
\end{equation*}

Note that if we fill in the entries given in the last table of Grama et al, this relation is satisfied.

\subsection{The Cohomology Ring of $Z$ and $Q$ is Different}

While computing the Chern numbers of $Q$, we showed that for $Q\subset \CP^6$, $c_1=5x$ with $x=\iota^*\alpha$ and $\alpha$ the standard generator of $H^2(\CP^6)$. 
\begin{equation*}
	c_1^5(Q)=6250=5^5\cdot x^5[Q]=5^5*2\implies x^5[Q]=2
\end{equation*}

We also computed the analogous quantities for $Z$. Recall that $c_1(Z)=3a$, where $a=c_1(L)$ is a generator of $H^2(G_2/U(2))$.
%WORK NEEDED: Why?!
Now, we know from the calculation of (some of) the Chern numbers that
\begin{equation*}
	c_1^5(Z)=4374=3^5\cdot a^5[Q]=3^5\cdot 18\implies a^5[Q]=18
\end{equation*}
Hence the multiplicative structure on the cohomology is not the same. We deduce that $Q$ and $Z$ are not even homotopy equivalent.

\subsection{Does the $S^2$-bundle $Z$ come from a complex bundle?}

In our case, the answer is no, as explained in Salamon's paper ``Quaternionic K\"{a}hler Manifolds''.

\subsection{Chern Classes of $Z$}

We know that, additively, $Z$ has the cohomology of $\CP^5$, i.e. the groups $H^{0,2,4,6,8,10}(Z)\cong\Z$ and the rest vanish.

\medskip

Call the positive generators $g_2,\dots,g_{10}$. Set $c_i=d_{2i}g_{2i}$. We will try to figure out the numbers $d_{2i}$ from the information we have on the Chern numbers. One of them is easy: $\langle c_5,[Z]\rangle=d_{10}=\chi(Z)=6$. For the others, we use the prime decomposition of the Chern numbers.

\subsubsection{Verified Chern Numbers}

For example:
\begin{equation*}
	c_1^5=d_2^5 \langle g_2^5,[Z]\rangle=4374=2\cdot 3^7
\end{equation*}
Since any factor of $d_2$ must appear with a power at least five, we see that $d_2=3$ or $d_2=1$. But $Z$ is the twistor space of a quaternionic K\"{a}hler manifold of dimension $8$. Such a space must have first Chern class with divisibility a multiple of $3$. (cf. Kotschick \& Terzic, page 600). Therefore $d_2=3$. The sign is ensured to be positive since $\langle g_2^5,[Z]\rangle$ must be positive, hence $18$. We continue in the same fashion:
\begin{equation*}
	c_1c_4=3d_8\langle g_2g_8,[Z]\rangle=90
	\implies d_8\langle g_2g_8,[Z]\rangle=30
\end{equation*}
By our discussion of the pairing of cohomology classes we know that $\langle g_2g_8,[Z]\rangle=1$, hence $d_8=30$.
\begin{equation*}
	c_1^3c_2=27d_4\langle g_2^3g_4,[Z]\rangle=2106
	\implies d_4\langle g_2^3g_4,[Z]\rangle=78=2\cdot 3\cdot 13
\end{equation*}
This does not give much information since we do not know $g_2^3$ in terms of $g_6$. However, since $g_2^3$ is a positive rational multiple of $[\omega]^3$ and $g_4$ a positive rational multiple of $[\omega]^2$, we see that $d_4$ must be non-negative.

\subsubsection{Chern Numbers taken from Grama et al.}

\begin{equation*}
	c_1c_2^2=3d_4^2\langle g_2g_4^2,[Z]\rangle=1014
	\implies d_4^2\langle g_2g_4^2,[Z]\rangle=338=2\cdot 13^2
\end{equation*}
Since any factor of $d_4$ must appear with power at least two, we see that $d_4\in \{1,13\}$.
\begin{equation*}
	c_1^2c_3=9d_6\langle g_2^2g_6,[Z]\rangle=594
	\implies d_6\langle g_2^2g_6,[Z]\rangle=66=2\cdot 3 \cdot 11
\end{equation*}
Hence $d_6\in\{1,2,3,11\}$. 
\begin{equation*}
	c_2c_3=d_4d_6\langle g_4g_6,[Z]\rangle=286\implies d_4d_6=286=2\cdot 11\cdot 13
\end{equation*}
Since $d_6$ is non-negative, this shows that $d_4$ is too. If $d_4=1$, we see that $d_6=286$, which is impossible because $d_3\langle g_2^2g_6,[Z]\rangle\geq d_3=66$. Therefore, $d_4=13$ and $d_6=22$.

Now, we have determined all the Chern classes; the result can be summarized in a table or by the total Chern class:
\begin{equation*}
	c(Z)=1+3g_2+13g_4+22g_6+30g_8+6g_{10}
\end{equation*}

\begin{table}[ht!]\centering
	\begin{tabular}{ll} \toprule
		Divisibility	& Value \\ \midrule
		$d_{10}$ 		& $6$ \\
		$d_8$ 			& $30$ \\
		$d_6$			& $22$ \\
		$d_4$			& $13$ \\
		$d_2$			& $3$ \\ \bottomrule
	\end{tabular}
\end{table}

A posteriori, we also see that $g_2^2=3g_4$, $g_2^3=6g_6$, $g_2^4=18g_8$, $g_2^5=18g_{10}$ and $g_4^2=2g_8$. 

\subsection{Chern numbers on $Z$ with Nearly K\"{a}hler Structure}

Since $Z$ is the twistor space of $G_2/SO(4)$, there is a hyperplane distribution $D$ such that $TZ\cong L\oplus D$ for a line bundle $L$. Let $N'$ denote the almost complex manifold obtained by replacing the complex structure on $Z$ with the nearly K\"{a}hler structure by flipping the fibers. Following Kotschick and Terzi\'{c} (pages 600-601), we see that $TN'\cong L^{-1}\oplus D$. In particular, we find:
\begin{prop}
	The total Chern class $c(N')$ is given by
	\begin{equation*}
		c(N')=c(Z)\frac{1-g_2}{1+g_2}=1+g_2+g_4-6g_6-18g_8-6g_{10}
	\end{equation*}
\end{prop}
\begin{myproof}
	By the direct sum decompositions of the tangent bundles, we have $c(N)=(1-c_1(L))c(D)$ and $c(Z)=(1+c_1(L))c(D)$. Now recall $c_1(Z)=3c_1(L)$. Note that $c_1(Z)$ is also $3g_2$, where $g_2$ is the positive generator of $H^2(Z)$. Since there is no torsion, the equality $3c_1(L)=3g_2$ allows us to conclude $c_1(L)=g_2$. The rest is a computation.
\end{myproof}

Computing the corresponding Chern numbers is now rather simple; one should observe that the almost complex structure $N'$ induces the opposite orientation on the underlying smooth manifold, i.e. $g_{10}$ is now a negative rather than a positive generator. The results are as follows:

\begin{table}[ht!]\centering
	\begin{tabular}{ll} \toprule
		Chern Class & Chern Number \\ \midrule
		$c_5$ 		& $6$ \\
		$c_1^5$ 	& $-18$ \\
		$c_1^3c_2$	& $-6$ \\
		$c_1^2c_3$	& $18$ \\
		$c_1c_4$	& $18$ \\
		$c_1c_2^2$	& $-2$ \\
		$c_2c_3$	& $6$ \\ \bottomrule
	\end{tabular}
	\caption{Chern Numbers of $Q$ with standard complex structure.}
\end{table}

\subsection{Consistency Checks for the Chern Classes of $Z$, $N'$}

\subsubsection{Pontryagin Classes}

Since Pontryagin classes are oriented diffeomorphism invariants, they should be the same for $Z,N'$. $p_1=c_1^2-2c_2$ and $p_2=2c_4-2c_1c_3+c_2^2$. We compute them for $Z$ and $N'$.

\medskip

For $Z$, we have 
\begin{equation*}
	p_1(Z)=(3g_2)^2-2(13g_4)=27g_4-26g_4=g_4
\end{equation*}
and 
\begin{equation*}
	p_2(Z)=2(30g_8)-2(3g_2)(22g_6)+(13g_4)^2=60g_8-132g_2g_6+169g_4^2=2g_8
\end{equation*}
while for $N'$ we find
\begin{equation*}
	p_1(N')=g_2^2-2g_4=g_4
\end{equation*}
and 
\begin{equation*}
	p_2(N')=2(-18g_8)-2g_2(-6g_6)+g_4^2=-36g_8+12g_2g_6+g_4^2=2g_8
\end{equation*}
which is consistent.

\subsubsection{Using the Hirzebruch-Riemann-Roch Theorem}

Earlier, we computed the Hilbert polynomial of the twistor space and used
\begin{align*}
	[\ch(L^r)\td(Z)]_5&=\frac{c_1^5}{29160}r^5+\frac{c_1^5}{3888}r^4
	+\frac{c_1^5+c_1^3c_2}{1944}r^3+\frac{c_1^3c_2}{432}r^2\\
	&+\frac{1}{2160}\bigg(-c_1^5+4c_1^3c_2+3c_1c_2^2
	+c_1^2c_3-c_1c_4\bigg)r\\
	&+\frac{1}{1440}\bigg(-c_1^3c_2+3c_1c_2^2+c_1^2c_3-c_1c_4\bigg)
\end{align*}
Where the $c_j$'s are the Chern \emph{classes} (not numbers). Using our expression for the Chern class:
\begin{equation*}
	c(Z)=1+3g_2+13g_4+22g_6+30g_8+6g_{10}
\end{equation*}
we get 
\begin{align*}
	[\ch(L^r)\td(Z)]_5&=\frac{g_2^5r^5}{120}+\frac{g_2^5r^4}{16}
	+\frac{g_2^5r^3}{8}+\frac{13g_2^3g_4r^3}{72}
	+\frac{13g_2^3g_4r^2}{16}\\
	&-\frac{9g_2^5r}{80}+\frac{13 g_2^3g_4r}{20}+\frac{169 g_2g_4^2r}{240}
	+\frac{11g_2^2g_6r}{120}-\frac{g_2g_8r}{24}\\
	&-\frac{39g_2^3g_4}{160}+\frac{169 g_2g_4^2}{160}
	+\frac{11g_2^2g_6}{80}-\frac{g_2g_8}{16}\\
	&=g_{10}\Bigg(\frac{3r^5}{20}+\frac{9r^4}{8}+\frac{9r^3}{4}
	+\frac{13r^3}{12}+\frac{39r^2}{8}\\
	&\qquad \quad -\frac{81 r}{40}+\frac{39 r}{10} +\frac{169r}{120} +\frac{11r}{40}-\frac{r}{24} \\
	&\qquad \quad -\frac{117}{80}+\frac{169}{80}+\frac{33}{80}-\frac{1}{16}\Bigg)\\
	&=g_{10}\Bigg(\frac{3}{20}r^5+\frac{9}{8}r^4+\frac{10}{3}r^3
	+\frac{39}{8}r^2+\frac{211}{60}r+1\Bigg)
\end{align*}
which precisely yields the expression for the Hilbert polynomial given by Semmelman and Weingart. Now, we can compute the same polynomial in the Chern classes for the nearly K\"{a}hler structure, using
\begin{equation*}
	c(N')=1+g_2+g_4-6g_6-18g_8-6g_{10}
\end{equation*}
We find
\begin{align*}
	&g_{10}\Bigg(\frac{1}{1620}r^5+\frac{1}{216}r^4
	+\frac{1}{81}r^3+\frac{1}{72}r^2\\
	&+\bigg(-\frac{1}{120}+\frac{1}{90}+\frac{1}{360}
	-\frac{1}{120}+\frac{1}{120}\bigg)r\\
	&+\bigg(-\frac{1}{240}+\frac{1}{240}-\frac{1}{80}+\frac{1}{80}\bigg)\Bigg)\\
	=&g_{10}\Bigg(\frac{1}{1620}r^5+\frac{1}{216}r^4+\frac{1}{81}r^3+\frac{1}{72}r^2
	+\frac{1}{180}r\Bigg)
\end{align*}
The main point of this calculation is actually the constant terms of these polynomials. This is simply the \emph{Todd genus} of $Z$ or $N'$. On $Z$, it is known to equal $1$ while with regards to $N'$, we can say that Kotschick observed that on many examples of flag manifolds equipped with a nearly K\"{a}hler structure coming from a twistor space construction, the Todd genus vanishes. This is all in accordance with our above calculations.



\end{document}