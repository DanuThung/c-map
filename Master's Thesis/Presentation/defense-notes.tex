\documentclass[parskip=half]{scrartcl}
% % % % % PACKAGES

%General Packages

\usepackage[automark]{scrlayer-scrpage}																	
\usepackage{amsfonts}												%Mathematics fonts
\usepackage{mathtools}												%General mathematics symbols
\usepackage{amsmath}
\usepackage{stmaryrd}												%Extra math symbols
\usepackage{amssymb}												%More symbols
\usepackage{extarrows}												%Extendible arrows
\usepackage{dsfont} 												%Identity matrix symbol
\usepackage{mathrsfs}												%To get mathscr
\usepackage{relsize}												%Scaling symbols with reference to pre-existing symbols
\usepackage{accents}												%Accents on math symbols
\usepackage[T1]{fontenc}											%Accents output improvement
\usepackage[latin1]{inputenc}										%Accents input improvement
\usepackage[english]{babel}
\usepackage{subcaption} 											%Subfigures etc
\usepackage{cancel} 												%Striking through things
\usepackage{setspace}												%line spacing
\usepackage[top=1in,bottom=1in,left=1.25in,right=1.25in]{geometry}	%margins
\usepackage[symbol]{footmisc}										%Some footnote margin thing
\usepackage{enumerate} 												%Numbered lists
\usepackage{booktabs}

%Pictures & TikZ Packages

\usepackage{graphicx}												%Pictures	
\usepackage{epstopdf}												%Converts .eps to .pdf files
\usepackage{tikz}													%TikZ Drawings
\usetikzlibrary{3d,patterns,arrows,bending,arrows.meta,				%TikZ Libraries
	shapes.geometric,knots,intersections,
	decorations.markings,decorations.pathmorphing,
	decorations.pathreplacing}										
\usepackage{tikz-cd}												%Commutative Diagrams


%Mathematics Packages

\usepackage{amsthm}													%Theorems etc

%Referencing

\usepackage[colorlinks=true]{hyperref}								%hyperlinks
\usepackage[noabbrev]{cleveref}										%better croff-refs
\crefname{prop}{proposition}{propositions}
\crefname{lem}{lemma}{lemmata}						


% % % % % CUSTOM COMMANDS

%Derivatives/Differentials

\let\underdot=\d
\newcommand{\od}[2]{\frac{\mathrm{d} #1}{\mathrm{d} #2}}
\newcommand{\odd}[2]{\frac{\mathrm{d}^2 #1}{\mathrm{d} #2^2}}
\newcommand{\p}{\partial}
\newcommand{\pd}[2]{\frac{\partial #1}{\partial #2}}
\newcommand{\pdd}[2]{\frac{\partial^2 #1}{\partial #2^2}}
\newcommand{\fd}[2]{\frac{\delta #1}{\delta #2}}
\renewcommand{\d}{\mathrm{d}}
\newcommand{\dif}{D}


%Common Sets/Spaces

\newcommand{\RP}{\mathbb{R}\mathrm{P}}
\newcommand{\CP}{\mathbb{C}\mathrm{P}}
\newcommand{\HP}{\mathbb{H}\mathrm{P}}
\renewcommand{\P}{\mathbb{P}}
\newcommand{\N}{\mathbb{N}}
\newcommand{\Z}{\mathbb{Z}}
\newcommand{\Q}{\mathbb{Q}}
\newcommand{\R}{\mathbb{R}}
\newcommand{\C}{\mathbb{C}}
\renewcommand{\H}{\mathbb{H}}
\renewcommand{\O}{\mathbb{O}}

%Math-operators



\renewcommand{\Im}{\operatorname{Im}}
\renewcommand{\Re}{\operatorname{Re}}

\DeclareMathOperator{\Graph}{graph}
\DeclareMathOperator{\Gr}{Gr}

\DeclareMathOperator{\im}{im}
\DeclareMathOperator{\rank}{rank}
\DeclareMathOperator{\ord}{ord}
\DeclareMathOperator{\tr}{tr}
\DeclareMathOperator{\incl}{incl}
\DeclareMathOperator{\pr}{proj}
\DeclareMathOperator{\diag}{diag}
\DeclareMathOperator{\Span}{span}
\DeclareMathOperator{\codim}{codim}

\DeclareMathOperator{\Hom}{Hom}
\DeclareMathOperator{\End}{End}
\DeclareMathOperator{\Aut}{Aut}
\DeclareMathOperator{\coker}{coker}
\DeclareMathOperator{\Stab}{Stab}
\DeclareMathOperator{\Diff}{Diff}
\DeclareMathOperator{\Bs}{Bs}
\DeclareMathOperator{\id}{id}
\DeclareMathOperator{\Mat}{Mat}
\newcommand{\Unit}{\mathds{1}}

\DeclareMathOperator{\td}{td}
\DeclareMathOperator{\ch}{ch}
\DeclareMathOperator{\Spin}{Spin}
\newcommand{\Spinc}{\Spin^c}

\DeclareMathOperator{\Ad}{Ad}
\DeclareMathOperator{\ad}{ad}

\DeclareMathOperator{\supp}{supp}
\DeclareMathOperator{\interior}{int}
\DeclareMathOperator{\vol}{vol}

\DeclareMathOperator{\sgn}{sgn}

\DeclareMathOperator{\Tor}{Tor}
\DeclareMathOperator{\Ext}{Ext}
\DeclareMathOperator*{\free}{\scalerel*{\ast}{\scaleobj{1}{\sum}}}

\newcommand{\trans}{\mathrel{\text{\tpitchfork}}}
\makeatletter
\newcommand{\tpitchfork}{%
	\vbox{
		\baselineskip\z@skip
		\lineskip-.52ex
		\lineskiplimit\maxdimen
		\m@th
		\ialign{##\crcr\hidewidth\smash{$-$}\hidewidth\crcr$\pitchfork$\crcr}
	}%
}
\makeatother

%Other

\newcommand{\action}{\curvearrowright}
\newcommand{\rightaction}{\curvearrowleft}

\newcommand{\ubar}[1]{\underaccent{\bar}{#1}}
\def\mathunderline#1#2{\color{#1}\underline{{\color{black}#2}}\color{black}}

\newcommand{\abs}[1]{\left\lvert #1 \right\rvert}
\newcommand{\norm}[1]{\left\lVert #1 \right\rVert}
\newcommand{\expvalue}[1]{\left\langle #1 \right\rangle}

\setlength{\parindent}{0pt}

\newcommand{\mf}[1]{\mathfrak{#1}}
\newcommand{\mc}[1]{\mathcal{#1}}
\newcommand{\ms}[1]{\mathscr{#1}}

\newcommand{\bdy}{\partial}
\newcommand{\pt}{\mathrm{pt}}
\DeclareMathOperator{\Bl}{Bl}


%Theorem Styles

\newtheoremstyle{mythm}% name of the style to be used
{}% measure of space to leave above the theorem. E.g.: 3pt
{}% measure of space to leave below the theorem. E.g.: 3pt
{\slshape}% name of font to use in the body of the theorem
{}% measure of space to indent
{\bfseries\sffamily}% name of head font
{.}% punctuation between head and body
{ }% space after theorem head; " " = normal interword space
{}% Manually specify head
\newtheoremstyle{mydef}% name of the style to be used
{}% measure of space to leave above the theorem. E.g.: 3pt
{}% measure of space to leave below the theorem. E.g.: 3pt
{}% name of font to use in the body of the theorem
{}% measure of space to indent
{\bfseries\sffamily}% name of head font
{.}% punctuation between head and body
{ }% space after theorem head; " " = normal interword space
{}% Manually specify head

\theoremstyle{mythm}
\newtheorem{thm}{Theorem}[section]
\newtheorem{prop}[thm]{Proposition}
\newtheorem{cor}[thm]{Corollary}
\newtheorem{lem}[thm]{Lemma}
\theoremstyle{mydef}
\newtheorem{mydef}[thm]{Definition}
\newtheorem{rem}[thm]{Remark}
\newtheorem{ex}[thm]{Example}
\newtheorem{exer}{Exercise}[subsection]
\newenvironment{myproof}[1][\proofname]{
	\proof[\sffamily\upshape#1]
}{\endproof}

\newcommand{\proofclear}{\hfill \qedsymbol}

% % % % % MISCELLANEOUS STUFF

\clearscrheadfoot
\ihead[]{}
\ohead[]{}
\cfoot[]{\pagemark}
\pagestyle{scrheadings}

\deffootnote[1em]{0em}{1em}{%
	\textsuperscript{\thefootnotemark}%
}
\setfootnoterule{3em}


\newcommand\numberthis{\stepcounter{equation}\tag{\theequation}}


\newenvironment{numberedlist}{\begin{enumerate}[\upshape(i)]}{\end{enumerate}}
\newenvironment{letteredlist}{\begin{enumerate}[\upshape a)]}{\end{enumerate}}

\renewcommand{\thesection}{\arabic{section}}
\renewcommand{\thesubsection}{(\alph{subsection})}
\renewcommand{\thesubsubsection}{(\roman{subsubsection})}
\renewcommand{\autodot}{}

%Inverse diagonal dots:

\makeatletter
\def\Ddots{\mathinner{\mkern1mu\raise\p@
		\vbox{\kern7\p@\hbox{.}}\mkern2mu
		\raise4\p@\hbox{.}\mkern2mu\raise7\p@\hbox{.}\mkern1mu}}
\makeatother


%TikZ

\tikzset{% 
	arrowat/.style={%
		postaction={decorate,decoration={
				markings,
				mark=at position #1 with {\arrow[xshift=2pt]{>}}}}
	}
}

\tikzset{% 
	backarrowat/.style={%
		postaction={decorate,decoration={
				markings,
				mark=at position #1 with {\arrow[xshift=2pt]{<}}}}
	}
}
\title{Invariant Geometric Structures and \\ Chern numbers of  $G_2$ Flag Manifolds}
\author{Dani\"el Thung}
\date{}
\begin{document}
\maketitle

\section{Outline}

First of all, I would like to thank everybody for being here. Today, I'll be talking about some work that I did together with professor Kotschick, my supervisor.

In my thesis, I mostly studied certain manifolds which are homogeneous spaces under $G_2$. They fall into a special class called \emph{generalized flag manifolds}. I'm going to start by explaining what a generalized flag manifold is. After that, I will introduce the exceptional Lie group $G_2$ and some of its homogeneous spaces, including its flag manifolds.

Then, I will give a more detailed description of two of them, which were my main objects of study. I will call them the \emph{twistor space} and the \emph{quadric}, and I'll explain these names in due time.

\section{Introduction}

Before giving the definition of a generalized flag manifold, I would like to motivate it by recalling what a classical flag manifold is. A flag is a strictly increasing sequence of subspaces of a vector space, $\{0\}=V_0\subset V_1\subset \dots \subset V_k=V$, where $V$ is an $n$-dimensional vector space. The dimension gets larger at every step, and in the special case where every dimension occurs we have a complete flag. Otherwise, we have a partial flag. The dimensions are encoded in the signature of the flag. For any given signature, a \emph{flag manifold} is the space of flags of this type.

Let's consider the case $V=\C^n$; the fact that any orthonormal basis may be turned into any other by means of a unitary transformation means that $U(n)$ and even $SU(n)$ can also turn any subspace into any other (of the same dimension) and similarly can transform any flag of given signature into any other of the same signature: This implies that any manifold of flags in $\C^n$ is a homogeneous space under $U(n)$ and even $SU(n)$. 

The simplest flag manifolds are those parametrizing flags consisting of a single $k$-plane: These are the Grassmannians. To express it as a homogeneous space, we have to find the stabilizer of a point. Consider the standard embedding of $\C^k$ into $\C^n$ onto the first $k$ coordinates. The unitary transformations that preserve this subspace correspond to the block-diagonal matrices $U(k)\times U(n-k)$. Thus, we find that 
\begin{equation*}
	\Gr_k(\C^n)\cong \frac{SU(n)}{S(U(k)\times U(n-k))}
\end{equation*}
Similarly, it is not hard to see that the manifold of complete flags is simply the quotient of $U(n)$ or $SU(n)$ by a maximal torus. Generalizing these simple cases, a manifold of flags in $\C^n$ is the quotient of $U(n)$ or $SU(n)$ by the \emph{centralizer} of a torus.

Now, the obvious generalization to compact, connected, semisimple Lie groups and defines a \emph{generalized flag manifold}: It is a quotient of such a Lie group by the centralizer of a torus. In my thesis, I study the geometry of certain generalized flag manifolds. 

The first question that one should ask is: Why should one study these spaces? What's interesting about them? To answer that question, recall that, on homogeneous spaces, there are certain privileged geometric structures, namely those compatible with the group translations, i.e.~the $G$-invariant ones. It turns out that, when it comes to $G$-invariant structures, generalized flag manifolds have some remarkable properties. Perhaps the most famous result is that generalized flag manifolds admit an invariant complex structure, which I'll call the ``canonical'' complex structure, and even a compatible, invariant  K\"ahler-Einstein metric, unique up to scaling, with positive scalar curvature. In fact, any manifold with a invariant K\"ahler metric is biholomorphic to a generalized flag manifold, equipped with the canonical invariant complex structure.

Besides this integrable complex structure, there are typically several invariant \emph{almost} complex structures. Invariant almost complex structure on arbitrary homogeneous spaces were studied in Lie-theoretic terms by Borel and Hirzebruch, in a famous series of papers published in the late 1950's. They gave a simple method to determine how many invariant almost complex structures there are, and a way to calculate interesting quantities such as Chern and Pontryagin numbers associated with the almost complex structures. In their work, flag manifolds popped up as simple examples of manifolds which admit multiple invariant almost complex structures with \emph{distinct} Chern numbers. This is a somewhat rare and therefore interesting phenomenon, which we will return to later.

All these invariant structures have been studied and relatively well-understood for a long time in terms of Lie theory. For instance, the existence result for the canonical invariant complex structure and the invariant K\"ahler-Einstein metric are all done in terms of roots and weights. This is natural, since it makes maximal use of homogeneity, which reduces geometric questions to algebraic ones. However, this algebraic approach somewhat obscures what's ``really going on'': How does one interpret, for instance, an invariant K\"ahler-Einstein metric? Where is it coming from?

These are the kind of questions that my thesis tries to answer. Taking a point of view that is in some sense complementary to the usual Lie-theoretic approach, we use differential-geometric techniques instead. We use them to study the invariant almost complex structures on two specific flag manifolds, including the canonical integrable structure and its K\"ahler-Einstein metric. As we will see, having a geometric picture of the situation naturally yields a clear interpretation to the invariant geometric structures. As the main application, we use our geometric understanding to compute the Chern numbers corresponding to all the invariant almost complex structures. At the same time, the geometric approach reveals a number of connections to other interesting topics, ranging from rigidity theorems for K\"ahler manifolds to the question of existence of complex structures on $S^6$. 

\section{\texorpdfstring{$G_2$}{G2} and its Flag Manifolds}

To introduce the examples I studied, recall that $G_2$, the smallest of the exceptional, simple Lie groups, can be viewed as the automorphism group of the octonion algebra. The non-trivial structure of the octonions is essentially contained in the imaginary part of the product if imaginary octonions, which is called the \emph{cross product}, and has properties similar to the familiar cross product in $\R^3$, and $G_2$ can equivalently be defined as the automorphism group of $\Im \O$, equipped with the cross product. As a vector space over $\R$, this is simply $\R^7$ and via this identification we find some interesting actions of $G_2$.

Recall that it has rank two, and therefore its maximal tori are two-dimensional. Since a maximal torus equals its own centralizer, this means that the ``complete'' $G_2$ flag manifold is $G_2/T^2$. Now, the $T^2$ only has two sub-tori, corresponding to the two circle factors, and $G_2/T^2$ fibers over both of them. The two other $G_2$ flag manifolds are both quotients of $G_2$ by a subgroup isomorphic to $U(2)$; but they are not conjugate, as we will prove later. To distinguish them, we denote them by $U(2)_\pm$. One of them can be nicely understood as the Grassmannian of oriented 2-planes in $\R^7$, viewed as the imaginary octonions, on which $G_2$ acts transitively. This action induces a transitive action on the six-sphere of unit imaginary octonions: We get a fibration over $S^6$, realized as a $G_2$-homogeneous space (\emph{not} a generalized flag manifold).

The final $G_2$ flag manifold, $G_2/U(2)_+$, also admits an octonionic description, but it is a lot less elegant and, as we will soon see, there is a better alternative. To give this better description, we will make use of one more ``auxiliary manifold'', the homogeneous space $G_2/SO(4)$. This is the space of all subalgebras of $\O$ isomorphic to $\H$. Both partial flag manifolds fiber over it, with fiber $\CP^1$. From now on, we will restrict our attention to the two partial flag manifolds, indicated in red. I will call them $Q$ and $Z$, respectively; note that they are both of real dimension ten, hence will be complex manifolds of dimension five.

\subsection{The Twistor Space}

We start with the study of $Z$. The fibration over $G_2/SO(4)$ exhibits it as the so-called \emph{twistor space} over $G_2/SO(4)$, which is a quaternionic K\"ahler manifold. To make sense out these words, let's first introduce quaternionic K\"ahler manifolds. These are oriented manifolds whose dimension is divisible by four, at least eight, and whose holonomy group is contained in the subgroup $Sp(n)\cdot Sp(1)$ of $SO(4n)$. One important feature of quaternionic K\"ahler manifolds is that they are always Einstein (hence have constant scalar curvature) but not necessarily Ricci-flat: Among the classes of manifolds with reduced holonomy, K\"ahler and quaternionic K\"ahler manifolds are the only ones with this property.

A quaternionic K\"ahler manifold is quaternionic in the sense that there are \emph{locally} (but in general not globally) defined almost complex structures $I,J$ and $IJ=K$ on the tangent bundle, compatible with the metric, which span a rank three subbundle of the endormorphism bundle which \emph{is} globally well-defined, even if these almost complex structures are not. The sphere bundle of this rank three vector bundle is what one calls the twistor space of the quaternionic K\"ahler manifold. Note that every point $z\in S(E)$ corresponds to an almost complex structure on $T_{\pi(z)}M$, compatible with the metric on $M$.

The usefulness of this construction derives from the fact that the twistor space comes equipped with a complex structure. This complex structure is defined as follows: Since $S(E)$ is associated to $Sp(n)\cdot Sp(1)$-principal frame bundle of $M$, the Levi-Civit\`a connection on $M$ induces a splitting $TS(E)=\mc V\oplus \mc H$, where in every point $z\in S(E)$, $\pi_*$ induces a linear isomorphism $T_{\pi(z)}M\cong \mc H_z$. Now, we will define the complex structure $J$ as a direct sum $J_v\oplus J_h$. $J_v$ is simple to define: Each fiber is just a copy of $\CP^1$, and we let $J_v$ be the standard complex structure of $\CP^1$. On the horizontal subbundle, we use the fact that $S(E)$ is a bundle of complex structures. Since $z\in S(E)$ is \emph{itself} a complex structure on $T_{\pi(z)}M$, we may use the isomorphism with $\mc H_z$ to define $J_h$. The Nijenhuis tensor is explicitly shown to vanish. 

I will not say much about another important property, namely that this complex structure admits a compatible K\"ahler-Einstein metric, in case the base manifold has positive scalar curvature. The metric is constructed as follows: Both the base, which is quaternionic K\"ahler, and the fibers, which are copies of $\CP^1$, admit Einstein metrics. The metric on the twistor space is, up to some scaling factors, simply the direct sum of the Einstein metric on $M$ with the Fubini-Study metric on $\CP^1$. One explicitly checks that the complex structure is parallel and that the resulting metric is Einstein.

This discussion already yields a geometric interpretation of the canonical invariant complex structure and the compatible invariant K\"ahler-Einstein metric on the space $G_2/U(2)_+$: These structures naturally arise from the fact that it is a twistor space over $G_2/SO(4)$, which is a quaternionic K\"ahler manifold with positive scalar curvature.

Traditionally, one uses the twistor space to study the geometry of the underlying quaternionic K\"ahler manifold. However, in our was we are more interested in $Z$ itself, and use its description as a sphere bundle over $G_2/SO(4)$ to study it. We have a Gysin sequence, which allows us to compute the cohomology ring and Pontryagin classes of $Z$ from those of $G_2/SO(4)$. Additively, the cohomology groups coincide with $\CP^5$, but not multiplicatively.

Perhaps surprisingly, this topological information turns out to give a lot of control over the manifold. In particular, we were able to prove the following result: Assume we have a manifold homeomorphic to the twistor space $Z$ which is furthermore K\"ahler, then it is even biholomorphic to the twistor space, equipped with its natural complex structure. This rigidity result for the K\"ahler structure on this manifold is an analog of classic uniqueness theorems due to Hirzebruch, Kodaira, Yau and Brieskorn. 

Hirzebruch and Kodaira proved that any manifold homeomorphic to $\CP^n$ is biholomorphic to it (under a technical assumption removed by Yau's solution of the Calabi conjecture), and Brieskorn proved the analogous statement for quadrics of dimension at least three.

Mimicking the proofs of these classic results, the main step in the proof is to determine the first Chern class. To do so, we use the simple structure of the cohomology. First of all, since the cohomology in even degrees is infinite cyclic, we only need to find which multiple of the positive generator it is. Here, the notion of positivity is defined by the K\"ahler metric. Secondly, since $X$ is K\"ahler by assumption, the Hodge numbers are those of $\CP^5$. The Chern number $c_1c_4$ is determined by the Hodge numbers (this was discovered by Libgober and Wood), and therefore coincides with that of $\CP^5$; this means that it is $90$.

Now, we see that $d$ must divide $90$ (but it can also be negative). The divisors can be checked one by one, and using the topology, the Pontryagin classes and further constraints coming from the Hirzebruch-Riemann-Roch theorem one arrives at a contradiction in every case except $d=3$.

Now that the first Chern class is determined, we have found that $X$ must be Fano, with Fano index three. The highest possible index for a Fano $n$-manifold is $n+1$, or six in our case. Thus, $X$ has Fano \emph{coindex} three. Fano manifolds with coindex three were classified by Mukai, under a technical assumption that was later removed by Mella. The classification of Mukai shows that a manifold with the cohomology ring of $Z$ and Fano coindex three must in fact be biholomorphic to $Z$, finishing our proof.

Knowing the first Chern class means knowing the entire Chern class, via the number $c_1c_4$ and the Pontryagin classes. Thus, it is not hard to obtain the Chern numbers for the twistor space $Z$. However, we are also interested in \emph{other} invariant almost complex structures. The classic papers by Borel and Hirzebruch include a simple recipe for determining the number of invariant almost complex structures (up to conjugation) of a homogeneous space, by counting irreducible summands of the so-called isotropy representation, a representation of the isotropy subgroup. On $Z$, there are two invariant almost complex structures up to conjugation.

The second one is obtained from the first one by what we call ``flipping the fiber''. It has a very simple geometric description: Simply replace the complex structure by non-integrable structure obtained by flipping the sign of $J_v$. This amounts to replacing the vertical subbundle by its complex conjugate. The resulting almost complex structure is of course still invariant, but it can be distinguished from the old one by the Chern numbers, which can easily be determined from the old ones, using the above description. We denote it by $N$ because it admits a compatible nearly K\"ahler metric.

One important thing to mention here is that the Chern numbers of the invariant almost complex structures on $Z$ were also computed in a paper published earlier this year by Grama, Negreiros and Oliveira. However, their calculation follows the Lie-theoretic recipe proposed by Borel and Hirzebruch and is therefore very different from ours. Furthermore, there are some errors in the numbers they get out. In any case, the Chern numbers are listed in the table. 

\subsection{The Quadric}

Now, we turn to the second flag manifold Let us start with some basic information about the space of oriented two-planes in $\Im\O$. Our first objective is to give a geometric interpretation of this manifold's invariant complex structure and the compatible K\"ahler-Einstein metric. This is done as follows: Let $P$ be an oriented 2-plane and let $\{e_1,e_2\}$ be a positive, orthonormal basis; note that this basis is unique up to a $U(1)$-rotation.

Now we complexify $\R^7$ to obtain $\C^7$, and consider the $\C$-bilinear extension of the inner product on $\R^7$. Then clearly $e_1+ie_2$ has zero ``inner product'' with itself and hence defines a point in the so-called \emph{zero quadric}, which is the vanishing locus of the extended inner product inside $\CP^6$. This map is actually a diffeomorphism, and reveals our reason for calling this space $Q$: It is a quadric!

This immediately gives us an interpretation of the invariant complex structure and K\"ahler-Einstein metric: They are inherited from the standard complex structure and the Fubini-Study metric on $\CP^6$, which restricts to an $SO(7)$-invariant K\"ahler-Einstein metric on $Q$. The earlier-mentioned rigidity theorem of Brieskorn for quadric hypersurfaces shows that this is the only complex structure that admits a K\"ahler metric.

%WORK NEEDED EXPAND THE NEXT PARAGRAPH
To find the other invariant almost complex structures, we need a little bit of octonionic algebra. The six-sphere admits a $G_2$-invariant complex structure, which at the point $x\in S^6$ is simply left-multiplication by $x$. This turns its tangent bundle into a complex vector bundle, so we can form the projectivization $\P(TS^6)$. It admits a $G_2$-invariant almost complex structure, constructed as the sum of the $G_2$-invariant almost complex structure on $S^6$ with the standard complex structure on $\CP^2$ on the vertical subbundle.

Now, a complex line in $T_x S^6$ can also be regarded as an oriented 2-plane. This identification sets up a diffeomorphism of $\P(TS^6)$ to $Q$. Thus, we have found a second invariant almost complex structure on $Q$. A third invariant almost complex structure is obtained by recognizing that the projectivized tangent bundle is diffeomorphic to the projectivized \emph{cotangent} bundle as well.

Now, if the $G_2$ invariant almost complex structure on $S^6$ were integrable, then we would have obtained two non-standard (non-K\"ahler, by Brieskorn's result!) complex structures on the quadric. This is of course known not to be the case, but in fact we obtain the same result if there is \emph{any} integrable almost complex structure on $S^6$. This is because of the fact that all almost complex structures on $S^6$ are homotopic as almost complex structures. A homotopy will produce a diffeomorphism between projectivized tangent bundles, hence even with a \emph{different} almost complex structure we will obtain complex manifolds diffeomorphic to the quadric. Thus, one might try to rule out the existence of a complex structure on $S^6$ by ruling out non-standard complex structures on $Q$. This is an analog of a well-known argument that relates complex structures on $S^6$ to complex structures on $\CP^3$, by blowing up a point.

We have now found three invariant almost complex structures, though we haven't proven yet that they are actually different. The work of Borel and Hirzebruch implies that there are four in total, so we are still looking for a fourth. The procedure of flipping the fiber on the twistor space inspires one to try a similar thing for $Q$, using the fibrations over $S^6$ and $G_2/SO(4)$. As before, the tangent vectors to the fibers yield complex subbundle, which we can flip. Flipping the fiber over $S^6$ yields a new invariant almost complex structure from the standard one, which can be distinguished by its Chern numbers. This gives us an interpretation of all the invariant almost complex structures. 

The Chern numbers of the standard structures are easy to calculate via the adjunction formula, and those of the fourth invariant almost complex structure are found through our description of flipping the fiber. For the remaining invariant almost complex structure one pulls back the tangent or cotangent bundle of $S^6$ to the total space of their projectivizations and uses a relative Euler sequence to determine the Chern class of the vertical subbundle: One obtains the total Chern class and hence the Chern numbers. Alternatively, it turns out that actually one can obtain \emph{all four} invariant almost complex structures from the standard one, by flipping the fibers over $S^6$ and $G_2/SO(4)$.




\end{document}