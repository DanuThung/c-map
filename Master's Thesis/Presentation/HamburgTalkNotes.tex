\documentclass[parskip=half]{scrartcl}
% % % % % PACKAGES

%General Packages

\usepackage[automark]{scrlayer-scrpage}																	
\usepackage{amsfonts}												%Mathematics fonts
\usepackage{mathtools}												%General mathematics symbols
\usepackage{amsmath}
\usepackage{stmaryrd}												%Extra math symbols
\usepackage{amssymb}												%More symbols
\usepackage{extarrows}												%Extendible arrows
\usepackage{dsfont} 												%Identity matrix symbol
\usepackage{mathrsfs}												%To get mathscr
\usepackage{relsize}												%Scaling symbols with reference to pre-existing symbols
\usepackage{accents}												%Accents on math symbols
\usepackage[T1]{fontenc}											%Accents output improvement
\usepackage[latin1]{inputenc}										%Accents input improvement
\usepackage[english]{babel}
\usepackage{subcaption} 											%Subfigures etc
\usepackage{cancel} 												%Striking through things
\usepackage{setspace}												%line spacing
\usepackage[top=1in,bottom=1in,left=1.25in,right=1.25in]{geometry}	%margins
\usepackage[symbol]{footmisc}										%Some footnote margin thing
\usepackage{enumerate} 												%Numbered lists
\usepackage{booktabs}

%Pictures & TikZ Packages

\usepackage{graphicx}												%Pictures	
\usepackage{epstopdf}												%Converts .eps to .pdf files
\usepackage{tikz}													%TikZ Drawings
\usetikzlibrary{3d,patterns,arrows,bending,arrows.meta,				%TikZ Libraries
	shapes.geometric,knots,intersections,
	decorations.markings,decorations.pathmorphing,
	decorations.pathreplacing}										
\usepackage{tikz-cd}												%Commutative Diagrams


%Mathematics Packages

\usepackage{amsthm}													%Theorems etc

%Referencing

\usepackage[colorlinks=true]{hyperref}								%hyperlinks
\usepackage[noabbrev]{cleveref}										%better croff-refs
\crefname{prop}{proposition}{propositions}
\crefname{lem}{lemma}{lemmata}						


% % % % % CUSTOM COMMANDS

%Derivatives/Differentials

\let\underdot=\d
\newcommand{\od}[2]{\frac{\mathrm{d} #1}{\mathrm{d} #2}}
\newcommand{\odd}[2]{\frac{\mathrm{d}^2 #1}{\mathrm{d} #2^2}}
\newcommand{\p}{\partial}
\newcommand{\pd}[2]{\frac{\partial #1}{\partial #2}}
\newcommand{\pdd}[2]{\frac{\partial^2 #1}{\partial #2^2}}
\newcommand{\fd}[2]{\frac{\delta #1}{\delta #2}}
\renewcommand{\d}{\mathrm{d}}
\newcommand{\dif}{D}


%Common Sets/Spaces

\newcommand{\RP}{\mathbb{R}\mathrm{P}}
\newcommand{\CP}{\mathbb{C}\mathrm{P}}
\newcommand{\HP}{\mathbb{H}\mathrm{P}}
\renewcommand{\P}{\mathbb{P}}
\newcommand{\N}{\mathbb{N}}
\newcommand{\Z}{\mathbb{Z}}
\newcommand{\Q}{\mathbb{Q}}
\newcommand{\R}{\mathbb{R}}
\newcommand{\C}{\mathbb{C}}
\renewcommand{\H}{\mathbb{H}}
\renewcommand{\O}{\mathbb{O}}

%Math-operators



\renewcommand{\Im}{\operatorname{Im}}
\renewcommand{\Re}{\operatorname{Re}}

\DeclareMathOperator{\Graph}{graph}
\DeclareMathOperator{\Gr}{Gr}

\DeclareMathOperator{\im}{im}
\DeclareMathOperator{\rank}{rank}
\DeclareMathOperator{\ord}{ord}
\DeclareMathOperator{\tr}{tr}
\DeclareMathOperator{\incl}{incl}
\DeclareMathOperator{\pr}{proj}
\DeclareMathOperator{\diag}{diag}
\DeclareMathOperator{\Span}{span}
\DeclareMathOperator{\codim}{codim}

\DeclareMathOperator{\Hom}{Hom}
\DeclareMathOperator{\End}{End}
\DeclareMathOperator{\Aut}{Aut}
\DeclareMathOperator{\coker}{coker}
\DeclareMathOperator{\Stab}{Stab}
\DeclareMathOperator{\Diff}{Diff}
\DeclareMathOperator{\Bs}{Bs}
\DeclareMathOperator{\id}{id}
\DeclareMathOperator{\Mat}{Mat}
\newcommand{\Unit}{\mathds{1}}

\DeclareMathOperator{\td}{td}
\DeclareMathOperator{\ch}{ch}
\DeclareMathOperator{\Spin}{Spin}
\newcommand{\Spinc}{\Spin^c}

\DeclareMathOperator{\Ad}{Ad}
\DeclareMathOperator{\ad}{ad}

\DeclareMathOperator{\supp}{supp}
\DeclareMathOperator{\interior}{int}
\DeclareMathOperator{\vol}{vol}

\DeclareMathOperator{\sgn}{sgn}

\DeclareMathOperator{\Tor}{Tor}
\DeclareMathOperator{\Ext}{Ext}
\DeclareMathOperator*{\free}{\scalerel*{\ast}{\scaleobj{1}{\sum}}}

\newcommand{\trans}{\mathrel{\text{\tpitchfork}}}
\makeatletter
\newcommand{\tpitchfork}{%
	\vbox{
		\baselineskip\z@skip
		\lineskip-.52ex
		\lineskiplimit\maxdimen
		\m@th
		\ialign{##\crcr\hidewidth\smash{$-$}\hidewidth\crcr$\pitchfork$\crcr}
	}%
}
\makeatother

%Other

\newcommand{\action}{\curvearrowright}
\newcommand{\rightaction}{\curvearrowleft}

\newcommand{\ubar}[1]{\underaccent{\bar}{#1}}
\def\mathunderline#1#2{\color{#1}\underline{{\color{black}#2}}\color{black}}

\newcommand{\abs}[1]{\left\lvert #1 \right\rvert}
\newcommand{\norm}[1]{\left\lVert #1 \right\rVert}
\newcommand{\expvalue}[1]{\left\langle #1 \right\rangle}

\setlength{\parindent}{0pt}

\newcommand{\mf}[1]{\mathfrak{#1}}
\newcommand{\mc}[1]{\mathcal{#1}}
\newcommand{\ms}[1]{\mathscr{#1}}

\newcommand{\bdy}{\partial}
\newcommand{\pt}{\mathrm{pt}}
\DeclareMathOperator{\Bl}{Bl}


%Theorem Styles

\newtheoremstyle{mythm}% name of the style to be used
{}% measure of space to leave above the theorem. E.g.: 3pt
{}% measure of space to leave below the theorem. E.g.: 3pt
{\slshape}% name of font to use in the body of the theorem
{}% measure of space to indent
{\bfseries\sffamily}% name of head font
{.}% punctuation between head and body
{ }% space after theorem head; " " = normal interword space
{}% Manually specify head
\newtheoremstyle{mydef}% name of the style to be used
{}% measure of space to leave above the theorem. E.g.: 3pt
{}% measure of space to leave below the theorem. E.g.: 3pt
{}% name of font to use in the body of the theorem
{}% measure of space to indent
{\bfseries\sffamily}% name of head font
{.}% punctuation between head and body
{ }% space after theorem head; " " = normal interword space
{}% Manually specify head

\theoremstyle{mythm}
\newtheorem{thm}{Theorem}[section]
\newtheorem{prop}[thm]{Proposition}
\newtheorem{cor}[thm]{Corollary}
\newtheorem{lem}[thm]{Lemma}
\theoremstyle{mydef}
\newtheorem{mydef}[thm]{Definition}
\newtheorem{rem}[thm]{Remark}
\newtheorem{ex}[thm]{Example}
\newtheorem{exer}{Exercise}[subsection]
\newenvironment{myproof}[1][\proofname]{
	\proof[\sffamily\upshape#1]
}{\endproof}

\newcommand{\proofclear}{\hfill \qedsymbol}

% % % % % MISCELLANEOUS STUFF

\clearscrheadfoot
\ihead[]{}
\ohead[]{}
\cfoot[]{\pagemark}
\pagestyle{scrheadings}

\deffootnote[1em]{0em}{1em}{%
	\textsuperscript{\thefootnotemark}%
}
\setfootnoterule{3em}


\newcommand\numberthis{\stepcounter{equation}\tag{\theequation}}


\newenvironment{numberedlist}{\begin{enumerate}[\upshape(i)]}{\end{enumerate}}
\newenvironment{letteredlist}{\begin{enumerate}[\upshape a)]}{\end{enumerate}}

\renewcommand{\thesection}{\arabic{section}}
\renewcommand{\thesubsection}{(\alph{subsection})}
\renewcommand{\thesubsubsection}{(\roman{subsubsection})}
\renewcommand{\autodot}{}

%Inverse diagonal dots:

\makeatletter
\def\Ddots{\mathinner{\mkern1mu\raise\p@
		\vbox{\kern7\p@\hbox{.}}\mkern2mu
		\raise4\p@\hbox{.}\mkern2mu\raise7\p@\hbox{.}\mkern1mu}}
\makeatother


%TikZ

\tikzset{% 
	arrowat/.style={%
		postaction={decorate,decoration={
				markings,
				mark=at position #1 with {\arrow[xshift=2pt]{>}}}}
	}
}

\tikzset{% 
	backarrowat/.style={%
		postaction={decorate,decoration={
				markings,
				mark=at position #1 with {\arrow[xshift=2pt]{<}}}}
	}
}
\title{Invariant Geometric Structures and \\ Chern numbers of  $G_2$ Flag Manifolds}
\author{Dani\"el Thung}
\date{}
\begin{document}
\maketitle

\section{Outline}

First of all, I would like to thank everybody for being here. Today, I'll be talking about some work that I did together with, and under supervision of, professor Kotschick, my supervisor.

In my thesis, I mostly studied two specific manifolds. They fall into a class called \emph{generalized flag manifolds}, so I'm going to start by explaining what a generalized flag manifold is. After that, I have to explain a little bit about the octonions and introduce the exceptional Lie group $G_2$, 

\section{Introduction}

I'll start by giving a general introduction to motivate the topic of my thesis. My thesis is about certain manifolds which fall in the class called \emph{generalized flag manifolds}. To introduce them, let's start with some linear algebra. A flag is an increasing sequence of subspaces of a vector space, $\{0\}=V_0\subset V_1\subset \dots \subset V_k=V$, where $V$ is an $n$-dimensional vector space. The dimension gets larger at every step, and in the special case where every dimension occurs we have a complete flag. Otherwise, we have a partial flag. The dimensions are encoded in the signature of the flag. For any given signature, a \emph{flag manifold} is the space of flags of this type.

The simplest case to consider is $V=\C^n$; the fact that any orthonormal basis may be turned into any other by means of a unitary transformation means that $U(n)$ acts transitively on the space of flags with given signature. This turns a flag manifold of flags in $\C^n$ into a homogeneous space under $U(n)$. 

Let's consider some simple examples. Of course, the complex projective spaces or more generally Grassmannians of $k$-planes are the flag manifolds for the simplest flags, consisting of a single non-trivial subspace of dimension $k$. To express it as a homogeneous space, we have to find the stabilizer of a point. Consider the standard embedding of $\C^k$ into $\C^n$ onto the first $k$ coordinates. The unitary transformations that preserve this subspace correspond to the block-diagonal matrices $U(k)\times U(n-k)$. Thus, we find that 
\begin{equation*}
	\Gr_k(\C^n)\cong \frac{U(n)}{U(k)\times U(n-k)}
\end{equation*}
Similarly, it is not hard to see that the manifold of complete flags is simply $U(n)/T^n$, the quotient of $U(n)$ by one of its maximal tori. In general, a manifold of flags in $\C^n$ is the quotient of $U(n)$ by the \emph{centralizer} of a torus. 

Now, we can generalize this idea to arbitrary compact, connected, semisimple Lie groups and define a \emph{generalized flag manifold} as the quotient of such a Lie group by the centralizer of a torus. In my thesis, I study the geometry of certain generalized flag manifolds. As with all homogeneous spaces, there are certain privileged geometric structures, namely those compatible with the group translations, i.e.~the $G$-invariant ones.

When it comes to $G$-invariant structures, generalized flag manifolds have some remarkable properties. The most important one is that they admit a K\"ahler-Einstein metric, unique up to scaling, with positive scalar curvature. In fact, results by Goto imply that they are projective manifolds, and even rational. 

Besides this integrable complex structure, there are also typically several $G$-invariant \emph{almost} complex structures. Invariant almost complex structure on arbitrary homogeneous spaces were studied in Lie-theoretic terms in a famous series of papers by Borel and Hirzebruch in the late 1950's. They gave a simple method to determine how many invariant almost complex structures there are, and a way to calculate interesting quantities such as Chern and Pontryagin numbers associated with the almost complex structures. In their work, flag manifolds popped up as simple examples of manifolds which admit multiple invariant almost complex structures with \emph{distinct} Chern numbers. This is a somewhat rare and therefore interesting phenomenon.

Now, all these structures have been studied and relatively well-understood for a long time, in terms of Lie algebras, doing what one I have heard described as ``digging roots and lifting weights''. But in my thesis, I take a point of view that is in some sense complementary, and avoid Lie theory as much as possible, in favor of more geometric methods. I study some concrete examples of generalized flag manifolds, namely $G_2$ flag manifolds, and develop a more differential-geometric picture of the invariant structures, avoiding Lie theory completely. In particular, we will give a geometric interpretation of the various invariant almost complex structures and the K\"ahler-Einstein metric.

The main application that we had in mind initially was to use our geometric picture to compute the Chern numbers of all the invariant almost complex structures on the manifolds we consider. However, in developing a geometric understanding of the situation, one runs into a number of surprising and interesting connections to several topics in complex and Riemannian geometry, ranging from quaternionic K\"ahler geometry to rigidity results for K\"ahler structures and the question of existence of complex structures on $S^6$.

\section{\texorpdfstring{$G_2$}{G2} and its flag manifolds}

To introduce the examples I studied, recall that $G_2$, the smallest of the exceptional, simple Lie groups, can be viewed as the automorphism group of the octonion algebra. The non-trivial structure of the octonions is essentially contained in the imaginary part of the product if imaginary octonions, which is called the \emph{cross product}, and has products similar to the familiar cross product in $\R^3$, and $G_2$ can equivalently be defined as the automorphism group of $\Im \O$, equipped with the cross product. As a vector space over $\R$, this is simply $\R^7$ and via this identification we find some interesting actions of $G_2$.

Recall that it has rank two, and therefore its maximal tori are two-dimensional. Since a maximal torus equals its own centralizer, this means that the ``complete'' $G_2$ flag manifold is $G_2/T^2$. Now, there are two more generalized flag manifolds, corresponding to dividing out the centralizers of one of the two circle factors of $T^2$, and $G_2/T^2$ fibers over both of them. The first one is a quotient of $G_2$ by a subgroup isomorphic to $U(2)$; it can be nicely understood as the Grassmannian of oriented 2-planes in $\R^7$, viewed as the imaginary octonions. Since $G_2$ acts transitively on the space of oriented 2-planes, it also acts transitively on the six-sphere of unit imaginary octonions: We get a fibration over $S^6$, realized as a $G_2$-homogeneous space (\emph{not} a generalized flag manifold).

Now, let's consider the last generalized flag manifold, $G_2$ quotiented by the stabilizer of the \emph{other} circle factor, also isomorphic to $U(2)$; this space is denoted by $G_2/U(2)_+$. Note that the two quotients of $G_2$ by subgroups isomorphic to $U(2)$ are \emph{not} diffeomorphic, or even homotopy equivalent. This completes our list of $G_2$ flag manifolds, and although these are the protagonists in my thesis, we will use one more ``auxiliary manifold'', the homogeneous space $G_2/SO(4)$. All these spaces and fibrations can be described in purely ``octonionic vocabulary'', though we will not do so here. In my thesis, I restricted my attention to the two partial flag manifolds, indicated in red. I will call them $Q$ and $Z$, respectively; note that they are both of real dimension ten, hence will be complex manifolds of dimension five.

\subsection{The Twistor Space}

We start with the study of $Z$. The fibration over $G_2/SO(4)$ exhibits it as the so-called \emph{twistor space} over $G_2/SO(4)$, which is a quaternionic K\"ahler manifold. To make sense out these words, let's first introduce quaternionic K\"ahler manifolds. These are oriented manifolds whose dimension is divisible by four, at least eight, and whose holonomy group is contained in the subgroup $Sp(n)\cdot Sp(1)$ of $SO(4n)$. One important feature of quaternionic K\"ahler manifolds is that they are always Einstein (hence have constant scalar curvature) but not necessarily Ricci-flat: Among the classes of manifolds with reduced holonomy, K\"ahler and quaternionic K\"ahler manifolds are the only ones with this property.

A quaternionic K\"ahler manifold is quaternionic in the sense that there are \emph{locally} (but in general not globally) defined almost complex structures $I,J$ and $IJ=K$ on the tangent bundle compatible with the metric, which span a rank three subbundle of the endormorphism bundle which \emph{is} globally well-defined, even if these almost complex structures are not. The sphere bundle of this rank three vector bundle is what one calls the twistor space of the quaternionic K\"ahler manifold.

The point of its construction is that the twistor space comes equipped with a complex structure, and is even K\"ahler-Einstein in case the base manifold has positive scalar curvature. These results allow one to reduce many geometric question about quaternionic K\"ahler manifolds to questions about complex or K\"ahler geometry of the twistor space.

As mentioned, $Z$ is the twistor space of $G_2/SO(4)$, which is a quaternionic K\"ahler manifold with positive scalar curvature. Thus, the K\"ahler-Einstein metric on $Z$ is actually the canonical twistor space metric. Since we are interested in $Z$ rather than $G_2/SO(4)$, we use the twistor space description to study it: Since it is an $S^2$ bundle, we have a Gysin sequence, which allows us to compute the cohomology ring and Pontryagin classes of $Z$ from those of $G_2/SO(4)$. Additively, the cohomology groups coincide with $\CP^5$, but not multiplicatively.

Perhaps surprisingly, this topological information turns out to be rather strong: This is because we were able to prove the following theorem: Assume we have a manifold homeomorphic to the twistor space $Z$ which is furthermore K\"ahler, then it is even biholomorphic to the twistor space, equipped with its natural complex structure. This rigidity result for the K\"ahler structure on this manifold is an analog of classic results due to Hirzebruch, Kodaira, Yau and Brieskorn. 

Hirzebruch and Kodaira proved that any manifold homeomorphic to $\CP^n$ is biholomorphic to it (under a technical assumption removed by Yau's solution of the Calabi conjecture), and Brieskorn proved the analogous statement for quadrics of dimension at least three.

The main step in the proof is to determine the first Chern class. To do so, we use the simple structure of the cohomology. First of all, since the cohomology in even degrees is infinite cyclic, we only need to find which multiple of the positive generator it is. Here, the notion of positivity is defined by the K\"ahler metric. Secondly, since $X$ is K\"ahler by assumption, the Hodge numbers are those of $\CP^5$. The Chern number $c_1c_4$ is determined by the Hodge numbers (this was discovered by Libgober and Wood), and therefore coincides with that of $\CP^5$; this means that it is $90$.

Now, we see that $d$ must divide $90$ (but it can also be negative). The divisors can be checked one by one, and using the topology, the Pontryagin classes and further constraints coming from the Hirzebruch-Riemann-Roch theorem one arrives at a contradiction in every case except $d=3$.

Now that the first Chern class is determined, we have found that $X$ must be Fano, with Fano index three. The highest possible index for a Fano $n$-manifold is $n+1$, or six in our case. Thus, $X$ has Fano \emph{coindex} three. Fano manifolds with coindex three were classified by Mukai, under a technical assumption that was later removed by Mella. The classification of Mukai shows that a manifold with the cohomology ring of $Z$ and Fano coindex three must in fact be biholomorphic to $Z$, finishing our proof.

%This result can be understood as a generalization of the results for $\CP^n$ and the quadrics. As I mentioned, the highest possible Fano index for an $n$-manifold is $n+1$. Kobayashi and Ochiai proved that if a manifold saturates this bound, then it is biholomorphic to $\CP^n$. If the Fano index is $n$, then it is the quadric. Hence, high Fano index seems to imply rigidity and our result extends the classical results to lower divisibility in dimension five.

Knowing the first Chern class means knowing the entire Chern class, via the number $c_1c_4$ and the Pontryagin classes. Thus, it is not hard to obtain the Chern numbers for the twistor space $Z$. However, we are also interested in \emph{other} invariant almost complex structures. The classic papers by Borel and Hirzebruch include a simple recipe for determining the number of invariant almost complex structures (up to conjugation) of a homogeneous space, by counting irreducible summands of the so-called isotropy representation of the isotropy subgroup. On $Z$, there are two invariant almost complex structures up to conjugation. 

The second one is obtained from the first one by what we call ``flipping the fiber''. The $\CP^1$ fibers of the twistor fibration are complex submanifolds, and they give rise to a complex subbundle of the tangent bundle consisting of tangent vectors along the fibers. Replacing the almost complex structure on this subbundle by its conjugate produces a new, still invariant, almost complex structure. It can be distinguished from the old one by the Chern numbers, which can easily be determined from the old ones, using the above description. We denote it by $N$ because it admits a nearly K\"ahler metric. 

One important thing to mention here is that the Chern numbers of the invariant almost complex structures on $Z$ were computed in a paper published earlier this year by Grama, Negreiros and Oliveira. However, their calculation follows the Lie-theoretic recipe proposed by Borel and Hirzebruch and is therefore very different from ours. Furthermore, there are some errors in the numbers they get out. In any case, the Chern numbers are listed in the table. 

\subsection{The Quadric}

Let us start with some basic information about the space of oriented two-planes in $\Im\O$. We denote it by the letter $Q$. Why? Because this Grassmannian is diffeomorphic to the (so-called zero) quadric in $\CP^6$. This immediately gives us an interpretation of the invariant K\"ahler-Einstein structure on $Q$, which is simply the metric obtained by restriction of the Fubini-Study metric; this is even $SO(7)$-invariant (and $SO(7)$ contains $G_2$). Of course, there is the rigidity result for the K\"ahler structure on $Q$ by Brieskorn, mentioned before.

On the other hand, the octonionic description allows one to see that $Q$ is diffeomorphic to the projectivized tangent bundle of $S^6$: The projection to $S^6$ is given by sending an oriented basis $\{x,y\}$ of 2-plane to the cross product $\{x\times y\}$. First of all, this yields a second invariant almost complex structure on $Q$, making use of the $G_2$-invariant almost complex structure that $S^6$ inherits from the octonions. A third invariant almost complex structure is obtained by recognizing that the projectivized tangent bundle is diffeomorphic to the projectivized \emph{cotangent} bundle as well. 

Now, if the $G_2$ invariant almost complex structure on $S^6$ were integrable, then we would obtain two non-standard (non-K\"ahler, by Brieskorn's result!) complex structures on the quadric. This is of course known not to be the case, but in fact we obtain the same result if there is \emph{any} integrable almost complex structure on $S^6$. This is because of the fact that all almost complex structures on $S^6$ are homotopic as almost complex structures. A homotopy will produce a diffeomorphism between projectivized tangent bundles, hence even with a \emph{different} almost complex structure we will obtain complex manifolds diffeomorphic to the quadric. Thus, one might try to rule out the existence of a complex structure on $S^6$ by ruling out non-standard complex structures on $Q$. This is an analog of a well-known argument that relates complex structures on $S^6$ to complex structures on $\CP^3$, by blowing up a point.

In any case, we have now found three invariant almost complex structures. The work of Borel and Hirzebruch implies that there are four in total, so we are still looking for a fourth. The procedure of flipping the fiber on the twistor space inspires one to try a similar thing for $Q$, using the fibrations over $S^6$ and $G_2/SO(4)$. As before, the tangent vectors to the fibers yield complex subbundle, which we can flip. Flipping the fiber over $S^6$ yields a new invariant almost complex structure from the standard one, which can be distinguished by its Chern numbers. This gives us an interpretation of all the invariant almost complex structures. 

The Chern numbers of the standard structures are easy to calculate via the adjunction formula, and those of the fourth invariant almost complex structure are found through our description of flipping the fiber. For the remaining invariant almost complex structure one pulls back the tangent or cotangent bundle of $S^6$ to the total space of their projectivizations and uses a relative Euler sequence to determine the Chern class of the vertical subbundle: One obtains the total Chern class and hence the Chern numbers. Alternatively, it turns out that actually one can obtain \emph{all four} invariant almost complex structures from the standard one, by flipping the fibers over $S^6$ and $G_2/SO(4)$.




\end{document}