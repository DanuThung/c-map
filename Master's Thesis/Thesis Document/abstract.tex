\documentclass[parskip=half]{scrbook}
\usepackage{scrlayer-scrpage}
\begin{document}
\chapter*{Abstract}

This thesis concerns a special class of homogeneous spaces, called \emph{generalized flag manifolds}. It has long been known that generalized flag manifolds admit interesting invariant geometric structures, particularly from the point of view of complex geometry. These are traditionally investigated by means of Lie theory, leveraging the homogeneity to reduce geometric questions to algebraic ones. 

In this work, however, we take a complementary approach. We study certain examples of generalized flag manifolds, namely those which are homogeneous under the exceptional Lie group $G_2$, from a \emph{geometric} point of view. Avoiding the use of Lie theory, we study their invariant almost complex structures by differential-geometric methods. In the process, we uncover surprising connections to other topics, ranging from the existence of complex structures on the six-sphere to rigidity theorems for K\"ahler manifolds. As a main application, we use our geometric picture to compute the Chern numbers of all invariant almost complex structures without appealing to Lie theory.

\end{document} 