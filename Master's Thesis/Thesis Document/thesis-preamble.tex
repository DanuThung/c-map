 % % % % % PACKAGES

%General Packages

\usepackage[automark]{scrlayer-scrpage}																	
\usepackage{amsfonts}												%Mathematics fonts
\usepackage{mathtools}												%General mathematics symbols
\usepackage{amsmath}
\usepackage{stmaryrd}												%Extra math symbols
\usepackage{amssymb}												%More symbols
\usepackage{extarrows}												%Extendible arrows
\usepackage{dsfont} 												%Identity matrix symbol
\usepackage{mathrsfs}												%To get mathscr
\usepackage{relsize}												%Scaling symbols with reference to pre-existing symbols
\usepackage{accents}												%Accents on math symbols
\usepackage[english]{babel}
\usepackage{csquotes}
\usepackage{subcaption} 											%Subfigures etc
\usepackage{cancel} 												%Striking through things
\usepackage{setspace}												%line spacing
\usepackage{footmisc}												%Some footnote margin thing
\usepackage{enumerate} 												%Numbered lists
\usepackage{etoolbox}												%Enables spacing adjustments for environments
\usepackage{booktabs}												%better tables
\usepackage{pdfpages}												%to add the title page

%Pictures & TikZ Packages

\usepackage{graphicx}												%Pictures	
\usepackage{epstopdf}												%Converts .eps to .pdf files
\usepackage{tikz}													%TikZ Drawings
\usetikzlibrary{3d,patterns,arrows,bending,arrows.meta,				%TikZ Libraries
	shapes.geometric,knots,intersections,
	decorations.markings,decorations.pathmorphing,
	decorations.pathreplacing}										
\usepackage{tikz-cd}												%Commutative Diagrams


%Mathematics Packages

\usepackage{amsthm}													%Theorems etc

%Referencing

\usepackage{chngcntr}												%continuous numbering of figures etc.
\usepackage[colorlinks=true,citecolor=blue,hidelinks]{hyperref}								%hyperlinks
\usepackage[noabbrev]{cleveref}										%better croff-refs
\usepackage[backend=biber, 
doi=false,
eprint=false,
url=false,
isbn=false,
style=numeric,
citestyle=numeric-comp,
sorting=nyt,
firstinits=true]{biblatex}											%bibliography

%Biblatex stuff

\addbibresource{thesis-references.bib}
\appto{\bibsetup}{\raggedright}
\renewbibmacro{in:}{%
	\ifentrytype{article}{}{\printtext{\bibstring{in}\intitlepunct}}}
\crefname{prop}{proposition}{propositions}
\crefname{lem}{lemma}{lemmata}						

\AtEveryBibitem{\clearfield{pagetotal}\clearfield{series}\clearfield{number}}
\AtEveryBibitem{%
	\ifentrytype{article}
	{}
	{\clearfield{volume}}}

%WORK NEEDED: Check bibliography settings and use hidelinks for hyperref before printing!
% % % % % CUSTOM COMMANDS

%Derivatives/Differentials

\let\underdot=\d
\newcommand{\od}[2]{\frac{\mathrm{d} #1}{\mathrm{d} #2}}
\newcommand{\odd}[2]{\frac{\mathrm{d}^2 #1}{\mathrm{d} #2^2}}
\newcommand{\p}{\partial}
\newcommand{\pd}[2]{\frac{\partial #1}{\partial #2}}
\newcommand{\pdd}[2]{\frac{\partial^2 #1}{\partial #2^2}}
\newcommand{\fd}[2]{\frac{\delta #1}{\delta #2}}
\renewcommand{\d}{\mathrm{d}}
\newcommand{\dif}{D}


%Common Sets/Spaces

\newcommand{\RP}{\mathbb{R}\mathrm{P}}
\newcommand{\CP}{\mathbb{C}\mathrm{P}}
\newcommand{\HP}{\mathbb{H}\mathrm{P}}
\renewcommand{\P}{\mathbb{P}}
\newcommand{\N}{\mathbb{N}}
\newcommand{\Z}{\mathbb{Z}}
\newcommand{\Q}{\mathbb{Q}}
\newcommand{\R}{\mathbb{R}}
\newcommand{\C}{\mathbb{C}}
\renewcommand{\H}{\mathbb{H}}
\renewcommand{\O}{\mathbb{O}}

%Math-operators

\renewcommand{\Im}{\operatorname{Im}}
\renewcommand{\Re}{\operatorname{Re}}

\DeclareMathOperator{\Graph}{graph}
\DeclareMathOperator{\Gr}{Gr}

\DeclareMathOperator{\im}{im}
\DeclareMathOperator{\rank}{rank}
\DeclareMathOperator{\ord}{ord}
\DeclareMathOperator{\tr}{tr}
\DeclareMathOperator{\incl}{incl}
\DeclareMathOperator{\pr}{proj}
\DeclareMathOperator{\diag}{diag}
\DeclareMathOperator{\Span}{span}

\DeclareMathOperator{\Hom}{Hom}
\DeclareMathOperator{\End}{End}
\DeclareMathOperator{\Iso}{Iso}
\DeclareMathOperator{\Aut}{Aut}
\DeclareMathOperator{\coker}{coker}
\DeclareMathOperator{\Ric}{Ric}
\DeclareMathOperator{\Hol}{Hol}
\DeclareMathOperator{\Pic}{Pic}
\DeclareMathOperator{\Spin}{Spin}
\DeclareMathOperator{\ind}{ind}
\DeclareMathOperator{\cl}{cl}
\DeclareMathOperator{\Bs}{Bs}
\DeclareMathOperator{\id}{id}
\DeclareMathOperator{\ch}{ch}
\DeclareMathOperator{\td}{td}
\newcommand{\Unit}{\mathds{1}}



\newcommand{\trans}{\mathrel{\text{\tpitchfork}}}
\makeatletter
\newcommand{\tpitchfork}{%
	\vbox{
		\baselineskip\z@skip
		\lineskip-.52ex
		\lineskiplimit\maxdimen
		\m@th
		\ialign{##\crcr\hidewidth\smash{$-$}\hidewidth\crcr$\pitchfork$\crcr}
	}%
}
\makeatother

\DeclareMathOperator{\Ad}{Ad}
\DeclareMathOperator{\ad}{ad}

\DeclareMathOperator{\supp}{supp}
\DeclareMathOperator{\interior}{int}
\DeclareMathOperator{\vol}{vol}
\DeclareMathOperator{\area}{area}
\DeclareMathOperator{\Area}{Area}

\DeclareMathOperator{\sgn}{sgn}

\DeclareMathOperator{\Tor}{Tor}
\DeclareMathOperator{\Ext}{Ext}

%Other

\newcommand{\action}{\curvearrowright}
\newcommand{\rightaction}{\curvearrowleft}

\newcommand{\ubar}[1]{\underaccent{\bar}{#1}}
\def\mathunderline#1#2{\color{#1}\underline{{\color{black}#2}}\color{black}}

\newcommand{\abs}[1]{\left\lvert #1 \right\rvert}
\newcommand{\norm}[1]{\left\lVert #1 \right\rVert}
\newcommand{\expvalue}[1]{\left\langle #1 \right\rangle}


\newcommand{\mf}[1]{\mathfrak{#1}}
\newcommand{\mc}[1]{\mathcal{#1}}
\newcommand{\ms}[1]{\mathscr{#1}}

\newcommand{\bdy}{\partial}
\newcommand{\pt}{\mathrm{pt}}
\DeclareMathOperator{\Bl}{Bl}


%Theorem Styles

\newtheoremstyle{mythm}% name of the style to be used
{}% measure of space to leave above the theorem. E.g.: 3pt
{}% measure of space to leave below the theorem. E.g.: 3pt
{\slshape}% name of font to use in the body of the theorem
{}% measure of space to indent
{\bfseries\sffamily}% name of head font
{.}% punctuation between head and body
{ }% space after theorem head; " " = normal interword space
{}% Manually specify head
\newtheoremstyle{mydef}% name of the style to be used
{}% measure of space to leave above the theorem. E.g.: 3pt
{}% measure of space to leave below the theorem. E.g.: 3pt
{}% name of font to use in the body of the theorem
{}% measure of space to indent
{\bfseries\sffamily}% name of head font
{.}% punctuation between head and body
{ }% space after theorem head; " " = normal interword space
{}% Manually specify head

\theoremstyle{mythm}
\newtheorem{thm}{Theorem}[chapter]
\newtheorem{prop}[thm]{Proposition}
\newtheorem{cor}[thm]{Corollary}
\newtheorem{lem}[thm]{Lemma}
\newtheorem*{con*}{Conjecture}
\theoremstyle{mydef}
\newtheorem{mydef}[thm]{Definition}
\newtheorem{rem}[thm]{Remark}
\newtheorem{ex}[thm]{Example}
\newenvironment{myproof}[1][\proofname]{
	\proof[\sffamily\upshape#1]
}{\endproof}

\newcommand{\proofclear}{\hfill \qedsymbol}


% % % % % MISCELLANEOUS STUFF

\AtBeginEnvironment{myproof}{\vspace{-1\baselineskip}}	%corrects spacing of environment

\makeatletter
\g@addto@macro\appendix{%
	\renewcommand*{\chapterformat}{%
		{\chapapp\nobreakspace\thechapter\autodot\enskip}%
	}
	\renewcommand*{\chaptermarkformat}{%
		{\chapapp\nobreakspace\thechapter\autodot\enskip}%
	}
	\let\oldaddcontentsline\addcontentsline
	\newcommand\hackedaddcontentsline[3]{\oldaddcontentsline{#1}{#2}{\chapapp\nobreakspace#3}}
	\let\oldchapter\chapter
	\renewcommand*{\chapter}[1]{%
		\let\addcontentsline\hackedaddcontentsline%
		\oldchapter{#1}%
		\let\addcontentsline\oldaddcontentsline%
	}
}
\makeatother					%appendix prefix in the TOC 


\pretocmd\printbibliography{%
	 \let\chapter\oldchapter%
	 \renewcommand{\bibname}{References}%
}{}{}							%the appendix thing messes up the name of things that come after the appendix; this fixes the title of the references

\deffootnote[1em]{0em}{1em}{%
	\textsuperscript{\thefootnotemark}%
}
\setfootnoterule{3em}

\counterwithin{equation}{chapter}
\counterwithout{figure}{chapter}
\counterwithout{table}{chapter}
\counterwithout{footnote}{chapter}
\renewcommand{\thetable}{\arabic{table}}
\renewcommand*{\figureformat}{%
	\figurename~\thefigure%
	%  \autodot% DELETED
}
\renewcommand*{\tableformat}{%
	\tablename~\thetable%
	%  \autodot% DELETED
}


\newcommand\numberthis{\stepcounter{equation}\tag{\theequation}}


\newenvironment{numberedlist}{\begin{enumerate}[\upshape(i)]}{\end{enumerate}}
\newenvironment{letteredlist}{\begin{enumerate}[\upshape a)]}{\end{enumerate}}

%\renewcommand{\thesection}{\arabic{section}}
%\renewcommand{\thesubsection}{(\alph{subsection})}
%\renewcommand{\thesubsubsection}{(\roman{subsubsection})}

%Inverse diagonal dots:

\makeatletter
\def\Ddots{\mathinner{\mkern1mu\raise\p@
		\vbox{\kern7\p@\hbox{.}}\mkern2mu
		\raise4\p@\hbox{.}\mkern2mu\raise7\p@\hbox{.}\mkern1mu}}
\makeatother