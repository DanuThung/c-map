\chapter{Quaternionic K\"ahler manifolds and twistor spaces}
\label{chap:twistor}

In this chapter, we discuss quaternionic K\"ahler manifolds and their \emph{twistor spaces}. There are a number of distinct approaches to this topic, corresponding to different motivations for studying quaternionic K\"ahler manifolds. We will roughly follow the discussion of Salamon in~\cite{Sal1982}, though some of the proofs are taken from~\cite{Bes2008} (again, we claim no originality). For alternative points of view, see for instance~\cite{BO1984,Sal1984,Sal1986} and the review~\cite{Sal1999}. Quaternionic K\"ahler manifolds have also attracted attention from physicists in the context of supergravity theories (see for instance the original reference~\cite{BW1983} or the recent book~\cite{Cor2010}). However, we will not investigate the connections to physics in this work.

\section{What is a quaternionic K\"ahler manifold?}

The classification of holonomy groups provides one of the motivations for studying quaternionic K\"ahler manifolds. To explain this, we will briefly discuss (without proofs) a few important results regarding holonomy groups; details can be found in~\cite{KN1963,Bes2008,Joy2000a}. Recall that the holonomy group of a connected Riemannian manifold $(M,g)$, denoted by $\Hol(g)$, is the compact (this is a theorem!) Lie subgroup of $O(T_pM)$ generated by parallel transport along all loops based at $p\in M$. Since changing the base point yields a conjugate subgroup, the choice of base point is immaterial (we will henceforth speak implicitly about conjugacy classes of subgroups). Restricting to null-homotopic loops yields the \emph{restricted} holonomy group $\Hol^0(g)$.

According to whether the holonomy representation on the tangent spaces is reducible or not, we call $(M,g)$ reducible or irreducible. In the former case, one obtains two complementary subbundles of $TM$. They are integrable (in the sense of Frobenius' theorem) and one may prove that $M$ is then locally isometric to a Riemannian product. If there is a subbundle on which $\Hol(g)$ acts trivially, then the corresponding integral manifold is locally isometric to a Euclidean space. From now on, we will assume that $M$ is complete. Then, after possibly passing to the universal covering space $(\tilde M,\tilde g)$, a famous theorem due to De Rham yields a unique way to decompose it into irreducible manifolds:

\begin{thm}[De Rham, {\cite[Sec. IV.6]{KN1963}}]
	A connected, simply connected, complete, reducible Riemannian manifold is isometric to a Riemannian product of connected, simply connected and complete Riemannian manifolds.
\end{thm}

Proceeding inductively, one finds:

\begin{thm}[De Rham Decomposition Theorem]
	A connected, simply connected, complete Riemannian manifold is isometric to a Riemannian product of connected, simply connected, complete and irreducible Riemannian manifolds with a Euclidean space (possibly of dimension zero). This decomposition is unique up to order.
\end{thm}

When studying the \emph{restricted} holonomy group, one can work under the assumption of simple connectedness without loss of generality. Indeed, null-homotopic loops based at $p$ are precisely the images of loops in the universal covering $(\tilde M,\tilde g)$ (based at a fixed lift of $p$), hence $\Hol^0(g)=\Hol^0(\tilde g)=\Hol(\tilde g)$.

In the early twentieth century, \'Elie Cartan succeeded in classifying the irreducible, simply connected symmetric spaces and computing their holonomy groups. Recall from \cref{chap:homogeneous} that symmetric spaces are Riemannian homogeneous spaces, i.e.~of the form $M\cong G/H$ where $G$ is a Lie group and $H$ a compact subgroup. It follows from Cartan's classification that, although $M$ may be homogeneous under different groups, there is only one combination $(G,H)$ (called a \emph{symmetric pair}) which turns $M$ into a symmetric space. It turns out that the holonomy group is given by the isotropy group $H$ itself. By our above remark, this determines the restricted holonomy groups of locally symmetric spaces. 

This begs the question: What about the holonomy of Riemannian manifolds which are \emph{not} locally symmetric? That question was answered by Berger in 1955:

\begin{thm}[Berger, {\cite{Ber1955}}]
	Let $(M,g)$ be a complete, connected, simply connected and irreducible Riemannian manifold which is not symmetric. Then its holonomy group occurs in the following list:
	\begin{center}
		\begin{tabular}{llll} \toprule
			Dimension				& Holonomy Group		\\ \midrule
			$n$ 					& $SO(n)$				\\
			$n=2m,\ m\geq 2$ 		& $U(m)$				\\
			$n=2m,\ m\geq 2$		& $SU(m)$ 				\\
			$n=4m,\ m\geq 2$		& $Sp(m)\cdot Sp(1)$	\\
			$n=4m,\ m\geq 2$		& $Sp(m)$				\\
			$n=8$					& $\Spin(7)$			\\
			$n=7$					& $G_2$					\\ \bottomrule
		\end{tabular}
	\end{center}
	where $Sp(m)\cdot Sp(1)=(Sp(m)\times Sp(1))/\Z_2$, where $(\pm \id,\pm 1)$ are identified.
\end{thm}

\begin{rem}\leavevmode
	\begin{numberedlist}
		\item The original list of Berger included the case $n=16$, with holonomy group $\Spin(9)$. Alekseevski\u{\i}~\cite{Ale1967} showed that such a manifold is always locally symmetric. A few years later, Brown and Gray~\cite{BG1972} gave an independent proof of this fact.
		\item $SO(n)$, the largest holonomy group, is known as the ``generic'' case; any oriented Riemannian manifold has holonomy group contained in $SO(n)$. Among the reduced holonomy groups, the most famous case is the group $U(m)$: This corresponds to K\"ahler manifolds. The inclusions of $SU(2m)$ and $Sp(m)$ into $U(2m)$ show that the corresponding manifolds, called Calabi-Yau and hyper-K\"ahler, respectively, are special types of K\"ahler manifolds.
		\item Besides giving the holonomy group, one should really specify how it acts. One may then observe that the above actions are always transitive on the unit spheres in the tangent spaces. This fact has been the focus of a lot of attention; direct proofs have been provided by Simons~\cite{Sim1962} and more recently Olmos~\cite{Olm2005}, whose proof is geometric (rather than algebraic) in nature.
	\end{numberedlist} 
\end{rem}

It is natural to expect, for instance by taking inspiration from the K\"ahler case, that each possible holonomy group corresponds to a distinct ``flavor'' of geometry. This provides a strong motivation for studying the geometric structures that arise for each holonomy group. Furthermore, Berger did not address the natural question whether there actually exist manifolds whose (global!) holonomy groups coincide with the groups that appear in his list. It took thirty more years to prove that this is indeed the case and there is considerable interest in constructing new examples, which are often relatively scarce. Thus, one is naturally led to study the following class of manifolds:

\begin{mydef}
	A quaternionic K\"ahler manifold $(M,g)$ is an (oriented) manifold of dimension $4n$, $n\geq 2$, whose holonomy group is contained in $Sp(n)\cdot Sp(1)$.
\end{mydef}

The case $n=1$ is excluded because $Sp(1)\cdot Sp(1)\cong SO(4)$. To see this, observe that $Sp(1)\cong SU(2)$. Indeed, under the standard identification of $\H$ with $\C^2$, right-multiplication with a unit quaternion $q=a+bi+cj+dk$ (where $a^2+b^2+c^2+d^2=1$) corresponds to left-multiplication by 
$\begin{psmallmatrix}
	a+bi & -c+di \\ c+di & a-bi
\end{psmallmatrix}$, which is precisely the general form of an element of $SU(2)$. Now, our assertion follows from the well-known fact that $(SU(2)\times SU(2))/\Z_2\cong SO(4)$.

More generally, $Sp(n)\cdot Sp(1)$ is realized as a subgroup of $SO(4n)$ as follows: Consider $\R^{4n}$ as a right $\H$-module by identifying it with $\H^n$, on which $\H$ acts from the right. The action of the unit quaternions then induces an embedding of $Sp(1)$ into $SO(4n)$. We identify $GL(n,\H)$ with the subgroup of $GL(2n,\C)$ formed by block-matrices of the form 
$\begin{psmallmatrix}
	A & -\bar B \\ B & \bar A
\end{psmallmatrix}$. Since $Sp(n)$ is defined as those elements of $GL(n,\H)$ that preserve the standard (symplectic) inner product on $\H^n$, which corresponds to that of $\C^{2n}$ under the natural identification, we see that $Sp(n)\subset U(2n)$. 

In fact, $Sp(n)\subset SU(2n)$; this is easiest seen by noting that elements of its Lie algebra are automatically traceless and skew-Hermitian. Thus, after embedding $SU(2n)$ into $SO(4n)$, we also obtain an embedding of $Sp(n)$ into $SO(4n)$. The image is the subgroup of $SO(4n)$ that commutes with the image of the embedding of $Sp(1)$, since it is precisely these elements that come from $\H$-linear maps. Abusing notation, we identify $Sp(1)$ and $Sp(n)$ with their images in $SO(4n)$ and form the product $Sp(n)\cdot Sp(1)$, which is easily seen to be isomorphic to $(Sp(n)\times Sp(1))/\Z_2$.

\begin{rem}\leavevmode
	\begin{numberedlist}
		\item The name ``quaternionic K\"ahler manifold'' suggests that these manifolds are the quaternionic analogs of K\"ahler manifolds. This may initially sound surprising, since the naive quaternionic analog of $U(n)$-holonomy is $Sp(n)$-holonomy. However, the inclusion $Sp(n)\subset SU(2n)$ shows that such (hyper-K\"ahler) manifolds  are even Calabi-Yau, and thus correspond to a very special subclass of K\"ahler manifolds.
		
		The analogy with $Sp(n)\cdot Sp(1)$, which includes hyper-K\"ahler manifolds as a special case, is better: Indeed, we will soon encounter a four-form $\Omega$ which plays a role similar to that of the K\"ahler form. Moreover, we will shortly see that there is a natural way to associate a complex manifold to a quaternionic K\"ahler manifold, which is in some cases even K\"ahler, so that K\"ahler geometry plays a role in the theory as well.
		\item The reader should be warned that, despite this analogy and the name, quaternionic K\"ahler manifolds are not complex or even \emph{almost} complex in general, let alone K\"ahler. Consider the quaternionic projective spaces $\HP^n\cong (Sp(n+1)/\Z_2)/Sp(n)\cdot Sp(1)$, which are symmetric spaces. This means that they have holonomy $Sp(n)\cdot Sp(1)$ and hence are quaternionic K\"ahler. However, it can be shown using characteristic classes that $\HP^n$ cannot be almost complex for any $n\in\N$~\cite{Hir1960,Mas1962}.
	\end{numberedlist}
\end{rem}

To understand the connection between quaternionic geometry and holonomy $Sp(n)\cdot Sp(1)$ more clearly, we first have to do some linear algebra (following Kraines~\cite{Kra1966}). Consider $\H^n$ as a right-module over $\H$ and recall the symplectic inner product which sends $P, Q\in \H^n$ to $\langle P,Q\rangle \coloneqq \sum_{a=1}^n p_a \bar q_a$. 

Now define a new inner product $(P,Q)\coloneqq \frac{1}{2}(\langle P,Q\rangle +\langle Q,P\rangle )$ (which will correspond to a Riemannian metric) and the three two-forms 
\begin{equation*}
	\Omega_I(P,Q)\coloneqq (P i,Q) \qquad \qquad 
	\Omega_J(P,Q)\coloneqq (P j,Q) \qquad \qquad 
	\Omega_K(P,Q)\coloneqq (P k,Q)
\end{equation*}
It is clear that $i^*\Omega_I=\Omega_I=\Omega_I$, $j^*\Omega_J=\Omega_J$ and $k^*\Omega_K=\Omega_K$ (remember that $i,j,k$ act by right-multiplication). More generally, a (tedious) computation shows:

\begin{lem}\label{lem:QKcovderivs}
	Let $\lambda=a+bi+cj+dk\in Sp(1)$. Then
	\begin{align*}
		\lambda^*\Omega_I&=(a^2+b^2-c^2-d^2)\Omega_I+2(ad+bc)\Omega_J
		+2(bd-ac)\Omega_K\\
		\lambda^*\Omega_J&=2(bc-ad)\Omega_I+(a^2-b^2+c^2-d^2)\Omega_J
		+2(ab+cd)\Omega_K\\
		\lambda^*\Omega_K&=2(ac+bd)\Omega_I+2(cd-ab)\Omega_J
		+(a^2-b^2-c^2+d^2)\Omega_K
	\end{align*}
\end{lem}

Now, we define the four-form $\Omega=\Omega_I\wedge \Omega_I+\Omega_J\wedge\Omega_J+\Omega_K\wedge\Omega_K$. The above lemma guarantees that $\lambda^*\Omega=\Omega$. $Sp(n)$-invariance of the inner product $\langle-,-\rangle$ shows that $\Omega$ is in fact $Sp(n)\cdot Sp(1)$-invariant. 

Now, starting from a quaternionic K\"ahler manifold, we may identify its tangent space over a given point with $\H^n$. This identification will in general not be preserved when passing to a different chart with overlapping domain, but since we may choose the transition functions to take values in $Sp(n)\cdot Sp(1)$ the form $\Omega$ (initially defined with respect to a specific identification) is well-defined. Moreover, $Sp(n)\cdot Sp(1)$-invariance shows that $\Omega$ is invariant under parallel transport along any loop, so we obtain:

\begin{prop}
	A quaternionic K\"ahler manifold admits a non-vanishing, parallel four-form. \proofclear
\end{prop}

In fact, with a little more effort one may prove that $\Omega$ is non-degenerate and use this in the study of the cohomology of $M$; one immediate corollary is that the Betti numbers $b_{4n}(M)$ are nonzero since any parallel form (such as $\Omega^n$) is closed and co-closed. Other applications include an analog of the Lefschetz decomposition on K\"ahler manifolds (see~\cite{Kra1966}). 

In the differential-geometric setting, \cref{lem:QKcovderivs} means that a quaternionic K\"ahler manifold locally admits three two-forms $\Omega_{I,J,K}$ whose covariant derivatives are linear combinations of $\Omega_{I,J,K}$ at each point. Since the Levi-Civit\`a connection is compatible with the metric, $(\nabla_X\Omega_I)(Y,Z)=g((\nabla_X R_I)Y,Z)$, where $R_I$ is the endomorphism (locally defined!) that corresponds to right-multiplication by $i\in \H$. 

Thus, we may equivalently say that the covariant derivatives of $R_{I,J,K}$ can be expressed in terms of the $R_{I,J,K}$ themselves, i.e.~the three-dimensional bundle of endomorphisms they span is preserved by covariant differentiation. This shows that quaternionic K\"ahler manifolds can be equipped with a covering by special charts (in the following, we identify $R_{I,J,K}$ with $I$, $J$, $K$):

\begin{prop}\label{prop:QKcharts}
	A quaternionic K\"ahler manifold $(M,g)$ admits an open covering $\{U_i\}$ with the following properties:
	\begin{numberedlist}
		\item On each $U_i$, there exist two almost complex structures $I$ and $J$ such that $IJ=-JI$.
		\item On $U_i$, $g$ is Hermitian with respect to $I$ and $J$ defined with respect to $U_i$.
		\item The covariant derivatives of $I$ and $J$ are (pointwise) linear combinations of $I$, $J$ and $K=IJ$.
		\item For every $x\in U_i\cap U_j$, the subspace of $\End(T_xM)$ spanned by $I$, $J$ and $K$ is independent of whether $I$ and $J$ are defined with respect to $U_i$ or $U_j$. In other words, they define a three-dimensional subbundle of $\End TM$.
	\end{numberedlist}
\end{prop}
\begin{myproof}
	The first three points follow from out discussion above. The final point is a consequence of the fact that $Sp(n)\cdot Sp(1)\subset SO(4n)$ preserves (under conjugation) the vector subspace of endomorphisms of $\R^{4n}=\H^n$ generated by multiplication by $i,j,k$. To see this, let $L\in Sp(n)\cdot Sp(1)$ be given by $v\mapsto (Av)q$, where $A\in Sp(n)$, $q\in Sp(1)$. Now let $p\in \Im \H$ (acting from the right); then $L^{-1}\circ p\circ L(v)=vq^{-1}pq$. Since $q^{-1}pq\in \Im\H$, $Sp(n)\cdot Sp(1)$ indeed preserves the subspace of endomorphisms induced by $\Im\H$.
\end{myproof}

\begin{rem}\leavevmode
	\begin{numberedlist}
		\item In fact, it is possible to prove (though we will not do so) that a manifold that admits such a covering must be quaternionic K\"ahler, thus providing an alternative characterization of quaternionic K\"ahler manifolds.
		\item Using the quaternionic relations between $I$, $J$ and $K$ we find that the third condition is equivalent to the existence of locally defined one-forms $\alpha$, $\beta$ and $\gamma$ such that:
		\begin{align*}
			\nabla_X I&= \alpha(X)J - \beta(X)K\\
			\nabla_X J&= -\alpha(X)I + \gamma(X)K\\
			\nabla_X K&= \beta(X)I - \gamma(X)J
		\end{align*}
		To see that $\nabla_X I$ has no term proportional to $I$ (and the same holds for $J$, $K$), one must consider $\nabla_X (I^2)=0$ and work out the left-hand side using the Leibniz rule. The relations between the coefficient one-forms are determined in analogous fashion, starting from $\nabla_X(JK)=\nabla_X I$ and its cyclic permutations.
	\end{numberedlist}
\end{rem}

Reduced holonomy typically implies heavy restrictions on curvature. In fact, of the holonomy groups that appear in Berger's list, $U(m)$ and $Sp(m)\cdot Sp(1)$ are the only ones that allow for manifolds that are \emph{not} necessarily Ricci-flat. However, for quaternionic-K\"ahler manifold we do have the following:

\begin{thm}[Berger,~\cite{Ber1966}\footnote{We were unable to consult this source in person; the theorem is credited to this paper by several independent sources.}]
	Every quaternionic K\"ahler manifold is Einstein.
\end{thm}
\begin{myproof}
	We reproduce the elementary proof due to Ishihara~\cite{Ish1974} (see also~\cite{Bes2008}). Since the Einstein condition is local, we may work inside one of the charts from \cref{prop:QKcharts}. Using $R(X,Y)=\nabla_X\nabla_Y-\nabla_Y\nabla_X-\nabla_{[X,Y]}$ and the above expression for $\nabla_X I$, we find:
	\begin{align*}
		[R(X,Y),I]
		&=\big(\nabla_X(\nabla_Y I)-\nabla_Y(\nabla_X I)-\nabla_{[X,Y]}I\big)\\
		&=\big(\d(\alpha(Y))(X)-\d(\alpha(X))(Y)-\alpha([X,Y])\big)J\\
		&\,-\big(\d(\beta(Y))(X)-\d(\beta(X))(Y)-\beta([X,Y])\big)K\\
		&\,+\alpha(Y)\nabla_X J-\alpha(X)\nabla_Y J
		-\beta(Y)\nabla_X K+\beta(X)\nabla_Y K
	\end{align*}
	The terms featuring covariant derivatives of $J$ and $K$ cancel, so we find that 
	\begin{equation}\label{eq:Icurvcommutator}
		[R(X,Y),I]=\alpha(X,Y)J-\beta(X,Y)K
	\end{equation}
	where $\alpha(X,Y)\coloneqq \d(\alpha(Y))(X)-\d(\alpha(X))(Y)-\alpha([X,Y])$ is a two-form and $\beta$ is defined analogously. In similar fashion, one finds that
	\begin{align}\label{eq:Jcurvcommutator}
		[R(X,Y),J]&=-\alpha(X,Y) I+\gamma(X,Y) K\\\label{eq:Kcurvcommutator}
		[R(X,Y),K]&=\beta(X,Y) I-\gamma(X,Y) J
	\end{align}
	Now we need a computational lemma: 
	
	\begin{lem}\label{lem:ricciforms}
		The forms $\alpha$, $\beta$, $\gamma$ are related to the Ricci curvature $r$ via:
		\begin{align*}
			\alpha(X,Y)=\frac{1}{n+2}r(KX,Y)\\ 
			\beta(X,Y)=\frac{1}{n+2}r(JX,Y)\\
			\gamma(X,Y)=\frac{1}{n+2}r(IX,Y)
		\end{align*}
		where $n=\dim M/4$.
	\end{lem}
	\begin{myproof}[Proof of Lemma]
		Starting from \eqref{eq:Kcurvcommutator}, we have:
		\begin{align*}
			g([R(X,Y),K]Z,JZ)&=
			g(\beta(X,Y)IZ,JZ)-g(\gamma(X,Y)JZ,JZ)\\
			&=-\gamma(X,Y)\lvert Z\rvert^2
		\end{align*}
		Using the quaternionic relations and the symmetries of $R$ on the left-hand side, this becomes
		\begin{equation}\label{eq:gammacurvrelation}
			g(R(X,Y)JZ,KZ)+g(R(X,Y)Z,IZ)=\gamma(X,Y)\lvert Z\rvert^2
		\end{equation}
		Now we pick a local orthonormal basis $\{X_i\}$ of $TM$, adapted to the quaternionic coordinates in the sense that if $X_i$ is a basis element, then $IX_i$, $JX_i$ and $K X_i$ are too (up to sign). This means that the set of pairs $(X_i,IX_i)$ is at the same time the set of pairs $(JX_i,KX_i)$; the above identity for $Z=X_i$ yields, when summed over $i$, the identity
		\begin{equation*}
			2n\gamma(X,Y)=\sum_{i=1}^{4n}g(R(X,Y)X_i,IX_i)
		\end{equation*}
		The first Bianchi identity applied to the last three entries of the right-hand side yields
		\begin{align*}
			2n\gamma(X,Y)
			&=\sum_{i=1}^{4n}\Big(g(R(X,X_i)Y,IX_i)-g(R(X,IX_i)Y,X_i)\Big)
		\end{align*}
		Both terms contribute equally so we find
		%WORK NEEDED: Why?
		\begin{equation*}
			n\gamma(X,Y)=\sum_ig(R(X,X_i)Y,IX_i)=-\sum_i g(IR(X,X_i)Y,X_i)
		\end{equation*}
		Now we can use \eqref{eq:Icurvcommutator}:
		\begin{align*}
			n\gamma(X,Y)&=\sum_i\Big(-g(R(X,X_i)IY,X_i)+\alpha(X,X_i)g(JY,X_i)
			-\beta(X,X_i)g(KY,X_i)\Big)\\
			&=-r(X,IY)+\alpha(X,JY)-\beta(X,KY)
		\end{align*}
		Replacing $Y$ by $IY$, this means that
		\begin{equation*}
			n\gamma(X,IY)+\beta(X,JY)+\alpha(X,KY)=r(X,Y)
		\end{equation*}
		Carrying out identical calculations for cyclic permutations of $\{I,J,K\}$, one obtains
		\begin{align*}
			\gamma(X,IY)+n\beta(X,JY)+\alpha(X,KY)=r(X,Y)\\
			\gamma(X,IY)+\beta(X,JY)+n\alpha(X,KY)=r(X,Y)
		\end{align*}
		Since $n\geq 2$, this suffices to conclude that
		\begin{equation*}
			\gamma(X,IY)=\beta(X,JY)=\alpha(X,KY)=\frac{1}{n+2}r(X,Y)
		\end{equation*}
		from which the claim easily follows.
	\end{myproof}
	Now it is not hard to finish the proof of the theorem. Note that
	\begin{equation*}
		r(X,Y)=r(IX,IY)=r(JX,JY)=r(KX,KY)
	\end{equation*}
	and therefore, using \eqref{eq:gammacurvrelation}, we find:
	\begin{align*}
		\frac{2\lvert Z\rvert^2}{n+2}r(X,X)
		&=\frac{\lvert Z\rvert^2}{n+2}(r(X,X)+r(JX,JX))\\
		&=g(R(X,IX)Z,IZ)+g(R(X,IX)JZ,KZ)\\
		&+g(R(JX,ZX)Z,IZ)+g(R(JX,ZX)JZ,KZ)
	\end{align*}
	for any $X,Z$. But the last expression is symmetric under exchanging $X,Z$ and therefore we find that
	\begin{equation*}
		r(X,X)\lvert Z\rvert^2=r(Z,Z)\lvert X\rvert^2
	\end{equation*}
	and therefore $r(X,X)/\lvert X\rvert^2$ is independent of $X$, hence simply a constant. This means that $r(X,X)=\lambda \lvert X\rvert^2$: We have proven that our manifold is Einstein.
\end{myproof}

\begin{rem}
	Salamon~\cite{Sal1982} gave a different proof, using representation theory. In fact, his method determines the precise form of the curvature tensor, relating it to the curvature tensor of $\HP^n$. As a corollary of his analysis one can prove, among other things, that a quaternionic K\"ahler manifold with nonzero scalar curvature is not (even locally) reducible in the sense of De Rham's theorem (cf.~\cite[Thm. 14.45]{Bes2008}).
\end{rem}

\begin{cor}
	A quaternionic K\"ahler manifold has vanishing Ricci curvature if and only if it is locally hyper-K\"ahler.
\end{cor}
\begin{myproof}
	The Ricci curvature vanishes precisely if the two-forms $\alpha$, $\beta$ and $\gamma$ introduced above all vanish. Then the locally defined endomorphisms $I$, $J$ and $K$ are parallel; the existence of such parallel almost complex structures is one of the standard ways of defining a hyper-K\"ahler structure.
\end{myproof}

By the reduction theorem, the reduced holonomy of a quaternionic K\"ahler manifold $(M,g)$ is equivalent to a reduction of the frame bundle of $M$ to a principal $Sp(n)\cdot Sp(1)$-bundle $P$, along with a reduction of the Levi-Civit\`a connection to a connection on $P$. 

Locally, there is no obstruction to lifting such an $Sp(n)\cdot Sp(1)$-structure to an $Sp(n)\times Sp(1)$-structure, obtaining a principal $Sp(n)\times Sp(1)$-bundle $\tilde P$. Thus, a representation $\rho$ of $Sp(n)\times Sp(1)$ on a vector space $V$ \emph{locally} yields a vector bundle $\tilde P\times_\rho V$. $V$ may fail to be globally defined: It can be constructed globally if the representation factors through $Sp(n)\cdot Sp(1)$ or if the lift $\tilde P$ exists globally.

Now consider the standard representations of $Sp(n)$ and $Sp(1)$ on $\C^{2n}$ and $\C^2$; we denote them by $E$ and $H$. Right-multiplication by $j\in \H$ yields a quaternionic structure $J_E$, $J_H$, and using the standard Hermitian product $\langle-,-\rangle$ we obtain a two-form $\omega_H(v,w)=\langle J_H v,w\rangle$ which satisfies $\omega_H(Jv,Jw)=\overline{\omega_H(v,w)}$ and $\omega_H(v,Jv)\geq 0$ (and analogously for $E$). They induce identifications $H\cong H^*$ ($E\cong E^*$). Since these representations are therefore faithful and self-dual, the Peter-Weyl theorem implies that every irreducible representation of $Sp(n)\times Sp(1)$ is contained in $(\otimes^p E)\otimes(\otimes^q H)$ for some $p,q\geq 0$ (cf.~\cite[Thm. III.4.4]{BT1982}). Note that these factor through $Sp(n)\cdot Sp(1)$ precisely if $p+q$ is even, since then the elements $(\pm \id,\pm \id)$ are sent to the same automorphism. 

All of this carries over to the associated vector bundles of $\tilde P$ and $P$. The quaternionic structure of the fibers gives us a notion of complex conjugation; fiberwise taking the subspace of invariant elements yields a real vector bundle of rank equal to the (complex) rank of the original associated bundle. We will from now on discuss these real vector bundles and denote them by the same letters as the representations that give rise to them. The fundamental representation of $Sp(n)\cdot Sp(1)$, $E\otimes H$, defines the (co)tangent bundle. We take the convention that $T^*M=E\otimes H$.

Clearly, a basic invariant of an $Sp(n)\cdot Sp(1)$-structure on $M$ is whether or not it lifts to an $Sp(n)\times Sp(1)$-structure. It turns out that this can be related to the bundle $S^2H$. The short exact sequence
\begin{equation}\label{eq:SpSES}
	\begin{tikzcd}
		0 \ar[r] & \Z_2 \ar[r] & Sp(n)\times Sp(1) \ar[r] & Sp(n)\cdot Sp(1) \ar[r] & 0
	\end{tikzcd}
\end{equation}
induces a long exact sequence and in particular a coboundary homomorphism
\begin{equation*}
	\begin{tikzcd}
		\delta\colon H^1(M;Sp(n)\cdot Sp(1))\ar[r] & H^2(M;\Z_2)
	\end{tikzcd}
\end{equation*}
The image of the $Sp(n)\cdot Sp(1)$-principal bundle $P$ under $\delta$ is the obstruction to lifting $P$ to a $Sp(n)\times Sp(1)$-principal bundle $\tilde P$. This class, which we will denote by $\varepsilon$, is equivalently the obstruction to the global existence of the vector bundles $E$ and $H$. There is another short exact sequence 
\begin{equation}\label{eq:SO3SES}
	\begin{tikzcd}
		0\ar[r] & \Z_2 \ar[r] & Sp(1) \ar[r] & SO(3) \ar[r] & 0
	\end{tikzcd}
\end{equation}
and the (three-dimensional) representation $S^2H$ of $Sp(n)\cdot Sp(1)$ determines a homomorphism from \eqref{eq:SpSES} to \eqref{eq:SO3SES}. On the level of the long exact sequence, we have a commutative ladder
\begin{equation*}
	\begin{tikzcd}[column sep=small]
		\dots \ar[r] & H^1(M;Sp(n)\times Sp(1)) \ar[r] \ar[d] & H^1(M;Sp(n)\cdot Sp(1)) \ar[r,"\delta"] \ar[d] 
		& H^2(M;\Z_2) \ar[r] \ar[draw=none]{d}[sloped,auto=false]{\scalebox{1.3}[1.3]{$=$}} & \dots \\
		\dots \ar[r] & H^1(M;Sp(1)) \ar[r] & H^1(M;SO(3)) \ar[r,"\delta'"] & H^2(M;\Z_2) \ar[r] & \dots \\
	\end{tikzcd}
\end{equation*}
The middle vertical map sends $P$ to $S^2H$, and $\delta'(S^2H)=w_2(S^2H)$, so we may identify $\varepsilon=w_2(S^2H)$. Using spectral sequences, Salamon relates this class to the characteristic classes of $TM$ (see Marchiafava \& Romani~\cite{MR1975} for an alternative approach). We state the result without proof:

\begin{prop}\label{prop:QK8nspin}
	If $(M^{4n},g)$ is quaternionic K\"ahler, then $w_2(M)=n\varepsilon$. In particular, $8n$-dimensional quaternionic K\"ahler manifolds are spin.
\end{prop}

\section{Examples of quaternionic K\"ahler manifolds}

We already saw that the quaternionic K\"ahler manifolds include both hyper-K\"ahler manifolds and the (not even almost complex) quaternionic projective spaces $\HP^n$. Since quaternionic K\"ahler manifolds are Einstein, they are naturally divided into three categories according to the sign of the scalar curvature $s_g$. In the following, we will focus on the case $s_g\neq 0$, as $s_g=0$ implies that the manifold is locally hyper-K\"ahler.

Though we will mainly be interested in the case $s_g>0$, we briefly comment on the negative scalar curvature case. Besides the examples due to Wolf, which we discuss below, there is a family of homogeneous but not always symmetric spaces that were first discovered by Alekseevski\u\i; their classification was completed by Cort\'es~\cite{Cor1996}. More examples arose in the context of supergravity theories in physics; see e.g.~\cite{ACDM2015,Cor2010}. One interesting fact is that all \emph{compact} examples constructed thus far are at least locally symmetric.

In the case of positive scalar curvature, Myers' theorem shows that complete examples must be compact. Here, too, there is a shortage of non-symmetric examples. In fact, the only known examples of quaternionic K\"ahler manifolds with $s_g>0$ are symmetric spaces, which were constructed by Wolf~\cite{Wol1965}. Wolf classified the \emph{symmetric} quaternionic K\"ahler manifolds, which are called \emph{Wolf spaces} in his honor. As remarked in the previous section, quaternionic K\"ahler manifold with nonzero scalar curvature are irreducible, hence the symmetric examples can be found from Cartan's classification of irreducible symmetric spaces. 

Representation-theoretic arguments can be used to pick out the correct entries from Cartan's classification (for details, see~\cite[\nopp \S 14.50]{Bes2008}). Since the irreducible symmetric spaces come in pairs (of the form $G/K$, $G^*/K$, where $G$ is a compact Lie group and $G^*$ is its non-compact dual), the Wolf spaces do too. In each pair, the compact example has $s_g>0$ and the non-compact one $s_g<0$. Thus, we can associate a compact, simple Lie group $G$ to each pair. The compact  Wolf spaces are then of the form $(G/Z(G))/H$, where $Z(G)$ denotes the center of $G$. They are listed in \cref{tab:Wolfspaces}.

\begin{table}[ht!]\centering
	\begin{tabular}{lll} \toprule
		$\dim M$	& $G$			
		& $H$							\\ \midrule
		$4n$ 				& $Sp(n+1)$		
		& $Sp(n)\cdot Sp(1)$			\\
		$4n$ 				& $SU(n+2)$		
		& $S(U(n)\times U(2))\cong U(n)\cdot Sp(1)$					\\
		$4n$					& $SO(n+4)$		
		& $S(O(n)\times O(4))\cong (SO(n)\times Sp(1))\cdot Sp(1)$	\\
		$8$					& $G_2$			
		& $SO(4)\cong Sp(1)\cdot Sp(1)$								\\
		$28$					& $F_4$			
		& $Sp(3)\cdot Sp(1)$			\\
		$40$				& $E_6$			
		& $SU(6)\cdot Sp(1) $			\\
		$64$				& $E_7$			
		& $\Spin(12)\cdot Sp(1)$		\\ 
		$112$				& $E_8$			
		& $E_7\cdot Sp(1)$				\\ \bottomrule
	\end{tabular}
	\caption{}\label{tab:Wolfspaces}
\end{table}

If one allows for $n=1$ in the three infinite families of \cref{tab:Wolfspaces}, one finds to obtain a correspondence with all compact, simple Lie groups (except $SU(2)$). The four-dimensional Wolf spaces are
\begin{equation*}
	\frac{Sp(2)/\Z_2}{Sp(1)\cdot Sp(1)}\cong \HP^1=S^4\cong \frac{SO(5)}{S(O(1)\times O(4))}
\end{equation*}
and
\begin{equation*}
	\frac{SU(3)}{S(U(1)\times U(2)}=\CP^2
\end{equation*}

Note that, except for $\HP^n\cong Sp(n+1)/(Sp(n)\times Sp(1))$ and $SO(5)/(Sp(1)\times Sp(1))$, none of these spaces can be written as $G/(Sp(n)\times Sp(1))$ and therefore the obstruction class $\varepsilon$ does not vanish for most Wolf spaces.

The difficulty in finding other examples of complete quaternionic K\"ahler manifolds with positive scalar curvature led to the following conjecture:

\begin{con*}[LeBrun-Salamon~\cite{LS1994}]
	The compact Wolf spaces are the only complete quaternionic K\"ahler manifolds with positive scalar curvature.
\end{con*}

An early theorem due to Alekseevski\u\i~\cite{Ale1968} shows that every compact, homogeneous quaternionic K\"ahler manifold with nonzero scalar curvature is a Wolf space. Further progress was made by Poon and Salamon~\cite{PS1991}, who proved the conjecture in dimension eight. A proof in dimension twelve was offered by Herrera and Herrera~\cite{HH2002}, but recently retracted. There are partial results in higher dimensions (cf.~\cite{Ama2012} and references therein).

Finally, a further comment on the work of Wolf is in order. Besides noting that compact, symmetric quaternionic K\"ahler manifolds correspond to compact, simple Lie groups, Wolf observed that there is another type of manifolds whose classification takes on a similar form:

\begin{mydef}\label{def:holoctcmnf}
	A complex manifold $M$ of odd (complex) dimension $2n+1$ is said to admit a \emph{holomorphic contact structure} if there exists a family $\{(U_i,\omega_i)\}$ with the following properties:
	\begin{numberedlist}
		\item $\{U_i\}$ is an open covering of $M$ and $\omega_i$ is a holomorphic one-form on $U_i$ such that $\omega_i\wedge (\d\omega_i)^n$ is a nowhere-vanishing local section of the canonical bundle $K_M$.
		\item If $U_i\cap U_j\neq \varnothing$, there exists a holomorphic function $f_{ij}$ on $U_i\cap U_j$ such that $\omega_i=f_{ij}\omega_j$.
	\end{numberedlist}
	The $\omega_i$ are then called (local) holomorphic contact forms. $M$ is called a \emph{homogeneous holomorphic contact manifold} if the group of biholomorphic contactomorphisms (i.e.~biholomorphisms $f:M\to M$ such that $f^*\omega_i$ is a local holomorphic contact form) acts transitively.
\end{mydef}

Boothby classified the compact, simply connected, homogeneous holomorphic contact manifolds in~\cite{Boo1961}. Just as the compact Wolf spaces, they correspond bijectively to compact, simple Lie groups $G$ via a quotient: $M=(G/Z(G))/L$ where $L$ is uniquely determined up to conjugacy and of the form $L_1\cdot U(1)$. Moreover, Boothby proved that these manifolds admit a K\"ahler metric.

Compact Wolf spaces are of a similar form, namely $(G/Z(G))/K$ where $K=K_1\cdot Sp(1)$. Wolf proved that $L_1=K_1$ and that the $U(1)$-factor embeds into $Sp(1)$, establishing a correspondence between the compact, simply connected, homogeneous holomorphic contact manifolds and compact Wolf spaces. This correspondence is given by a fiber bundle $\pi: G/L\to G/K$ with fiber $Sp(1)/U(1)=\CP^1$. Thus, every compact Wolf space has a canonically associated $S^2$-bundle over it, the total space of which is a K\"ahler manifold. This motivates, and is generalized by, the construction of the \emph{twistor space} associated to a quaternionic K\"ahler manifold, which we will discuss next.

\section{The twistor space}

In order to introduce the twistor space of a quaternionic K\"ahler manifold $M$, we give another way to view the bundle $S^2H$ over $M$ (due to Salamon), which plays a fundamental role. Since $Sp(1)$ consists of automorphisms of $H$, we may regard its Lie algebra $\mf{sp}(1)=\mf{su}(2)$ as a subset of $H\otimes H^*$. Under the identification with $H\otimes H$ provided by $\omega_H$, $\mf{su}(2)$ corresponds precisely to $S^2H$. There is an action on the tangent bundle $TM=E^*\otimes H^*$ given by the composition
\begin{equation*}
	\begin{tikzcd}
		TM\otimes S^2H\ar[r,hook] &
		(E^*\otimes\, \underbracket{\!\! H^*)\otimes (H\!\!}\,\otimes H^*)
		\ar[r] & TM
	\end{tikzcd}
\end{equation*}
Moreover, for $J,K\in S^2H\subset H\otimes H$ we have the identity $JK+KJ=-\langle J,K\rangle \id$ as endomorphisms, where the inner product $\langle -,-\rangle$ is induced by the standard Hermitian product on $H$. Therefore, we may think of $S^2H$ as a ``bundle of quaternionic coefficients'', acting by quaternionic multiplication from the right. Locally, it has a basis $\{I,J,K\}$ that satisfies the standard quaternionic relations. Of course, what we are describing is nothing but the three-dimensional (sub)bundle of endomorphisms that features in the characterization of quaternionic K\"ahler manifolds given in \cref{prop:QKcharts}.

\begin{mydef}
	The \emph{twistor space} $Z$ associated to a quaternionic K\"ahler manifold $(M,g)$ is the unit sphere bundle of $S^2H$.
\end{mydef}

\begin{rem}
	Locally, we may pick a frame for $S^2H$ and describe the twistor space as the bundle of unit length quaternions, acting on $TM$ by right-multiplication. The fiber $Z_x$ over a point $x\in M$ then consists of those almost complex structures of $T_xM$ that are compatible with the $Sp(n)\cdot Sp(1)$ structure. 
	
	To make this more precise, recall that on an even-dimensional, oriented Riemannian manifold $M^{2n}$, the space of almost complex structures on $T_xM$ compatible with the metric and orientation is $SO(2n)/U(n)$. Analogously, the space of almost complex structures compatible with the $Sp(n)\cdot Sp(1)$ structure of a quaternionic K\"ahler manifold $M^{4n}$ is
	\begin{equation*}
		\frac{Sp(n)\cdot Sp(1)}{U(2n)\cap (Sp(n)\cdot Sp(1))}\cong 
		\frac{Sp(n)\cdot Sp(1)}{Sp(n)\cdot U(1)}\cong \frac{Sp(1)}{U(1)}\cong \CP^1
	\end{equation*}
	%The intersection with U(2n) is Sp(n)U(1) since Sp(n)\subset U(2n) and those units of H that commute with the standard copmlex structure on R^4 are exactly e^{i\varphi},i.e. a U(1) subgroup
	Because of the identity $JK+KJ=-\langle J,K\rangle \id$, the unit length quaternions correspond to those $J\in S^2H$ of length $\sqrt 2$ with respect to the inner product $\langle-,-\rangle$, but we may of course rescale our inner product to describe $Z$ as the \emph{unit} sphere bundle. 
\end{rem}

We give yet another way of viewing the twistor bundle, which provides a clear way of seeing that it is a bundle of almost complex structures. Over a quaternionic K\"ahler manifold $M$, we may \emph{locally} define the vector bundles $E$ and $H$. Any element $h\in H_x\setminus \{0\}$ defines a subspace of $(1,0)$-forms (and therefore an almost complex structure):
\begin{equation*}
	\bigwedge\nolimits^{1,0}_xM=E_x\otimes \C h\subset T_x^*M\otimes_\R\C 
\end{equation*}
Complex conjugation shows that $\bigwedge^{0,1}_x M=E_x\otimes \C\tilde h$, where $\tilde h$ satisfies $\omega_H(h,\tilde h)=1$. The induced almost complex structure is unchanged if we consider a complex multiple of $h$ instead, hence the space of almost complex structures can be identified with $\P(H)$, the projectivized bundle. Since the bundle of almost complex structures is the twistor space $Z$, we find that $Z=\P(H)$ (of course, this characterization is only local, as $H$ is not globally defined in general). 

\section{Properties of the twistor space}

The usefulness of the twistor space construction mainly derives from the following fundamental theorem:

\begin{thm}[Salamon~\cite{Sal1982}]\label{thm:twistorcpx}
	The twistor space $\pi: Z\to M$ of a quaternionic K\"ahler manifold admits an (integrable!) complex structure such that the fibers are complex submanifolds.
\end{thm}
\begin{myproof}
	Recall that we may view $Z$ as the bundle whose fibers $Z_x$ consist of the almost complex structures of $T_xM$ compatible with the $Sp(n)\cdot Sp(1)$ structure, i.e.~$Sp(n)\cdot Sp(1)/(U(2n)\cap Sp(n)\cdot Sp(1))\cong \CP^1$. Since this bundle is associated to the (reduced) principal $Sp(n)\cdot Sp(1)$ frame bundle, the Levi-Civit\`a connection on $M$ induces a connection on $Z$, hence a splitting $TZ=\mc H\oplus \mc V$, where $\mc V=T\pi$ is the vertical distribution of tangent vectors along the fibers. 
	
	We know that, locally, $Z=\P(H)$ and therefore we equip the fibers with the standard complex structure on $\CP^1$, induced by $H_x=\C^2$. This defines an almost complex structure $J_v$ on $\mc V$. Now, we use the fact that $p\in Z_x$ is an almost complex structure on the tangent space $T_xM$. $\dif\pi$ induces an isomorphism $\mc H_p\cong T_xM$ and therefore we find an almost complex structure on $\mc H_p$. Doing this in every point, we find a tautological almost complex structure $J_h$ on $\mc H$ (it is clear that $J_h$ depends smoothly on the point in $Z$, because it essentially \emph{is} the point). Using the splitting $TZ=\mc H\oplus \mc V$, we define the almost complex structure on $TZ$ by $J=J_h\oplus J_v$. Note that the fibers are automatically (almost) complex submanifolds.
	
	Now, we have to prove integrability of $J$. The celebrated Newlander-Nirenberg theorem~\cite{NN1957} reduces this to showing that the Nijenhuis tensor $N_J$ associated to $J$ vanishes. The Nijenhuis tensor is the $(1,2)$-tensor field given by
	\begin{equation*}
		N_J(X,Y)=[X,Y]+J[JX,Y]+J[X,JY]-[JX,JY] \qquad \qquad X,Y\in T_pZ
	\end{equation*}
	Using the decomposition $TZ=\mc H\oplus \mc V$, it suffices to show that $N_J(\mc H,\mc H)=0$, $N_J(\mc H,\mc V)=0$ and $N_J(\mc V,\mc V)=0$. The last of these is the easiest: Since the fibers are (almost) complex submanifolds, $[\mc V,\mc V]\subset \mc V$ and $J\mc V=J_v\mc V\subset \mc V$. This means that $N_J\big|_{\mc{V}_x}=N_{J_v}$ is simply the Nijenhuis tensor on $\CP^1$ associated with the (standard) complex structure, which is of course integrable. Hence $N_J(\mc V,\mc V)=0$. 
	
	To prove the vanishing of the remaining components of the Nijenhuis tensor, we will need some notions from the theory of Riemannian submersions, introduced in \cref{app:submersions}, in particular that of a basic vector field. A vector field $X$ is called basic if it is horizontal an $\pi$-related to a vector field $\check X$ on $M$. An important property is that, if $U$ is a vertical vector field, then $[X,U]$ is vertical.
	
	Our next step is to prove that for $N_J(\mc H,\mc V)=0$. Given a horizontal and a vertical tangent vector, we extend them to a basic and a vertical vector field, which we call $X$ and $U$. The action of $Sp(n)\cdot Sp(1)$ on the locally defined bundle $H$ factors through $U(2)$ and therefore the horizontal transport associated to $\mc H$ leaves the Fubini-Study metric on $Z_x=\P(H_x)=\CP^1$ invariant, as well as the orientation. $J_v$ is uniquely determined in terms of these data, hence must be preserved as well. This means that $[X,JU]=J[X,U]$ (we used that the term $(JU)X$ on the left hand side does not contribute, as $[X,JU]$ is vertical).
	%Since in coordinates obtianed by horizontally transporting a frame adapted to the product structure, $\partial_i J=0$ where i are the horizontal coordinates
	Thus, the first and third terms of the Nijenhuis tensor cancel. 
	
	To investigate the second and fourth terms, we fist consider the vertical projection $\mc V N_J(X,U)=\mc V(J[JX,U]-[JX,JU])$. Though $JX$ is not necessarily basic, the vertical projection ensures that we still have $\mc V[JX,JU]=\mc VJ[JX,U]=J\mc V[JX,U]$, hence $\mc V N_J(X,U)=0$. Now consider $\mc H N_J(X,U)$, which vanishes if and only if its projection to the base does. Define a map $\varphi:Z\to \End TM$ by sending $z\in Z_x$ to the corresponding complex structure on $T_xM$. Expanding in local coordinates adapted to the local product structure of $Z$, we see that the following holds pointwise:
	\begin{equation*}
		\dif \pi(\mc H[J X,U])=-\dif \varphi(U)\check X
	\end{equation*}
	where $\check X=\dif \pi(X)$ is $\pi$-related to $X$. The identity $\dif_z\varphi((JU)_z)=\varphi(z)\circ \dif_z \varphi(U_z)$,
	%Think about U, JU geometrically as elements of the sphere/R^3 given by endomorphism of TM already; then this is easy if you just realize that the complex structure is given by z\times where z\in S^2\subset \R^3
	which holds for all vertical vectors $U$, then shows that the projection to the base vanishes. Indeed:
	\begin{align*}
		\dif_z\pi(\mc H N_J(X,U)_z)
		&=\dif_z\pi (\mc HJ[JX,U]_z)-\dif_z\pi (\mc H[JX,JU]_z)\\
		&=\dif_z\pi (J_h\mc H[JX,U]_z)
		+\varphi(z)\circ\dif_z\varphi(U_z) \check X_{\pi(z)}\\
		&=-\varphi(z)\circ \dif_z \varphi(U)\check X_{\pi(z)}
		+\varphi(z)\circ\dif_z\varphi(U_z) \check X_{\pi(z)}=0
	\end{align*}
	In the last line we used that, for $X$ horizontal, $\dif_z \pi(J_h X_z)=\varphi(z)\circ\dif_z \pi(X_z)$ essentially by definition of $J_h$.
	
	Finally, we have to prove that $N_J(\mc H,\mc H)=0$. Regard $Z$ as a subspace of the rank three bundle $p:S^2H\to M$ and take any point $z\in Z$. We can find a section $s$ of $\pi: Z\to M$, defined on a neighborhood $V$ of $\pi(z)$, that passes through $z$ and is parallel at that point with respect to $\nabla$, induced on $S^2H$ by the Levi-Civit\`a connection $\check\nabla$ on $M$. Then $s$ defines an almost complex structure $S$ on $V$ and on its image $s(V)$, $\dif\pi(\mc H N_J(\mc H,\mc H))$ equals $N_S$. Since $(\nabla S)_{\pi(z)}=0$ and $\nabla$ is torsion-free, $N_S$ vanishes identically at $\pi(z)$. We may do this for any point $z\in Z$ and therefore we conclude that $\mc H N_J\equiv 0$.
	
	It remains to prove that $\mc V N_J(\mc H,\mc H)$ vanishes. On $S^2H$, we define $A:\mc H\times \mc H\to \mc V$, $A_X Y=\frac{1}{2}\mc V[X,Y]$ (compare with O'Neill's $A$-tensor from \cref{app:submersions}). It measures the obstruction to integrability of $\mc H$ and, after identifying $\mc V T_z S^2H_{\pi(z)}$ with $S^2H_{\pi(z)}$, it is related to the curvature via $(A_X Y)_{s(x)}=-\frac{1}{2}R^\nabla(\dif p(X),\dif p(Y))s(x)$ for any section $s\in \Gamma(S^2H)$. Since $S^2H$ is a bundle of endomorphisms, we may consider $A_X Y$ as an endomorphism (this amounts to applying $\varphi$ to it), and $R^\nabla$ as the curvature naturally induced on $\End TM$ by $\check \nabla$. The equation now becomes:
	\begin{equation*}
		\varphi(A_XY)=-\frac{1}{2}[R^{\check\nabla}(\dif \pi(X), \dif \pi(Y)),S]
	\end{equation*} 
	where $S\in \End TM$ is the image of the section $s$ as before, and we used that $R^{\End E}(a)=[R^E,a]$ for a vector bundle $E$ and $a\in\End E$. We also replaced $\dif p$ by $\dif \pi$, which is allowed since $X$ is a vector field on $Z\subset S^2H$. $\mc VN_J(\mc H,\mc H)$ vanishes if and only if $\varphi(\mc VN_J(\mc H,\mc H)$ does. By the above, we may express this as follows:
	\begin{equation*}
		-[R^{\check{\nabla}}(\check X,\check Y),S]
		-S[R^{\check{\nabla}}(S\check X,\check Y),S]
		-S[R^{\check{\nabla}}(\check X,S\check Y),S]
		+[R^{\check{\nabla}}(S\check X,S\check Y),S]=0
	\end{equation*}
	Here, $S$ is the complex structure on $T_x M$ corresponding to $s(x)\in Z_x$ and $\check X,\check Y$ correspond to basic vector fields on $Z$. To prove this identity, we use \cref{eq:Icurvcommutator,eq:Jcurvcommutator,eq:Kcurvcommutator} and \cref{lem:ricciforms}. They imply that, if $S=aI+bJ+cK$, then
	\begin{equation*}
		(n+2)[R^{\nabla}(X,Y),S]
		=(bK-cJ)r(IX,Y) +(cI-aK)r(JX,Y)+(aJ-bI)r(KX,Y)
	\end{equation*}
	where we have suppressed the accents for simplicity. Using the analogous equations for $R^{\nabla}(SX,Y)$ etc., as well as the equations $r(SX,SY)=r(X,Y)$, $r(ISX,Y)=r(IX,SY)-2ar(X,Y)$ and its analogs for $J$ and $K$, we can write our expression in terms of $r(X,Y)$, $r(IX,Y)$, $r(IX,SY)$ and similar terms for $J$ and $K$. 
	
	Now, it is a tedious but in principle simple process to eliminate all occurrences of $S$, using relations such as $S(bK-cJ)=(b^2+c^2)I-abJ-acK$ and $r(IX,SY)=ar(X,Y)+br(KX,Y)-cr(JX,Y)$, as well as the fact that $a^2+b^2+c^2=1$. Once this task has been completed, all terms cancel and the expression vanishes.
\end{myproof}

This important result allows one to study the geometry of quaternionic K\"ahler manifolds though the complex geometry of the twistor space, although we will not use the twistor space for these purposes in this work. 

In forming expectations of what one may be able to prove about quaternionic K\"ahler manifolds, some guidance is provided by the Wolf spaces. Recall that each Wolf space comes with a homogeneous holomorphic contact manifold fibering over it with fiber $\CP^1$: This is precisely the twistor space. Thus, the above theorem generalizes Wolf's result regarding the complex structure. Our next aim is to prove that a large class of twistor spaces carry a holomorphic contact structure. To this end, we first give some more information about holomorphic contact manifolds.

The kernels of the local contact forms on a holomorphic contact manifold $X$ of complex dimension $2n+1$ (cf.~\cref{def:holoctcmnf}) unambiguously define a codimension one distribution $D$. Picking a complementary complex line bundle $F$, the non-degeneracy condition on the $\omega_i$'s is equivalent to the statement that $(\d \omega_i)^n\big|_D$ is nowhere vanishing. Because $\omega_i\big|_D=0$, we have $\d \omega_i\big|_D=f_{ij}\d \omega_j\big|_D$, hence we get an up to a multiple well-defined two-form of maximal rank on $D$. 

Note that $\omega_i\wedge (\d \omega_i)^n=f_{ij}^{n+1}\omega_j\wedge (\d\omega_j)^n$, hence sections of the canonical bundle $K_X$ transform under transition functions via multiplication by $f_{ij}^{n+1}$. This means that the canonical bundle is defined by the transition functions $\big\{f_{ij}^{-(n+1)}\big\}$. On the other hand, the line bundle $F$ is defined by $\{f_{ij}\}$ and therefore $K_X\cong F^{-(n+1)}$. This means that $c_1(X)=(n+1)c_1(F)$. Thus, we have proven:

\begin{prop}[Kobayashi~\cite{Kob1959}]\label{prop:cpxctcdivisible}
	If $X$ is a complex manifold of dimension $2n+1$ which admits a holomorphic contact structure, then $c_1(X)$ is divisible by $n+1$.
\end{prop}

There is a \emph{global} alternative for the (local) definition of a holomorphic contact manifold that we have used thus far. Consider a complex manifold $X$ of dimension $2n+1$ which admits a (complex) codimension one holomorphic distribution $D\subset TX$. The quotient $TX/D$ yields a holomorphic line bundle $F$:
\begin{equation*}
	\begin{tikzcd}
		D\ar[r,hook] & TX \ar[r,"\alpha"] & F
	\end{tikzcd}
\end{equation*}
and we may regard the projection $\alpha$ as a holomorphic one-form with values in $F$, i.e.~a global section of $\Omega^1(X)\otimes F$. Note that it is nowhere vanishing (since $F$ is of constant rank one). Now, $D$ defines a holomorphic contact structure on $X$ if it satisfies a condition known as maximal non-integrability. Concretely, this translates to the condition $\alpha\wedge (\d \alpha)^n\neq 0$ (via a variant of Frobenius' theorem) on the form $\alpha$. Note that, although the exterior derivative of this bundle-valued form depends on a choice of connection, the contact condition does not.

The definition of a holomorphic contact structure in terms of local contact forms is recovered upon picking local trivializations of $F$: The globally defined form $\alpha$, which takes values in $F$, can then be viewed as a collection of locally defined forms $\alpha_i$ (complex-valued) on $TZ$ of the form $\alpha_i=\pi^*\omega_i$, where $\pi:TZ\to Z$ is the projection. The locally defined forms $\omega_i$ on $Z$ then satisfy the conditions of \cref{def:holoctcmnf}. We use this global point of view to prove that twistor spaces over quaternionic K\"ahler manifolds with nonzero Einstein constant are holomorphic contact manifolds:

\begin{thm}[Salamon \cite{Sal1982}]\label{thm:twistorctc}
	The twistor space $\pi:Z\to M$ of a quaternionic K\"ahler manifold with nonzero scalar curvature admits a holomorphic contact structure.
\end{thm}
\begin{myproof}
	Using the splitting $TZ=\mc V\oplus \mc H$ induced by the Levi-Civit\`a connection, we can view the projection onto $\mc V$ (also denoted by $\mc V$) as a $\mc V$-valued holomorphic one-form $\alpha$ on $Z$. It suffices to show that $\d^{\hat\nabla}\alpha$ is nowhere vanishing and of maximal rank, when restricted to $\mc H$. Here, $\hat\nabla$ is the connection on $\mc V$ induced by the Fubini-Study metric on each fiber.
	
	Let $X,Y$ be horizontal tangent vectors. A short, local computation shows that $(\d^{\hat\nabla}\alpha)(X,Y)=-2\alpha(A_XY)$, where $A_XY\coloneqq\frac{1}{2}\mc V[X,Y]$. We will now express $A_X Y$, which we view as an endomorphism, in terms of the Ricci curvature (and hence the metric) and use this to prove that $A$ is non-degenerate. Then $\d^{\hat\nabla}\alpha$ is non-degenerate and hence $\alpha\wedge (\d^{\hat\nabla}\alpha)^n\neq 0$, i.e.~$Z$ is a holomorphic contact manifold.
	
	To this end, recall the following identity from the proof of \cref{thm:twistorcpx}:
	\begin{equation*}
		2\varphi(A_XY)=-[R^{\check\nabla}(\check X,\check Y),S]
	\end{equation*}
	Here, we identify $A_XY\in \mc V$ with an element of $S^2H$, or we may alternatively apply $\dif\varphi$. Furthermore, $\check\nabla$ is the Levi-Civit\`a connection on $M$ and $S=\varphi(z)$ is the point in which we are computing, regarded as an endomorphism. Setting $S=aI+bJ+cK$, \cref{eq:Icurvcommutator,eq:Jcurvcommutator,eq:Kcurvcommutator} and \cref{lem:ricciforms} show that
	\begin{equation*}
		2(n+2)\varphi(A_XY)=-\lambda\Big[(bK-cJ)g(I\check X,\check Y)
		+(cI-aK)g(J\check X,\check Y)
		+(aJ-bI)g(K\check X,\check Y)\Big]
	\end{equation*}
	where $\lambda\in\R$ is the Einstein constant. As long as $\lambda\neq 0$, the right-hand side is manifestly non-degenerate; it shows that $A_X(IX)$, $A_X(JX)$ and $A_X(KX)$ cannot simultaneously vanish.
\end{myproof}

In case the underlying quaternionic K\"ahler manifold has positive Einstein constant, we can say even more about the twistor space:

\begin{thm}[Salamon~\cite{Sal1982}, B\'erard-Bergery (unpublished)\footnote{See \cite[Thm. 14.80]{Bes2008} for a more precise reference.}]\label{thm:twistorKE}
	If $(M,g)$ is a quaternionic K\"ahler manifold with positive scalar curvature, then its twistor space $Z$ admits a K\"ahler-Einstein metric with positive scalar curvature, such that $\pi:Z\to M$ is a Riemannian submersion with totally geodesic fibers\footnote{See \cref{app:submersions} for a brief introduction to Riemannian submersions and related concepts.}.
\end{thm}
\begin{myproof}
	Since the base space $(M,g)$ is Einstein, we may rescale to obtain $r=(n+2)g$. As usual, we use the splitting $TZ=\mc H\oplus \mc V$ induced by the Levi-Civit\`a connection of $(M,g)$. Equip the fibers, which are copies of $\CP^1$, with the Fubini-Study metric (with constant sectional curvature equal to one), and define the metric $\tilde g$ on $Z$ to agree with it on vertical tangent vectors. 
	
	Furthermore, declare horizontal and vertical tangent vectors to be orthogonal with respect to $\tilde g$ and define the metric on horizontal tangent vectors $X,Y$ via $\tilde g(X,Y)=\pi^*g(X,Y)$, so that $(Z,\tilde g)$ becomes a Riemannian submersion; the fibers are automatically totally geodesic. This turns $Z$, equipped with its natural complex structure $J$, into a Hermitian manifold (by definition of $J$). Now, we must verify that this defines a K\"ahler-Einstein metric. In order to do so, we need another computational lemma:
	
	\begin{lem}
		Let $U$ be vertical and $X,Y$ basic vector fields on $Z\subset S^2H$. Let $A$ be O'Neill's $A$-tensor. Then:
		\begin{numberedlist}
			\item $\nabla_U (JX)=JA_XU$.
			\item $JA_XU=A_X(JU)$ and $JA_XY=A_X(JY)$.
			\item $\tilde g(A_X, A_X)=\frac{1}{2}\tilde g(X,X)$ and $\tilde g(AU,AU)=n\tilde g(U,U)$, where $n=\dim M/4$.
		\end{numberedlist}
	%WORK NEEDED: The signs and factors in part 1 and 3 are sketchy
		For the relevant definitions, see \cref{app:submersions}.
	\end{lem}
	
	We have omitted the (lengthy) proof, which consists of repeated application of a number of identities: $2\dif\varphi(A_XY)=-[R^{\check\nabla}(\check X,\check Y),S]$ and $\dif\pi(\mc H[JX,U])=-S\check X$, the \crefrange{eq:Icurvcommutator}{eq:Kcurvcommutator}, the fact that $\dif\pi(JX)=S\dif\pi(X)$ for any horizontal $X$ and the relation $r=(n+2)g$ between the Ricci and metric tensors of $M$.
%	\begin{myproof}[Proof of Lemma]\leavevmode
%		\begin{numberedlist}
%			\item We compute both sides of the equation. Since O'Neill's $T$-tensor vanishes (i.e. the fibers are totally geodesic), $\nabla_U(JX)$ is horizontal and we may compute $\dif \pi (\nabla_U (JX))$ instead. Recall the identity $\dif\pi(\mc H[JX,U])=\dif\varphi(U)\check X$, where $\check X=\dif\pi (X)$, from the proof of \cref{thm:twistorcpx}. Since the subbundle spanned by $I$, $J$ and $K$ is preserved by covariant differentiation, $\mc H(\nabla_{JX}U)=0$ 
%			\begin{equation*}
%				\dif\pi(\nabla_U (JX))=-\dif\varphi(U)\check X
%				+\dif\pi(\mc H\nabla_{JX}U)=-(xI+yJ+zK)\check X
%			\end{equation*}
%			where we set $\dif\varphi(U)=xI+yJ+zK$.
%			
%			Now we rewrite the right-hand side. We have
%			\begin{align*}
%				\tilde g(A_XU,Y)=-\tilde g(U,A_XY)
%				&=-g(\dif\varphi(U),\dif\varphi(A_XY))\\
%				&=\frac{1}{2}g(xI+yJ+zK,[R(\check X,\check Y),S])
%			\end{align*}
%			where we used that $\varphi$ is an isometry ($g$ is the induced metric on $\End TM$) as well as the same identity that appeared in the proof of the previous theorem. Since the Einstein constant of $(M,g)$ is $n+2$, we have
%			\begin{equation*}
%				[R(\check X,\check Y),S]=(bK-cJ)g(I\check X,\check Y)
%				+(cI-aK)g(J\check X,\check Y)+(aJ-bI)g(K\check X,\check Y)
%			\end{equation*}
%			where we set $S=aI+bJ+cK$. We deduce that
%			\begin{equation*}
%				\tilde g(A_XU,Y)
%				=(zb-yc)g(I\check X,\check Y)+(xc-za)g(J\check X,\check Y)+(ya-xb)g(K\check X,\check Y)
%			\end{equation*}
%			and therefore
%			\begin{equation*}
%				\dif\pi(A_XU)=\big[(zb-yc)I+(xc-za)J+(ya-xb)K\big]\check X
%			\end{equation*}
%			Furthermore, $\dif\pi(J(A_XU))=S\dif\pi(A_XU)$, hence
%			\begin{equation*}
%				\dif\pi(J(A_XU))=(aI+bJ+cK)\dif \pi(A_XU)
%				=-(xI+yJ+zK)\bar X
%			\end{equation*}
%			where the last step follows from a short computation, using the fact that $ax+by+cz=0$, as this sum geometrically represents the inner product of the unit normal to $S^2\subset \R^3$ with a tangent vector.
%			\item The skew-symmetry of the $A$-tensor implies that the two claims are equivalent. To see that $A_X(JU)=JA_X U$, use the fact that $\dif\varphi(JU)=S\dif\varphi(U)$ to see that
%			\begin{equation*}
%				\tilde g(A_X(JU),Y)=\frac{1}{2}g(S\dif\varphi(U),[R(\check X,\check Y),S])
%			\end{equation*}
%			From here, the same arguments as for the previous claim show that 
%			\begin{equation*}
%				\dif\pi(A_X(JU))=-(xI+yJ+zK)\check X=\dif\pi(JA_X U)
%			\end{equation*}
%			where we once again set $\dif\varphi(U)=xI+yJ+zK$, and the last equality was obtained while establishing the previous part.
%			\item To prove the first assertion, we pick a local orthonormal basis $\{U_1,U_2\}$ of $\mc V$ which corresponds to the standard (positively oriented) basis of $TS^2\subset T\R^3$ in each fiber. Then $A_X U_1=\frac{1}{2}U_2\check X$ and $A_X U_2=-\frac{1}{2}U_1\check X$. Using the fact that $U_1,U_2$ are compatible with $g$ and the fact that $\tilde g\big|_{\mc H}=\pi^*g$, we find
%			\begin{align*}
%				\tilde g(A_X,A_X)&=\sum_{j=1}^2\tilde g(A_X U_j,A_X U_j)
%				=\frac{1}{4}g(U_1 \check X,U_1 \check X) + \frac{1}{4}g(U_2 \check X, U_2 \check X)\\
%				&=\frac{1}{2}\tilde g(X,X)
%			\end{align*}
%			as claimed. The second claim is proven in identical fashion.
%		\end{numberedlist}
%	\end{myproof}
%WORK NEEDED: This tedious proof... Include or exclude? The factors are iffy.

	Now, we will prove that $\tilde\nabla J=0$. If $U,V$ are vertical then $(\tilde\nabla_U J)V=0$ since the fibers are K\"ahler. Now let $X$ be a horizontal vector. Since O'Neill's $T$-tensor vanishes, $\mc V\tilde\nabla_U X=0$ and since we may extend $X$ to a basic vector field, $\mc H\nabla_U X=\mc H\tilde\nabla_X U=A_XU$. Thus, $(\tilde\nabla_UJ)X=\tilde\nabla_U(JX)-J\tilde\nabla_U X=0$ by our lemma. 
	
	Now, we prove that $(\tilde\nabla_XJ)U=0$. Note that $\mc V(\tilde\nabla_XU)=\mc V(\tilde\nabla_U X+[X,U])=\mc V[X,U]=[X,U]$ since $X$ is basic. Similarly, $\mc V(\tilde\nabla_X (JU))=\mc V([X,JU]+\tilde\nabla_{JU}X)=[X,JU]=J[X,U]$. This proves that $\mc V(\tilde\nabla_X J)U=0$. The fact that $\mc H(\tilde\nabla_XJ)U=A_X(JU)-J(A_X U)=0$ follows from the second claim of the lemma.
	
	Finally, we consider $(\tilde\nabla_XJ)Y$. Its vertical projection vanishes due to part two of the lemma (as above), while $\mc H(\tilde\nabla_XJ)Y$ vanishes by an argument analogous to what we used to prove that $\mc HN_J(\mc H,\mc H)=0$ in the proof of \cref{thm:twistorcpx}. This completes the proof that $(Z,\tilde g,J)$ is K\"ahler.
	
	To show that $(Z,\tilde g)$ is Einstein, we use the criteria provided by \cref{prop:totallygeodesicEinstein} and see that the Ricci curvature $\tilde r$ satisfies $\tilde r=(n+1)\tilde g$.
\end{myproof}

Recall that a compact K\"ahler manifold with positive first Chern class is called a \emph{Fano manifold} and the divisibility of its first Chern class is known as the Fano index. Here, a positive cohomology class is one that may be represented by a real $(1,1)$-form $\omega$ such that for every $v\in T^{1,0}_\C X$, $-i\omega(v,\bar v)>0$. The archetypal example is the K\"ahler class associated to a K\"ahler metric.

The existence of a K\"ahler-Einstein metric with positive scalar curvature on the twistor space implies that it is Fano. To see this, recall that on a K\"ahler manifold $(X,g,J)$ the isomorphism of complex vector bundles $(TX,J)\cong T^{1,0}_\C X$ (where the latter underlies the holomorphic tangent bundle $\mc TX$) identifies the Chern connection $\nabla$ on $\mc TX$ with the Levi-Civit\`a connection $D$ on $TX$, as well as the corresponding curvature tensors $F_\nabla$ and $R$. 

Defining the \emph{Ricci form} $\rho(X,Y)\coloneqq r(JX,Y)$, a standard computation with respect to a local frame of $TX$, $\{x_1,Jx_1,\dots,x_n,Jx_n\}$, shows that $\rho=i\tr_\C (F_\nabla)$ (where $\tr_\C$ traces over the endomorphism-part of $F_\nabla$). 
%WORK NEEDED: What's going on with these signs?!
Using Chern-Weil theory, we see that $\frac{1}{2\pi}\rho$ represents the first Chern class $c_1(X)$. The fact that the Ricci curvature is proportional to the metric with positive constant of proportionality means that $\rho$ is a positive form. Thus, $c_1(X)$ is positive and $X$ is Fano.

\begin{cor}\label{cor:twistorfano}
	If $(M^{4n},g)$ is quaternionic K\"ahler with positive scalar curvature, then its twistor space $Z$ is a Fano manifold with Fano index a multiple of $n+1$.
\end{cor}
\begin{myproof}
	By virtue of \cref{thm:twistorKE} and the above discussion, $Z$ is Fano. The proof of \cref{thm:twistorctc} shows that it carries a holomorphic contact structure, defined by the projection onto the line bundle $\mc V$. By \cref{prop:cpxctcdivisible}, $c_1(Z)=(n+1)c_1(\mc V)$. Thus, $c_1(Z)$ is divisible by $n+1$.
\end{myproof}

%WORK NEEDED: Nearly Kähler structure!!

%WORK NEEDED: Also mention various other concepts of twistor spaces, including the harmonic one from e.g. Salamon and Wood & Svensson