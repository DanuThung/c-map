\chapter{Introduction}

This thesis concerns certain manifolds that fall into the special class of homogeneous spaces known as \emph{generalized flag manifolds}. In the 1950's, it was realized that generalized flag manifolds possess remarkable properties, particularly from the point of view of complex geometry. Indeed, they are distinguished among compact, simply connected homogeneous spaces by the fact that they carry an invariant complex structure which even admits a compatible, invariant K\"ahler-Einstein metric. 

Other interesting geometric structures on generalized flag manifolds include other Einstein metrics, and almost complex structures (not necessarily integrable), which were studied by Borel and Hirzebruch in a famous series of papers. They gave a method to compute the corresponding Chern numbers and pointed out that, in some examples, the Chern numbers distinguish different invariant almost complex structures on a flag manifold. This phenomenon relates to the question of which sets of Chern numbers can be realized on a single smooth manifolds, which is of independent interest.

It is natural to study these invariant structures by means of Lie theory, leveraging the homogeneity to reduce geometric questions to algebraic ones. This has historically been the most popular approach. However, in taking it one relinquishes the use of geometric intuition, making it harder to give a concrete interpretation of the invariant geometric structures. This motivates the complementary approach taken in this work: We study certain examples of generalized flag manifolds, namely those which are homogeneous under the exceptional Lie group $G_2$, from a \emph{geometric} point of view. Avoiding the use of Lie theory, we rely on differential-geometric methods instead. 

In developing our geometric understanding of these spaces, we highlight the various branches of (almost) complex and Riemannian geometry that play a role in the study of generalized flag manifolds. In the process, we uncover surprising connections to a variety of topics ranging from the existence of complex structures on the six-sphere to rigidity theorems for K\"ahler manifolds. Our methods enable us to recover, and give an interpretation of, all the invariant almost complex structures---including the invariant K\"ahler-Einstein metric---of the manifolds we study. We then use our geometric description to compute the corresponding Chern numbers without appealing to Lie theory.

The presence of several interesting geometric structures, which interact in non-trivial ways, ensures that techniques from many different areas of mathematics find application in the study of generalized flag manifolds. By means of our detailed exposition of a few examples, we hope to convey some of its beauty to the reader.

In the first three chapters, we discuss background material. \Cref{chap:homogeneous} contains some basic information on homogeneous spaces and invariant geometric structures, while \cref{chap:twistor} is an exposition of the fundamentals of the theory of quaternionic K\"ahler manifolds, with emphasis on the associated twistor spaces. In \cref{chap:uniqueness}, we review classical rigidity theorems for the complex projective spaces and quadric hypersurfaces, in anticipation of a result that appears in \cref{chap:invarstructures}. The last two chapters are dedicated to the study of $G_2$ flag manifolds, which we introduce after discussing octonionic linear algebra in \cref{chap:flags}. The sixth and final chapter contains our main results, namely computations of the Chern numbers associated to invariant almost complex structures, as well as a rigidity theorem for one of the manifolds under consideration. 

Naturally, our choices regarding which pieces of background material to include and which to leave out reflect the prior knowledge of the author. Thus, we do not assume much background in Riemannian geometry beyond an introductory course, but nevertheless expect the reader to be familiar with the fundamentals of complex geometry and the theory of characteristic classes, as well as algebraic topology. We intend this work to be readable for geometrically-minded graduate students. 

\subsubsection{Acknowledgments}

I am greatly indebted to professor Kotschick, my supervisor, for always guiding me in the right direction, answering my questions, and shaping the way I think about mathematics. I am also grateful to Rui Coelho, who helped me get unstuck during many a tricky calculation.

I would like to thank my office mates (and our frequent office guests!), and especially Anthony, for discussions about typographical nitpicks and other daily distractions.

Finally, I want to express my gratitude to my dear friend Alex, for all the time we spent and the things we learned together. The past three years wouldn't have been the same without you.